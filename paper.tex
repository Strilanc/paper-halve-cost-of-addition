\documentclass[twocolumn,longbibliography]{quantumarticle-customized}
\usepackage{amsmath}
\usepackage{graphicx}
\usepackage[pdfpagelabels,pdftex,bookmarks,breaklinks]{hyperref}
\usepackage[all]{hypcap}
\hypersetup{colorlinks,citecolor=blue,urlcolor=blue,linkcolor=blue}

\title{Saving $T$ gates by reusing $|A\rangle$ states}
\author{Craig Gidney}
\affiliation{Google, Santa Barbara, CA 93117, USA}
\email{craiggidney@google.com}

\def\sectionautorefname{Section}

\begin{document}
\maketitle

\begin{abstract}
A temporary Toffoli gate (sometimes called ``phase-congruent" or ``matched") must apply the correct state permutation, but may incorrectly phase or entangle the involved qubits until a later Toffoli gate fixes these errors while uncomputing the permutation.
We improve the number of T gates needed to perform temporary Toffoli gates from 4 to 3 via non-destructive use of an ancilla in the state $|A\rangle = \frac{1}{\sqrt{2}} |0\rangle + \frac{1+i}{2} |1\rangle$.

Many common circuits use temporary Toffoli gates, and $T$ gates dominate the cost of error corrected quantum computing, so this optimization is significant and widely applicable.
For example, we reduce the T-cost of $n$-bit addition from $8n + O(1)$ to $6n + O(1)$ which in turn reduces the overall cost of Shor's factoring algorithm by nearly 25\%.
\end{abstract}

\section{Introduction} \label{sec:introduction}

When no ancillae are available, a Toffoli gate requires seven T gates to perform [[[cite]]].
If ancillae are available, that cost can be reduced to four [[[[cite]]] by doing a measurement teleportation thing.
When a Toffoli gate is applied temporarily, in a fashion where phase errors are acceptable until the Toffoli is uncomputed, a simpler inline trick suffices for achieving 4 T gates per Toffoli.




See \autoref{fig:inline-toffoli}.

\begin{figure}
  \centering
  \makebox[\linewidth]{
  }
  \caption{
	Simple Toffoli construction with 7 T gates.
	Doesn't require any ancillae or a later paired Toffoli.
  }
  \label{fig:inline-toffoli}
\end{figure}

\begin{figure}
  \centering
  \makebox[\linewidth]{
  }
  \caption{
	Toffoli construction with 4 T gates based on measurement teleportation.
  }
  \label{fig:ancilla-toffoli}
\end{figure}

\begin{figure}
  \centering
  \makebox[\linewidth]{
  }
  \caption{
	Existing temporary Toffoli construction with 4 T gates.
  }
  \label{fig:inline-temporary-toffoli}
\end{figure}

\begin{figure}
  \centering
  \makebox[\linewidth]{
  }
  \caption{
	Our temporary Toffoli construction, which uses an $|A\rangle$ ancilla and 3 T gates.
  }
  \label{fig:ancilla-temporary-toffoli}
\end{figure}

\begin{figure}
  \centering
  \makebox[\linewidth]{
  }
  \caption{
	$n$-bit inline addition with T-cost of $8n + O(1)$ based on [[[cite]]].
  }
  \label{fig:inline-addition}
\end{figure}

\begin{figure}
  \centering
  \makebox[\linewidth]{
  }
  \caption{
	$n$-bit addition with T-cost of $6n + O(1)$.
  }
  \label{fig:ancilla-addition}
\end{figure}

\begin{figure}
  \centering
  \makebox[\linewidth]{
  }
  \caption{
	Temporary addition construction.
  }
  \label{fig:temporary-ancilla-addition}
\end{figure}

Mention binary to unary as an example?





\section{Acknowledgements}

Austin.


\bibliographystyle{plain}
\bibliography{citations}

\cite{barenco1995}
https://arxiv.org/pdf/quant-ph/9503016.pdf   ->   the paired Toffoli construction that uses 4T

https://arxiv.org/pdf/cond-mat/9409111.pdf   -> shows CCNOT can be done with three 2-qubit interactions, ignores single-qubit???


https://arxiv.org/pdf/1308.4134.pdf -> proof of 7T gate minimum with no ancilla

https://arxiv.org/pdf/1212.5069.pdf  ->   does a single Toffoli with four T gates by turning one of the controls classical

https://arxiv.org/pdf/1210.0974.pdf  ->   a phase-neglected 4T gate construction that only uses a single layer of T gates


Quantum Computation and Quantum Information -> shows standard decomposition into 7T gates. But do they say it's optimal?

https://arxiv.org/pdf/1206.0758.pdf -> just an example of the 7T construction;;; probablynot relevant

\end{document}
