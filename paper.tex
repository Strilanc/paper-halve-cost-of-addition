\documentclass[twocolumn,longbibliography]{quantumarticle-customized}
\usepackage{amsmath}
\usepackage{graphicx}
\usepackage[pdfpagelabels,pdftex,bookmarks,breaklinks]{hyperref}
\usepackage{tikz}
\usepackage[all]{hypcap}
\hypersetup{colorlinks,citecolor=blue,urlcolor=blue,linkcolor=blue}

\input{Qcircuit}
\newcommand{\qH}{\gate{H}}
\newcommand{\qT}{\gate{T}}
\newcommand{\qTi}{\gate{T^\dagger}}
\newcommand{\qS}{\gate{S}}
\newcommand{\qSi}{\gate{S^\dagger}}
\newcommand{\qA}{\lstick{|A\rangle}}
\newcommand{\qO}{\lstick{|0\rangle}}

\title{T gates, temporary ANDs, and halving the cost of quantum addition}
\author{Craig Gidney}
\affiliation{Google, Santa Barbara, CA 93117, USA}
\email{craiggidney@google.com}

\def\sectionautorefname{Section}

\begin{document}
\maketitle

\begin{abstract}
We improve the number of T gates needed to perform an $n$-bit adder from $8n + O(1)$ \cite{Amy2013, AustinDiscussionsAndEmails2017} to $4n + O(1)$.
We do so via a ``temporary AND gate" construction, which uses four T gates to store the logical-and of two qubits into an ancilla and no T gates to later erase the ancilla.

Temporary AND gates are a useful tool when creating other low T cost circuits.
They can be applied to other integer arithmetic, modular arithmetic, Shor's algorithm, the quantum Fourier transform, Grover oracles, rotation synthesis, and many other circuits.
Because T gates dominate the cost of quantum computation based on the surface code, and the temporary AND gate is widely applicable, our constructions represent a significant reduction in projected costs of quantum computation.

In addition to our $n$-bit adder circuit with T-cost of $4n-4$, we construct an $n$-bit controlled adder circuit with T-cost of $8n-4$, a temporary adder circuit that can be computed for the same cost as the normal adder but whose result can be kept until later uncomputed at no T-cost, an approximate quantum Fourier transform circuit with T-cost of $8 n \lg b + O(b (\lg n) (\lg b))$ where $b = \lg \frac{1}{\epsilon}$ is the desired bit-precision of the operation, and a circuit to rotate $n$ qubits by any angle $\theta$ with T-cost of $4n + O(b \lg n)$.
\end{abstract}


\section{Introduction}
\label{sec:introduction}

The surface code is a quantum error correcting code that works on a 2D nearest-neighbour array of qubits and achieves a threshold error rate of approximately 1\% [[[[cite so much]]]].
This makes the surface code a likely component in the architecture of future error corrected quantum computers, because 2D arrays of qubits with nearest-neighbor connections are possible with many qubit technologies [[[[cite so much]]]] and other known error correcting codes have lower thresholds or require stronger connectivity [[[[cite so much]]]].

One of the downsides of the surface code is that it has no cheap mechanism to apply non-Clifford operations such as $T$ gates.
Instead, $T$ gates are performed by distilling and consuming $|A\rangle = \frac{1}{\sqrt{2}} (|0\rangle + e^{i \pi/4} |1\rangle)$ states.
Consuming an $|A\rangle$ state to perform a T gate is simple (see \autoref{fig:t-equals-a}), but distilling $|A\rangle$ states has significant cost.

\begin{figure}
  \resizebox{\linewidth}{!}{
    \Qcircuit @R=0.7em @C=0.7em {
      &\gate{T}&\qw &&=&&&          &&\qw &\ctrl{1}&\qw    &\qS            &\qw &\\
      &        &    && &&&|A\rangle &&\qw &\targ   &\meter &\cw\cwx\bullet &    &\\
    }
  }
  \caption{
	Performing a T gate with Clifford operations by consuming an $|A\rangle$ state, where $|A\rangle = \frac{1}{\sqrt{2}} (|0\rangle + e^{i \pi/4} |1\rangle$.
  }
  \label{fig:t-equals-a}
\end{figure}

In [[[[cite]]]] the $|A\rangle$ state distillation is done by preparing an EPR pair $(P, Q)$, translating $P$ out of the surface code and into a different error correcting code that can perform $T$ gates, and using a $T$ gate within that code to rotate and measure $P$ along an axis correlated with the diagonal $X+Y$ axis of $Q$.
If this process succeeds without an error being detected, it steers \cite{Wiseman2007} $Q$ into the $|A\rangle$ state or else into the complementary state $Z|A\rangle$.
The $Z$ rotation is easily corrected, and the result of measuring $P$ indicates whether the correction is needed or not.

The resulting `foothold' $|A\rangle$ states have a non-negligible amount of noise, but can be used to perform approximate T gates powering additional rounds of distillation occurring entirely within the surface code.
After accounting for the cost of distilling a sufficiently accurate $|A\rangle$ state, the spacetime volume cost of performing a T gate is one to two orders of magnitude higher than performing operations native to the surface code (such as the CNOT gate) [[[[cite]]]].

Because T gates are so expensive for the surface code, and the surface code is a likely component of future quantum computers, it is important to consider and optimize the number of T gates used by quantum circuits.

In this paper, we improve the number of T gates needed to perform Toffoli gates that will later be uncomputed by a second Toffoli gate.
Our construction involves changing the Toffoli gates' target to a clean ancilla qubit.
This makes the function our construction similar to classical AND gates, which produce a new output instead of toggling an existing value.
As such, we refer to our construction as a "temporary AND gate" ( as opposed to e.g. ``temporary Toffoli gate" or ``matched Toffoli gate pair").

Toffolis in compute/uncompute pairs appear in many circuits, especially ones that involve computing and uncomputing classical functions, so our temporary AND gate construction is widely applicable.

Our paper is divided into six sections.
In \autoref{sec:introduction}, we motivate the problem and insert self-referential descriptions.
\autoref{sec:review} discusses how previous work on optimizing the T gate counts of Toffolis improved the cost of temporary Toffolis from 14 to 8.
In \autoref{sec:invest}, we show how to improve the cost of temporary Toffolis from 8 to 6 by investing and later recovering an $|A\rangle$ state instead of consuming it to perform T gates.
\autoref{sec:temporary-and} explains an ancilla erasing trick that improves the cost of temporary Toffolis further, from 6 to 4.
This leads to \autoref{sec:circuit-constructions}, where we demonstrate how to use our constructions to improve the T-cost of several basic arithmetic tasks (e.g. halving the T-cost of addition).
Finally, \autoref{sec:conclusion} concludes and discusses future work.


\section{Previous Work}
\label{sec:review}

The textbook construction of a Toffoli gate uses seven T gates \cite{Nielsen2009} (see \autoref{fig:textbook-toffoli}).
Assuming we aren't permitted to involve other qubits or to share work with other operations, this construction is optimal \cite{Gosset2014}.
Of course, in practice, we can do both.
For example, when several adjacent Toffoli gates share the same controls, all but one can be replaced by CNOT operations (see \autoref{fig:shared-controls}).

\begin{figure}
  \resizebox{\linewidth}{!}{
    \Qcircuit @R=1.5em @C=0.7em {
      &\ctrl{1}&\qw & &   & & &\ctrl{1}&\qw  &\ctrl{1}&\qT &\qw     &\qw  &\ctrl{2}&\qw &\qw     &\qw  &\ctrl{2}&\qw &\qw \\
      &\ctrl{1}&\qw & & = & & &\targ   &\qTi &\targ   &\qT &\ctrl{1}&\qw  &\qw     &\qw &\ctrl{1}&\qw  &\qw     &\qw &\qw \\
      &\targ   &\qw & &   & & &\qw     &\qw  &\qH     &\qT &\targ   &\qTi &\targ   &\qT &\targ   &\qTi &\targ   &\qH &\qw \\
    }
  }
  \caption{
	Textbook Toffoli construction from \cite{Nielsen2009}.
	Uses eight Clifford gates and seven T gates.
  }
  \label{fig:textbook-toffoli}
\end{figure}

\begin{figure}
  \resizebox{\linewidth}{!}{
    \Qcircuit @R=1.5em @C=0.7em {
      &\ctrl{1}&\ctrl{1}&\ctrl{1}&\ctrl{1}&\qw && &&&\qw     &\qw     &\qw     &\ctrl{1}&\qw     &\qw     &\qw     &\qw &\\
      &\ctrl{1}&\ctrl{2}&\ctrl{3}&\ctrl{4}&\qw &&=&&&\qw     &\qw     &\qw     &\ctrl{1}&\qw     &\qw     &\qw     &\qw &\\
      &\targ   &\qw     &\qw     &\qw     &\qw && &&&\ctrl{3}&\ctrl{2}&\ctrl{1}&\targ   &\ctrl{1}&\ctrl{2}&\ctrl{3}&\qw &\\
      &\qw     &\targ   &\qw     &\qw     &\qw && &&&\qw     &\qw     &\targ   &\qw     &\targ   &\qw     &\qw     &\qw &\\
      &\qw     &\qw     &\targ   &\qw     &\qw && &&&\qw     &\targ   &\qw     &\qw     &\qw     &\targ   &\qw     &\qw &\\
      &\qw     &\qw     &\qw     &\targ   &\qw && &&&\targ   &\qw     &\qw     &\qw     &\qw     &\qw     &\targ   &\qw &\\
    }
  }
  \caption{
	The T-cost of $N$ adjacent Toffolis sharing the same controls is $0 \cdot N + O(1)$.
	The marginal T-cost is 0 because each additional Toffoli can be replaced with CNOTs framing a root Toffoli.
  }
  \label{fig:shared-controls}
\end{figure}

It isn't common for adjacent Toffolis to have the same controls, but it is common for a Toffoli to later be uncomputed by a second matching Toffoli (i.e. for the Toffoli's effect to be temporary).
When this occurs, the three $T$ gates on the control qubits of the textbook construction can be omitted.
This introduces phase errors (see \autoref{fig:bad-phase-toffoli}), but the second Toffoli gate can uncompute those errors while uncomputing the state permutation \cite{Barenco1995} (see \autoref{fig:cancelled-bad-phase-toffoli}).

\begin{figure}
  \resizebox{\linewidth}{!}{
    \Qcircuit @R=1.5em @C=0.7em {
      &\ctrl{1} &\qw & &       & & &\ctrl{1}     &\ctrl{1} &\qw & &   & & &\qw &\qw &\qw      &\qw  &\ctrl{2} &\qw &\qw      &\qw  &\qw &\qw \\
      &\ctrl{1} &\qw & & \cong & & &\qS          &\ctrl{1} &\qw & & = & & &\qw &\qw &\ctrl{1} &\qw  &\qw      &\qw &\ctrl{1} &\qw  &\qw &\qw \\
      &\targ    &\qw & &       & & &\gate{Z}\qwx &\targ    &\qw & &   & & &\qH &\qT &\targ    &\qTi &\targ    &\qT &\targ    &\qTi &\qH &\qw \\
    }
  }
  \caption{
	Starting with \autoref{fig:textbook-toffoli} then dropping T gates on the controls produces an operation with a T-cost of 4 that performs the correct permutation.
	However, the operation introduces phase errors.
  }
  \label{fig:bad-phase-toffoli}
\end{figure}

\begin{figure}
  \resizebox{\linewidth}{!}{
    \Qcircuit @R=1.5em @C=0.7em {
      &\ctrl{1}&\qw &\qw &\qw    &\qw &\qw &\ctrl{1}&\qw && &&&\ctrl{1}     &\ctrl{1}  &\qw &\qw &\qw    &\qw &\qw     &\ctrl{1}&\ctrl{1}     &\qw & \\
      &\ctrl{1}&\qw &\qw &\qw    &\qw &\qw &\ctrl{1}&\qw &&=&&&\qS          &\ctrl{1}  &\qw &\qw &\qw    &\qw &\qw     &\ctrl{1}&\qSi         &\qw & \\
      &\targ   &\qw &    &\ldots &    &    &\targ   &\qw && &&&\gate{Z}\qwx &\targ     &\qw &    &\ldots &    &        &\targ   &\gate{Z}\qwx &\qw & \\
    }
  }
  \caption{
	When two Toffolis form a compute/uncompute pair, they can cancel each others' phase errors.
	Fixes the problem with the construction in \autoref{fig:bad-phase-toffoli}, and achieves a per-Toffoli T-cost of 4 for paired Toffolis \cite{Barenco1995}.
  }
  \label{fig:cancelled-bad-phase-toffoli}
\end{figure}

In \cite{Jones2013}, Jones shows how to perform an {\em unpaired} Toffoli with four T gates.
They use an ancilla qubit and an erasure technique that turns what would have been a quantum control into a classical control.
They also re-arrange the remaining T gates into a single column, creating a circuit with T-depth of 1.
We show this construction in \autoref{fig:jones-toffoli}.

\begin{figure}
  \resizebox{\linewidth}{!}{
    \Qcircuit @R=1.5em @C=0.7em {
      &\ctrl{1} &\qw & &   & & &     &\qw &\qw &\qw      &\qw  &\ctrl{3} &\qw &\qw      &\qw  &\qw &\qw  &\qw       &\qw &\qw    &\ctrl{1}       &\qw \\
      &\ctrl{1} &\qw & & = & & &     &\qw &\qw &\ctrl{2} &\qw  &\qw      &\qw &\ctrl{2} &\qw  &\qw &\qw  &\qw       &\qw &\qw    &\gate{Z}       &\qw \\
      &\targ    &\qw & &   & & &     &\qw &\qw &\qw      &\qw  &\qw      &\qw &\qw      &\qw  &\qw &\qw  &\targ     &\qw &\qw    &\qw\cwx        &\qw \\
      &         &    & &   & & & \qO &\qH &\qT &\targ    &\qTi &\targ    &\qT &\targ    &\qTi &\qH &\qSi &\ctrl{-1} &\qH &\meter &\cw\cwx\bullet & \\
    }
  }
  \caption{
	Toffoli construction with T-cost of 4 from \cite{Jones2013}.
  }
  \label{fig:jones-toffoli}
\end{figure}

By using Jones' construction twice (or the phase-correct-by-pairing construction), a temporary Toffoli can be computed with eight T gates \cite{Amy2013}.
Four T gates for the initial Toffoli, and four T gates to uncompute its effect later with another Toffoli.


\section{Investing $|A\rangle$ states}
\label{sec:invest}

The seed for this paper was an idea that accidentally improved the T-cost of a Toffoli from 7 to 6 in a surprising way.
Instead of performing a Toffoli directly onto the target qubit, we performed the Toffoli indirectly via an ancilla.
We started with a clean ancilla in the $|0\rangle$ state, applied a Toffoli to store the logical-and of the two controls in the ancilla, used the ancilla to control a CNOT onto the intended target, then uncomputed the ancilla.

Initially, this indirect-Toffoli construction appears to have a T-cost of 8.
However, as shown in \autoref{fig:indirect-toffoli}, the last T gate in the circuit is being used to uncompute an $|A\rangle$ state on the ancilla.
Dropping this T gate not only reduces the T-cost from 8 to 7, it recovers an $|A\rangle$ state that can be consumed to perform a T gate elsewhere.
This improves the net T-cost to 6.

\begin{figure}
  \resizebox{\linewidth}{!}{
    \Qcircuit @R=1.5em @C=0.7em {
      &\ctrl{1} &\qw && &&&    &\qw &\qw &\qw      &\qw  &\ctrl{3} &\qw &\qw      &\qw  &\qw &\qw       &\qw &\qw &\qw      &\qw  &\ctrl{3} &\qw &\qw      &\qw  &\qw &\qw &&\\
      &\ctrl{1} &\qw &&=&&&    &\qw &\qw &\ctrl{2} &\qw  &\qw      &\qw &\ctrl{2} &\qw  &\qw &\qw       &\qw &\qw &\ctrl{2} &\qw  &\qw      &\qw &\ctrl{2} &\qw  &\qw &\qw &&\\
      &\targ    &\qw && &&&    &\qw &\qw &\qw      &\qw  &\qw      &\qw &\qw      &\qw  &\qw &\targ     &\qw &\qw &\qw      &\qw  &\qw      &\qw &\qw      &\qw  &\qw &\qw &&\\
      &         &    && &&&\qO &\qH &\qT &\targ    &\qTi &\targ    &\qT &\targ    &\qTi &\qH &\ctrl{-1} &\qH &\qT &\targ    &\qTi &\targ    &\qT &\targ    &\qTi &\qH &\qw 
          \gategroup{4}{6}{4}{10}{.7em}{--} \gategroup{4}{26}{4}{30}{.7em}{--} &|0\rangle & \\
      &         &    && &&&    &    |A\rangle &  & &     &         &    &         &     &    &          &    &    &         &    &         &     &         &     & |A\rangle
    }
  }
  \caption{
	Performing a Toffoli indirectly, by applying a pair of Toffolis to an ancilla and using the intermediate value to control a CNOT onto the actual target, appears to have a T-cost of 8.
	However, the initial T gate is being used to compute an $|A\rangle$ state and the final T gate is being used to uncompute the $|A\rangle$ state.
    Instead of spending $|A\rangle$ states to perform T gates to make and unmake $|A\rangle$ states, we pass in and later recover a single $|A\rangle$ state.
	This ``investment" reduces the net T-cost to 6.
  }
  \label{fig:indirect-toffoli}
\end{figure}

A Toffoli with T-cost of 6 isn't optimal, but the reduction from 7 was done in an unexpected way.
While searching for other other circuits where this optimization might apply, we realized it would be useful for the paired-Toffolis case.
Instead of computing and uncomputing the ancilla for the first Toffoli, then recomputing and reuncomputing it for the second Toffoli, we simply skipped the first ancilla uncomputation and kept the ancilla until it was needed again.
This separated the construction into a control-combining step and an entanglement-erasing step.
Combining the controls had a net T-cost of 4, because it performed three $T$ gates and required investing an $|A\rangle$ state, but erasing the entanglement had a net T-cost of only 2 because although it performed three $T$ gates it recovered the invested $|A\rangle$ state.
See \autoref{fig:ancilla-temporary-toffoli}.

\begin{figure}
  \resizebox{\linewidth}{!}{
    \Qcircuit @R=1.5em @C=0.7em {
      &\ctrl{1}&\qw &\qw &\qw    &\qw &\qw &\ctrl{1}&\qw && &&&    &\qw     &\qw  &\ctrl{3}&\qw &\qw     &\qw  &\qw &\qw      &\qw &\qw &\qw    &\qw &\qw &\qw      &\qw &\qw &\qw     &\qw  &\ctrl{3}&\qw &\qw     &\qw &&\\
      &\ctrl{1}&\qw &\qw &\qw    &\qw &\qw &\ctrl{1}&\qw &&=&&&    &\ctrl{2}&\qw  &\qw     &\qw &\ctrl{2}&\qw  &\qw &\qw      &\qw &\qw &\qw    &\qw &\qw &\qw      &\qw &\qw &\ctrl{2}&\qw  &\qw     &\qw &\ctrl{2}&\qw &&\\
      &\targ   &\qw &    &\ldots &    &    &\targ   &\qw && &&&    &\qw     &\qw  &\qw     &\qw &\qw     &\qw  &\qw &\targ    &\qw &    &\ldots &    &    &\targ    &\qw &\qw &\qw     &\qw  &\qw     &\qw &\qw     &\qw &&\\
      &        &    &    &       &    &    &        &    && &&&\qA &\targ   &\qTi &\targ   &\qT &\targ   &\qTi &\qH &\ctrl{-1}&\qw &\qw &\qw    &\qw &\qw &\ctrl{-1}&\qH &\qT &\targ   &\qTi &\targ   &\qT &\targ   &\qw &|A\rangle &\\
    }
  }
  \caption{
	Computing and uncomputing a Toffoli gate with a net T-cost of 6.
	Invests an $|A\rangle$ state when computing the Toffoli, holds an ancilla qubit storing the logical-and of the controls until it's time to uncompute the Toffoli, then recovers the $|A\rangle$ state.
  }
  \label{fig:ancilla-temporary-toffoli}
\end{figure}

Many circuits involve computing and later uncomputing a Toffoli (e.g. addition), and investing an $|A\rangle$ state reduced the T-cost of doing so from 8 to 6.
We realized this represented an improvement over the state of the art but, while reviewing the literature, we found a better method for erasing the entanglement.
We mention investing $|A\rangle$ states here only because we expect that there are circuits that will benefit from it but not our improved technique.


\section{The temporary AND gate}
\label{sec:temporary-and}

In the previous section, we showed how to compute and uncompute the logical-AND of two controls by investing an $|A\rangle$ state and performing three $T$ gates.
This construction be improved in several ways.

First, the computation is introducing phase errors (as shown in \autoref{fig:bad-phase-toffoli}).
Normally, correcting these phase errors would require controlled-Z on the target (applied before the Toffoli) and a controlled-S between the two controls.
We know our target is $|0\rangle$ beforehand, so the controlled-Z has no effect and can be ignored.
The effect of the controlled-S gate is to apply a phase factor of $i$ to the amplitudes of computational basis states where both controls on.
In other words, the controlled-S phases the logical-and of the two controls.
We happen to have a qubit storing that value.
We can correct computation's phase error by applying an uncontrolled S gate to the ancilla storing the logical-and.

Second, the construction is unnecessarily deep.
Every T gate is associated with an S gate that will happen conditionally, based on the result of a measurement.
It takes time for the measurement result to become available, and it is beneficial to be waiting for several results in parallel instead of one after another.
We can reduce our measurement depth to 1 using the same technique as in \cite{Jones2013}.
We will continue to pass an $|A\rangle$ state into the circuit, instead of using that $|A\rangle$ state to perform a T gate, because doing so has no depth cost.

Third, taking another hint from \cite{Jones2013}, we can perform the uncomputation by using a measure-and-correct process.
To erase the ancilla produced by the logical AND, we use a technique we refer to as ``erasure".
We note that a Toffoli gate would obviously clear the value.
Then, since the target qubit can be discarded after clearing it, we are free to introduce a Hadamard gate and a measurement before discarding it.
We then hop the Hadamard over the Toffoli, transforming it into a CCZ operation.
The controls and targets of a CCZ are equivalent, so we rewrite the circuit into a CZC.
We then hop the Measurement over the CZC, turning a quantum control into a classical control.
These circuit moves turned the Toffoli gate into a CZ gate that we apply or not based on the outcome of the measreument
We ``Clifford-ized" the circuit by adding a Hadamard gate and a measurement.

With these three improvements applied, we have the three key ingredients of our temporary AND gate with a T-cost of 4: creating, using, and erasing.

In diagrams, we draw the AND gate as a wire emerging vertically from two controls then heading rightward, as shown in \autoref{fig:compute-logical-and}.
We draw the uncomputation analogously, with a wire storing the logical-and coming in from the left then merging vertically into the two controls that created it, as shown in \autoref{fig:erase-logical-and}.

\begin{figure}
  \resizebox{\linewidth}{!}{
    \Qcircuit @R=1.5em @C=0.7em {
      &x &&\ctrl{1} &\qw & x  && &&          &&\targ    &\ctrl{2}&\qw     &\qTi &\qw     &\ctrl{2}&\targ     &\qw &\qw &\qw &\\
      &y &&\ctrl{1} &\qw & y  &&=&&          &&\targ    &\qw     &\ctrl{1}&\qTi &\ctrl{1}&\qw     &\targ     &\qw &\qw &\qw &\\
      &  &&         &\qw & xy && &&|A\rangle &&\ctrl{-2}&\targ   &\targ   &\qT  &\targ   &\targ   &\ctrl{-2} &\qH &\qS &\qw &\\
    }
  }
  \caption{
	Computing the logical-and of two qubits, with a T-cost of 4 and a measure-react depth of 1.
  }
  \label{fig:compute-logical-and}
\end{figure}

\begin{figure}
  \resizebox{\linewidth}{!}{
    \Qcircuit @R=1.5em @C=0.7em {
      &x  &&\ctrl{1} &\qw &x && &&\qw &\qw    &\ctrl{1} &\qw \\
      &y  &&\ctrl{1} &\qw &y &&=&&\qw &\qw    &\gate{Z} &\qw \\
      &xy &&\qw      &    &  && &&\qH &\meter &\cw \cwx \bullet &    \\
    }
  }
  \caption{
	Uncomputing the logical-and of two qubits, with a T-cost of 0 and a measure-react depth of 1.
  }
  \label{fig:erase-logical-and}
\end{figure}

During the time that the logical-and ancilla wire exists, it can be used to control operations onto other qubits.
In effect, these operations are doubly-controlled by the inputs that produced the ancilla.
For example, we can perform a Toffoli by computing the logical-and of the two controls, applying a CNOT from the logical-and to the actual target, then uncomputing the logical-and.
This construction, shown in \autoref{fig:merge-use-erase-toffoli}, is equivalent to the one in \cite{Jones2013} (but broken into pieces that can be used independently).

\begin{figure}
  \resizebox{\linewidth}{!}{
    \Qcircuit @R=1.5em @C=0.7em {
      &\ctrl{1} &\qw && &&\ctrl{1} &\qw      &\ctrl{1} &\qw && &&          &&\targ    &\ctrl{2}&\qw     &\qTi &\qw     &\ctrl{2}&\targ     &\qw &\qw &\qw      &\qw &\qw    &\ctrl{1}         &\qw \\
      &\ctrl{2} &\qw && &&\ctrl{1} &\qw      &\ctrl{1} &\qw && &&          &&\targ    &\qw     &\ctrl{1}&\qTi &\ctrl{1}&\qw     &\targ     &\qw &\qw &\qw      &\qw &\qw    &\gate{Z}         &\qw \\
      &         &    &&=&&         &\ctrl{1} &\qw      &    &&=&&|A\rangle &&\ctrl{-2}&\targ   &\targ   &\qT  &\targ   &\targ   &\ctrl{-2} &\qH &\qS &\ctrl{1} &\qH &\meter &\cw \cwx \bullet &    \\
      &\targ    &\qw && &&\qw      &\targ    &\qw      &\qw && &&          &&\qw      &\qw     &\qw     &\qw  &\qw     &\qw     &\qw       &\qw &\qw &\targ    &\qw &\qw    &\qw              &\qw \\
    }
  }
  \caption{
	Performing a Toffoli gate by computing the logical-and of its controls, using the result, then erasing it.
	Has a T-cost of 4 and a measure-react depth of 2.
	Equivalent to the construction from \cite{Jones2013} shown in \autoref{fig:jones-toffoli}.
  }
  \label{fig:merge-use-erase-toffoli}
\end{figure}

The logical AND construction consumes four $|A\rangle$ states, so its T-cost is 4.


We represent erasing a logical AND with a wire terminating into two controls, as shown in \autoref{fig:erase-and} which also shows the underlying construction.
The T-cost of erasure is 0.

The benefit of separating the logical-and-creating operation from the logical-and-erasing operation is that, as long as we have the ancilla storing the logical AND around, we can Cliffordize any Toffoli gate whose controls would be the two qubits we merged by replacing the Toffoli with a CNOT with the ancilla qubit as a control.
When the parts are simply placed next to each other, we end up with a construction exactly equivalent to the ones presented by Jones \cite{Jones2013}.
See \autoref{fig:merge-use-erase-toffoli}.


In order to make savings, we need to keep the ancilla around for longer.
We need a situation like the one shown in \autoref{fig:paired-toffoli-to-logical-and}, where a single logical-AND computation replaces two Toffolis.

\begin{figure}
  \resizebox{\linewidth}{!}{
    \Qcircuit @R=1.5em @C=0.7em {
      &\ctrl{1} &\qw &\qw &\qw    &\qw &\qw &\ctrl{1} &\qw && &&\ctrl{1} &\qw      &\qw &\qw &\qw    &\qw &\qw &\qw      &\ctrl{1} &\qw & \\
      &\ctrl{2} &\qw &\qw &\qw    &\qw &\qw &\ctrl{2} &\qw && &&\ctrl{1} &\qw      &\qw &\qw &\qw    &\qw &\qw &\qw      &\ctrl{1} &\qw & \\
      &         &    &    &       &    &    &         &    &&=&&         &\ctrl{1} &\qw &\qw &\qw    &\qw &\qw &\ctrl{1} &\qw      &    & \\
      &\targ    &\qw &    &\ldots &    &    &\targ    &\qw && &&\qw      &\targ    &\qw &    &\ldots &    &    &\targ    &\qw      &\qw & \\
    }
  }
  \caption{
	Replacing a pair of compute/uncompute Toffolis with a temporary AND gate and Clifford operations.
	Improves the T-cost from 8 to 4.
  }
  \label{fig:paired-toffoli-to-logical-and}
\end{figure}

Fortunately, this pattern of distant Toffolis sharing controls is very common in quantum circuits.
It occurs during uncomputation, and also in circuits that sweep back and forth across qubits like addition.


\section{T-optimized Circuit Constructions}
\label{sec:circuit-constructions}

In this section we will show how to apply temporary Toffolis to improve the complexity of several circuits.

First, we note that existing adder constructions contain temporary Toffolis.
The T-cost of adding two $n$-bit numbers was thought to be $8n + O(1)$.
We replace the basic adder unit with the building block shown in \ref{fig:full-adder-block}, putting it together as shown in \ref{fig:multi-bit-adder-example} in order to create an adder with a T-cost of $4n + O(1)$.

\begin{figure}
  \resizebox{\linewidth}{!}{
    \Qcircuit @R=1.5em @C=0.7em {
      &c_k &&\ctrl{2}&\qw     &\ctrl{3}&\qw &\qw &\qw    &\qw &\qw &\qw &\qw    &\qw &\qw &\qw &\qw    &\qw &\qw &\ctrl{3}&\qw     &\ctrl{1}&\qw     &\qw &c_k     &&&&& \\
      &i_k &&\targ   &\ctrl{1}&\qw     &\qw &\qw &\qw    &\qw &\qw &\qw &\qw    &\qw &\qw &\qw &\qw    &\qw &\qw &\qw     &\ctrl{1}&\targ   &\ctrl{1}&\qw &i_k     &&&&& \\
      &t_k &&\targ   &\ctrl{1}&\qw     &\qw &\qw &\qw    &\qw &\qw &\qw &\qw    &\qw &\qw &\qw &\qw    &\qw &\qw &\qw     &\ctrl{1}&\qw     &\targ   &\qw &&&(t+i)_k &&& \\
      &    &&        &        &\targ   &\qw &    &c_{k+1}&    &    &    &\ldots &    &    &    &c_{k+1}&    &    &\targ   &\qw     &        &        &    &&&        &&& \\
    }
  }
  \caption{
	Adder building block with a T-cost of 4.
  }
  \label{fig:full-adder-block}
\end{figure}

\begin{figure}
  \resizebox{\linewidth}{!}{
    \Qcircuit @R=1.5em @C=0.7em {
      &i_0 &&\ctrl{1} &\qw     &\qw     &\qw     &\qw &\qw     &\qw     &\qw     &\qw &\qw     &\qw     &\qw     &\qw     &\qw     &\qw     &\qw     &\qw     &\qw     &\qw     &\qw     &\qw     &\qw     &\ctrl{1}&\ctrl{1}&\qw &i_0     &&&&&\\
      &t_0 &&\ctrl{1} &\qw     &\qw     &\qw     &\qw &\qw     &\qw     &\qw     &\qw &\qw     &\qw     &\qw     &\qw     &\qw     &\qw     &\qw     &\qw     &\qw     &\qw     &\qw     &\qw     &\qw     &\ctrl{1}&\targ   &\qw &&&(t+i)_0 &&&\\
      &    &&         &\ctrl{2}&\qw     &\ctrl{3}&\qw &\qw     &\qw     &\qw     &\qw &\qw     &\qw     &\qw     &\qw     &\qw     &\qw     &\qw     &\qw     &\qw     &\qw     &\ctrl{3}&\qw     &\ctrl{1}&\qw     &        &    &&&        &&&\\
      &i_1 &&\qw      &\targ   &\ctrl{1}&\qw     &\qw &\qw     &\qw     &\qw     &\qw &\qw     &\qw     &\qw     &\qw     &\qw     &\qw     &\qw     &\qw     &\qw     &\qw     &\qw     &\ctrl{1}&\targ   &\qw     &\ctrl{1}&\qw &i_1     &&&&&\\
      &t_1 &&\qw      &\targ   &\ctrl{1}&\qw     &\qw &\qw     &\qw     &\qw     &\qw &\qw     &\qw     &\qw     &\qw     &\qw     &\qw     &\qw     &\qw     &\qw     &\qw     &\qw     &\ctrl{1}&\qw     &\qw     &\targ   &\qw &&&(t+i)_1 &&&\\
      &    &&         &        &        &\targ   &\qw &\ctrl{2}&\qw     &\ctrl{3}&\qw &\qw     &\qw     &\qw     &\qw     &\qw     &\qw     &\qw     &\ctrl{3}&\qw     &\ctrl{1}&\targ   &\qw     &        &        &        &    &&&        &&&\\
      &i_2 &&\qw      &\qw     &\qw     &\qw     &\qw &\targ   &\ctrl{1}&\qw     &\qw &\qw     &\qw     &\qw     &\qw     &\qw     &\qw     &\qw     &\qw     &\ctrl{1}&\targ   &\qw     &\qw     &\qw     &\qw     &\ctrl{1}&\qw &i_2     &&&&&\\
      &t_2 &&\qw      &\qw     &\qw     &\qw     &\qw &\targ   &\ctrl{1}&\qw     &\qw &\qw     &\qw     &\qw     &\qw     &\qw     &\qw     &\qw     &\qw     &\ctrl{1}&\qw     &\qw     &\qw     &\qw     &\qw     &\targ   &\qw &&&(t+i)_2 &&&\\
      &    &&         &        &        &        &    &        &        &\targ   &\qw &\ctrl{2}&\qw     &\ctrl{3}&\qw     &\ctrl{3}&\qw     &\ctrl{1}&\targ   &\qw     &        &        &        &        &        &        &    &&&        &&&\\
      &i_3 &&\qw      &\qw     &\qw     &\qw     &\qw &\qw     &\qw     &\qw     &\qw &\targ   &\ctrl{1}&\qw     &\qw     &\qw     &\ctrl{1}&\targ   &\qw     &\qw     &\qw     &\qw     &\qw     &\qw     &\qw     &\ctrl{1}&\qw &i_3     &&&&&\\
      &t_3 &&\qw      &\qw     &\qw     &\qw     &\qw &\qw     &\qw     &\qw     &\qw &\targ   &\ctrl{1}&\qw     &\qw     &\qw     &\ctrl{1}&\qw     &\qw     &\qw     &\qw     &\qw     &\qw     &\qw     &\qw     &\targ   &\qw &&&(t+i)_3 &&&\\
      &    &&         &        &        &        &    &        &        &        &    &        &        &\targ   &\ctrl{2}&\targ   &\qw     &        &        &        &        &        &        &        &        &        &    &&&        &&&\\
      &i_4 &&\qw      &\qw     &\qw     &\qw     &\qw &\qw     &\qw     &\qw     &\qw &\qw     &\qw     &\qw     &\qw     &\qw     &\qw     &\qw     &\qw     &\qw     &\qw     &\qw     &\qw     &\qw     &\qw     &\ctrl{1}&\qw &i_4     &&&&&\\
      &t_4 &&\qw      &\qw     &\qw     &\qw     &\qw &\qw     &\qw     &\qw     &\qw &\qw     &\qw     &\qw     &\targ   &\qw     &\qw     &\qw     &\qw     &\qw     &\qw     &\qw     &\qw     &\qw     &\qw     &\targ   &\qw &&&(t+i)_4 &&&\\
    }
  }
  \caption{
	5-bit adder with T-cost of 16.
	General $n$-bit adder has T-cost of $4n - 4$.
  }
  \label{fig:multi-bit-adder-example}
\end{figure}

\begin{figure}
  \resizebox{\linewidth}{!}{
    \Qcircuit @R=1.5em @C=0.7em {
      &c_k &&\ctrl{2}&\qw     &\ctrl{3}&\qw     &\qw &&c_k     &&& \\
      &i_k &&\targ   &\ctrl{1}&\qw     &\ctrl{1}&\qw &&i_k     &&& \\
      &t_k &&\targ   &\ctrl{1}&\qw     &\targ   &\qw &&&(t+i)_k && \\
      &    &&        &        &\targ   &\qw     &\qw &&c_{k+1} &&& \\
    }
  }
  \caption{
	  Temporary addition building block with a T-cost of 4 to compute and a T-cost of 0 to uncompute (not shown).
  }
  \label{fig:temporary-full-adder-block}
\end{figure}

We also note how to transform our adder into a controlled adder by controlling one of the operations.
The cost of this controlled adder is $8n + O(1)$; a significant improvement over a recently proposed construction with a T-cost of $21n + O(1)$.


\begin{figure}
  \resizebox{\linewidth}{!}{
    \Qcircuit @R=1.5em @C=0.7em {
      &\text{control} &&&&\qw     &\qw     &\qw     &\qw &\qw &\qw    &\qw &\qw &\qw &\qw    &\qw &\qw &\qw &\qw    &\qw &\qw &\qw     &\qw     &\ctrl{2}&\qw     &\qw &\\
      &&c_k            &&&\ctrl{2}&\qw     &\ctrl{3}&\qw &\qw &\qw    &\qw &\qw &\qw &\qw    &\qw &\qw &\qw &\qw    &\qw &\qw &\ctrl{3}&\qw     &\qw     &\ctrl{2}&\qw &\\
      &&i_k            &&&\targ   &\ctrl{1}&\qw     &\qw &\qw &\qw    &\qw &\qw &\qw &\qw    &\qw &\qw &\qw &\qw    &\qw &\qw &\qw     &\ctrl{1}&\ctrl{1}&\targ   &\qw &\\
      &&t_k            &&&\targ   &\ctrl{1}&\qw     &\qw &\qw &\qw    &\qw &\qw &\qw &\qw    &\qw &\qw &\qw &\qw    &\qw &\qw &\qw     &\ctrl{1}&\targ   &\targ   &\qw &\\
      &               &&&&        &        &\targ   &\qw &    &c_{k+1}&    &    &    &\ldots &    &    &    &c_{k+1}&    &    &\targ   &\qw     &        &        &    &\\
    }
  }
  \caption{
	Controlled full adder block with a T-cost of 8.
  }
  \label{fig:controlled-full-adder-block}
\end{figure}




\begin{figure}
  \centering
  \resizebox{\linewidth}{!}{
    \Qcircuit @R=1.5em @C=0.7em {
      &\text{control} &&&&\qw      &\qw     &\qw     &\qw     &\qw &\qw     &\qw     &\qw     &\qw &\qw     &\qw     &\qw     &\qw     &\ctrl{13}&\qw     &\qw     &\qw     &\ctrl{10}&\qw     &\qw     &\ctrl{7}&\qw     &\qw     &\qw     &\ctrl{4}&\qw     &\qw     &\ctrl{1}&\qw &&&\text{control}               &&&&&&\\
      &&i_0            &&&\ctrl{1} &\qw     &\qw     &\qw     &\qw &\qw     &\qw     &\qw     &\qw &\qw     &\qw     &\qw     &\qw     &\qw      &\qw     &\qw     &\qw     &\qw      &\qw     &\qw     &\qw     &\qw     &\qw     &\qw     &\qw     &\qw     &\ctrl{1}&\ctrl{1}&\qw &i_0                          &&&&&&&&\\
      &&t_0            &&&\ctrl{1} &\qw     &\qw     &\qw     &\qw &\qw     &\qw     &\qw     &\qw &\qw     &\qw     &\qw     &\qw     &\qw      &\qw     &\qw     &\qw     &\qw      &\qw     &\qw     &\qw     &\qw     &\qw     &\qw     &\qw     &\qw     &\ctrl{1}&\targ   &\qw &&&&&&(t+i \cdot \text{control})_0 &&&\\
      &&               &&&         &\ctrl{2}&\qw     &\ctrl{3}&\qw &\qw     &\qw     &\qw     &\qw &\qw     &\qw     &\qw     &\qw     &\qw      &\qw     &\qw     &\qw     &\qw      &\qw     &\qw     &\qw     &\qw     &\ctrl{3}&\qw     &\qw     &\ctrl{2}&\qw     &        &    &&&                             &&&&&&\\
      &&i_1            &&&\qw      &\targ   &\ctrl{1}&\qw     &\qw &\qw     &\qw     &\qw     &\qw &\qw     &\qw     &\qw     &\qw     &\qw      &\qw     &\qw     &\qw     &\qw      &\qw     &\qw     &\qw     &\qw     &\qw     &\ctrl{1}&\ctrl{1}&\targ   &\qw     &\qw     &\qw &i_1                          &&&&&&&&\\
      &&t_1            &&&\qw      &\targ   &\ctrl{1}&\qw     &\qw &\qw     &\qw     &\qw     &\qw &\qw     &\qw     &\qw     &\qw     &\qw      &\qw     &\qw     &\qw     &\qw      &\qw     &\qw     &\qw     &\qw     &\qw     &\ctrl{1}&\targ   &\targ   &\qw     &\qw     &\qw &&&&&&(t+i \cdot \text{control})_1 &&&\\
      &&               &&&         &        &        &\targ   &\qw &\ctrl{2}&\qw     &\ctrl{3}&\qw &\qw     &\qw     &\qw     &\qw     &\qw      &\qw     &\qw     &\qw     &\qw      &\ctrl{3}&\qw     &\qw     &\ctrl{2}&\targ   &\qw     &        &        &        &        &    &&&                             &&&&&&\\
      &&i_2            &&&\qw      &\qw     &\qw     &\qw     &\qw &\targ   &\ctrl{1}&\qw     &\qw &\qw     &\qw     &\qw     &\qw     &\qw      &\qw     &\qw     &\qw     &\qw      &\qw     &\ctrl{1}&\ctrl{1}&\targ   &\qw     &\qw     &\qw     &\qw     &\qw     &\qw     &\qw &i_2                          &&&&&&&&\\
      &&t_2            &&&\qw      &\qw     &\qw     &\qw     &\qw &\targ   &\ctrl{1}&\qw     &\qw &\qw     &\qw     &\qw     &\qw     &\qw      &\qw     &\qw     &\qw     &\qw      &\qw     &\ctrl{1}&\targ   &\targ   &\qw     &\qw     &\qw     &\qw     &\qw     &\qw     &\qw &&&&&&(t+i \cdot \text{control})_2 &&&\\
      &&               &&&         &        &        &        &    &        &        &\targ   &\qw &\ctrl{2}&\qw     &\ctrl{3}&\qw     &\qw      &\qw     &\ctrl{3}&\qw     &\qw      &\ctrl{2}&\qw     &        &        &        &        &        &        &        &        &    &&&                             &&&&&&\\
      &&i_3            &&&\qw      &\qw     &\qw     &\qw     &\qw &\qw     &\qw     &\qw     &\qw &\targ   &\ctrl{1}&\qw     &\qw     &\qw      &\qw     &\qw     &\ctrl{1}&\ctrl{1} &\targ   &\qw     &\qw     &\qw     &\qw     &\qw     &\qw     &\qw     &\qw     &\qw     &\qw &i_3                          &&&&&&&&\\
      &&t_3            &&&\qw      &\qw     &\qw     &\qw     &\qw &\qw     &\qw     &\qw     &\qw &\targ   &\ctrl{1}&\qw     &\qw     &\qw      &\qw     &\qw     &\ctrl{1}&\targ    &\targ   &\qw     &\qw     &\qw     &\qw     &\qw     &\qw     &\qw     &\qw     &\qw     &\qw &&&&&&(t+i \cdot \text{control})_3 &&&\\
      &&               &&&         &        &        &        &    &        &        &        &    &        &        &\targ   &\ctrl{1}&\qw      &\ctrl{1}&\targ   &\qw     &         &        &        &        &        &        &        &        &        &        &        &    &&&                             &&&&&&\\
      &&i_4            &&&\qw      &\qw     &\qw     &\qw     &\qw &\qw     &\qw     &\qw     &\qw &\qw     &\qw     &\qw     &\targ   &\ctrl{1} &\targ   &\qw     &\qw     &\qw      &\qw     &\qw     &\qw     &\qw     &\qw     &\qw     &\qw     &\qw     &\qw     &\qw     &\qw &i_4                          &&&&&&&&\\
      &&t_4            &&&\qw      &\qw     &\qw     &\qw     &\qw &\qw     &\qw     &\qw     &\qw &\qw     &\qw     &\qw     &\qw     &\targ    &\qw     &\qw     &\qw     &\qw      &\qw     &\qw     &\qw     &\qw     &\qw     &\qw     &\qw     &\qw     &\qw     &\qw     &\qw &&&&&&(t+i \cdot \text{control})_4 &&&\\
    }
  }
  \caption{
	5-bit controlled adder with a T-cost of 36.
	General $n$-bit controlled adder has T-cost of $8n - 4$.
  }
  \label{fig:ancilla-addition}
\end{figure}

Other examples of operations which benefit from this optimization include:

- Integer comparisons.

- Integer multiplication.

- Incrementing and counting.

- Modular arithmetic.

- Operations with a target qubit indexed by a binary qubit register.

- Expanding a binary register into a unary register.

- Phasing a register by a computable function $f$ of its value (i.e. applying the operation $U_f = \exp\left( i \sum_k f(k) |k\rangle \langle k| \right)$).

- Temporary permutations.

- Oracles in Grover's algorithm.

- etc


\section{Conclusion}
\label{sec:conclusion}

It was commonly believed that the optimal number of T gates to perform addition was 8, or 7 \cite{AustinDiscussionsAndEmails2017}.
We showed that the number of T gates is actually 4.
We are not aware of any method for improving the T cost further, but we do prove the temporary AND gatea lower bound of 2 In \autoref{fig:lower-bound} we show a lower bound of 2


\section{Acknowledgements}

Austin [reference finding, idea bouncing, review].


\bibliographystyle{plain}
\bibliography{citations}

https://arxiv.org/pdf/quant-ph/9503016.pdf   ->   the paired Toffoli construction that uses 4T

https://arxiv.org/pdf/cond-mat/9409111.pdf   -> shows CCNOT can be done with three 2-qubit interactions, ignores single-qubit???


https://arxiv.org/pdf/1308.4134.pdf -> proof of 7T gate minimum with no ancilla

https://arxiv.org/pdf/1212.5069.pdf  ->   does a single Toffoli with four T gates by turning one of the controls classical

https://arxiv.org/pdf/1210.0974.pdf  ->   a phase-neglected 4T gate construction that only uses a single layer of T gates


Quantum Computation and Quantum Information -> shows standard decomposition into 7T gates. But do they say it's optimal?

https://arxiv.org/pdf/1206.0758.pdf -> just an example of the 7T construction;;; probablynot relevant




>>>>>>>>>>>>>>>>>>>>>>> https://arxiv.org/pdf/1206.0758.pdf          paper that says addition costs 8N T


https://arxiv.org/pdf/1706.05113.pdf   has a ctrl-add with 21N + O(1). This technique+commutators achieves 8N.


https://pdfs.semanticscholar.org/8b84/d1bc2928922937a16205f9d8c925ff76689b.pdf     has a pretty good ripple carry


\end{document}
