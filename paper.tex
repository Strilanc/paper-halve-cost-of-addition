\documentclass{quantumarticle-customized}
\usepackage{amssymb}
\usepackage{amsmath}
\usepackage{graphicx}
\usepackage[sort,numbers]{natbib}
\usepackage[pdfpagelabels,pdftex,bookmarks,breaklinks]{hyperref}
\usepackage{tikz}
\usepackage[all]{hypcap}
\hypersetup{colorlinks,citecolor=blue,urlcolor=blue,linkcolor=blue}

%    Q-circuit version 2
%    Copyright (C) 2004  Steve Flammia & Bryan Eastin
%    Last modified on: 9/16/2011
%
%    This program is free software; you can redistribute it and/or modify
%    it under the terms of the GNU General Public License as published by
%    the Free Software Foundation; either version 2 of the License, or
%    (at your option) any later version.
%
%    This program is distributed in the hope that it will be useful,
%    but WITHOUT ANY WARRANTY; without even the implied warranty of
%    MERCHANTABILITY or FITNESS FOR A PARTICULAR PURPOSE.  See the
%    GNU General Public License for more details.
%
%    You should have received a copy of the GNU General Public License
%    along with this program; if not, write to the Free Software
%    Foundation, Inc., 59 Temple Place, Suite 330, Boston, MA  02111-1307  USA

% Thanks to the Xy-pic guys, Kristoffer H Rose, Ross Moore, and Daniel Müllner,
% for their help in making Qcircuit work with Xy-pic version 3.8.  
% Thanks also to Dave Clader, Andrew Childs, Rafael Possignolo, Tyson Williams,
% Sergio Boixo, Cris Moore, Jonas Anderson, and Stephan Mertens for helping us test 
% and/or develop the new version.

\usepackage[color]{xy}
\UseCrayolaColors
\xyoption{matrix}
\xyoption{frame}
\xyoption{arrow}
\xyoption{arc}

\usepackage{ifpdf}
\ifpdf
\else
\PackageWarningNoLine{Qcircuit}{Qcircuit is loading in Postscript mode.  The Xy-pic options ps and dvips will be loaded.  If you wish to use other Postscript drivers for Xy-pic, you must modify the code in Qcircuit.tex}
%    The following options load the drivers most commonly required to
%    get proper Postscript output from Xy-pic.  Should these fail to work,
%    try replacing the following two lines with some of the other options
%    given in the Xy-pic reference manual.
\xyoption{ps}
\xyoption{dvips}
\fi

% The following resets Xy-pic matrix alignment to the pre-3.8 default, as
% required by Qcircuit.
\entrymodifiers={!C\entrybox}

%\newcommand{\bra}[1]{{\left\langle{#1}\right\vert}}
%\newcommand{\ket}[1]{{\left\vert{#1}\right\rangle}}
    % Defines Dirac notation. %7/5/07 added extra braces so that the commands will work in subscripts.
\newcommand{\qw}[1][-1]{\ar @{-} [0,#1]}
\newcommand{\eqw}[1][-1]{\ar @{-} @[Red] [0,#1]}
    % Defines a wire that connects horizontally.  By default it connects to the object on the left of the current object.
    % WARNING: Wire commands must appear after the gate in any given entry.
\newcommand{\qwx}[1][-1]{\ar @{-} [#1,0]}
    % Defines a wire that connects vertically.  By default it connects to the object above the current object.
    % WARNING: Wire commands must appear after the gate in any given entry.
\newcommand{\cw}[1][-1]{\ar @{=} [0,#1]}
    % Defines a classical wire that connects horizontally.  By default it connects to the object on the left of the current object.
    % WARNING: Wire commands must appear after the gate in any given entry.
\newcommand{\cwx}[1][-1]{\ar @{=} [#1,0]}
    % Defines a classical wire that connects vertically.  By default it connects to the object above the current object.
    % WARNING: Wire commands must appear after the gate in any given entry.
\newcommand{\gate}[1]{*+<.6em>{#1} \POS ="i","i"+UR;"i"+UL **\dir{-};"i"+DL **\dir{-};"i"+DR **\dir{-};"i"+UR **\dir{-},"i" \qw}
\newcommand{\eboxgate} [1]{*+<.6em>{#1} \POS ="i","i"+UR;"i"+UL **[red]\dir{-};"i"+DL **[red]\dir{-};"i"+DR **[red]\dir{-};"i"+UR **[red]\dir{-},"i" \eqw}
\newcommand{\circgate}[1]{*+<0.6em>[o][F-]{#1} \eqw}
\newcommand{\ecircgate}[1]{*+<0.6em>[o][F-:red]{#1} \eqw}
\newcommand{\filtergt}[1]{\eboxgate{\scriptscriptstyle{#1}}}
\newcommand{\idealdec}{*+<1.2em>{\phantom{*}} \POS ="i","i"+UL;"i"+DL **[red]\dir{-};"i"+R **[red]\dir{-};"i"+UL **[red]\dir{-},"i" \eqw}

    % Boxes the argument, making a gate.
\newcommand{\meter}{*=<1.8em,1.4em>{\xy ="j","j"-<.778em,.322em>;{"j"+<.778em,-.322em> \ellipse ur,_{}},"j"-<0em,.4em>;p+<.5em,.9em> **\dir{-},"j"+<2.2em,2.2em>*{},"j"-<2.2em,2.2em>*{} \endxy} \POS ="i","i"+UR;"i"+UL **\dir{-};"i"+DL **\dir{-};"i"+DR **\dir{-};"i"+UR **\dir{-},"i" \qw}
    % Inserts a measurement meter.
    % In case you're wondering, the constants .778em and .322em specify
    % one quarter of a circle with radius 1.1em.
    % The points added at + and - <2.2em,2.2em> are there to strech the
    % canvas, ensuring that the size is unaffected by erratic spacing issues
    % with the arc.
\newcommand{\measure}[1]{*+[F-:<.9em>]{#1} \qw}
    % Inserts a measurement bubble with user defined text.
\newcommand{\measuretab}[1]{*{\xy*+<.6em>{#1}="e";"e"+UL;"e"+UR **\dir{-};"e"+DR **\dir{-};"e"+DL **\dir{-};"e"+LC-<.5em,0em> **\dir{-};"e"+UL **\dir{-} \endxy} \qw}
    % Inserts a measurement tab with user defined text.
\newcommand{\measureD}[1]{*{\xy*+=<0em,.1em>{#1}="e";"e"+UR+<0em,.25em>;"e"+UL+<-.5em,.25em> **\dir{-};"e"+DL+<-.5em,-.25em> **\dir{-};"e"+DR+<0em,-.25em> **\dir{-};{"e"+UR+<0em,.25em>\ellipse^{}};"e"+C:,+(0,1)*{} \endxy} \qw}
\newcommand{\emeasureD}[1]{*{\xy*+=<0em,.1em>{#1}="e";"e"+UR+<0em,.25em>;"e"+UL+<-.5em,.25em> **[red]\dir{-};"e"+DL+<-.5em,-.25em> **[red]\dir{-};"e"+DR+<0em,-.25em> **[red]\dir{-};{"e"+UR+<0em,.25em>\ellipse{}};"e"+C:,+(0,1)*{} \endxy} \qw}
    % Inserts a D-shaped measurement gate with user defined text.
\newcommand{\multimeasure}[2]{*+<1em,.9em>{\hphantom{#2}} \qw \POS[0,0].[#1,0];p !C *{#2},p \drop\frm<.9em>{-}}
    % Draws a multiple qubit measurement bubble starting at the current position and spanning #1 additional gates below.
    % #2 gives the label for the gate.
    % You must use an argument of the same width as #2 in \ghost for the wires to connect properly on the lower lines.
\newcommand{\multimeasureD}[2]{*+<1em,.9em>{\hphantom{#2}} \POS [0,0]="i",[0,0].[#1,0]="e",!C *{#2},"e"+UR-<.8em,0em>;"e"+UL **\dir{-};"e"+DL **\dir{-};"e"+DR+<-.8em,0em> **\dir{-};{"e"+DR+<0em,.8em>\ellipse^{}};"e"+UR+<0em,-.8em> **\dir{-};{"e"+UR-<.8em,0em>\ellipse^{}},"i" \qw}
    % Draws a multiple qubit D-shaped measurement gate starting at the current position and spanning #1 additional gates below.
    % #2 gives the label for the gate.
    % You must use an argument of the same width as #2 in \ghost for the wires to connect properly on the lower lines.
\newcommand{\control}{*!<0em,.025em>-=-<.2em>{\bullet}}
    % Inserts an unconnected control.
\newcommand{\controlo}{*+<.01em>{\xy -<.095em>*\xycircle<.19em>{} \endxy}}
    % Inserts a unconnected control-on-0.
\newcommand{\ctrl}[1]{\control \qwx[#1] \qw}
    % Inserts a control and connects it to the object #1 wires below.
\newcommand{\ctrlo}[1]{\controlo \qwx[#1] \qw}
    % Inserts a control-on-0 and connects it to the object #1 wires below.
\newcommand{\targ}{*+<.02em,.02em>{\xy ="i","i"-<.39em,0em>;"i"+<.39em,0em> **\dir{-}, "i"-<0em,.39em>;"i"+<0em,.39em> **\dir{-},"i"*\xycircle<.4em>{} \endxy} \qw}
    % Inserts a CNOT target.
\newcommand{\qswap}{*=<0em>{\times} \qw}
    % Inserts half a swap gate.
    % Must be connected to the other swap with \qwx.
\newcommand{\multigate}[2]{*+<1em,.9em>{\hphantom{#2}} \POS [0,0]="i",[0,0].[#1,0]="e",!C *{#2},"e"+UR;"e"+UL **\dir{-};"e"+DL **\dir{-};"e"+DR **\dir{-};"e"+UR **\dir{-},"i" \qw}
    % Draws a multiple qubit gate starting at the current position and spanning #1 additional gates below.
    % #2 gives the label for the gate.
    % You must use an argument of the same width as #2 in \ghost for the wires to connect properly on the lower lines.
\newcommand{\ghost}[1]{*+<1em,.9em>{\hphantom{#1}} \qw}
    % Leaves space for \multigate on wires other than the one on which \multigate appears.  Without this command wires will cross your gate.
    % #1 should match the second argument in the corresponding \multigate.
\newcommand{\push}[1]{*{#1}}
    % Inserts #1, overriding the default that causes entries to have zero size.  This command takes the place of a gate.
    % Like a gate, it must precede any wire commands.
    % \push is useful for forcing columns apart.
    % NOTE: It might be useful to know that a gate is about 1.3 times the height of its contents.  I.e. \gate{M} is 1.3em tall.
    % WARNING: \push must appear before any wire commands and may not appear in an entry with a gate or label.
\newcommand{\gategroup}[6]{\POS"#1,#2"."#3,#2"."#1,#4"."#3,#4"!C*+<#5>\frm{#6}}
    % Constructs a box or bracket enclosing the square block spanning rows #1-#3 and columns=#2-#4.
    % The block is given a margin #5/2, so #5 should be a valid length.
    % #6 can take the following arguments -- or . or _\} or ^\} or \{ or \} or _) or ^) or ( or ) where the first two options yield dashed and
    % dotted boxes respectively, and the last eight options yield bottom, top, left, and right braces of the curly or normal variety.  See the Xy-pic reference manual for more options.
    % \gategroup can appear at the end of any gate entry, but it's good form to pick either the last entry or one of the corner gates.
    % BUG: \gategroup uses the four corner gates to determine the size of the bounding box.  Other gates may stick out of that box.  See \prop.

\newcommand{\rstick}[1]{*!L!<-.5em,0em>=<0em>{#1}}
    % Centers the left side of #1 in the cell.  Intended for lining up wire labels.  Note that non-gates have default size zero.
\newcommand{\lstick}[1]{*!R!<.5em,0em>=<0em>{#1}}
    % Centers the right side of #1 in the cell.  Intended for lining up wire labels.  Note that non-gates have default size zero.
\newcommand{\ustick}[1]{*!D!<0em,-.5em>=<0em>{#1}}
    % Centers the bottom of #1 in the cell.  Intended for lining up wire labels.  Note that non-gates have default size zero.
\newcommand{\dstick}[1]{*!U!<0em,.5em>=<0em>{#1}}
    % Centers the top of #1 in the cell.  Intended for lining up wire labels.  Note that non-gates have default size zero.
\newcommand{\Qcircuit}{\xymatrix @*=<0em>}
    % Defines \Qcircuit as an \xymatrix with entries of default size 0em.
\newcommand{\link}[2]{\ar @{-} [#1,#2]}
    % Draws a wire or connecting line to the element #1 rows down and #2 columns forward.
\newcommand{\pureghost}[1]{*+<1em,.9em>{\hphantom{#1}}}
    % Same as \ghost except it omits the wire leading to the left. 
%%%%%%%%%%%%%%%%%%%%%%%%%%%%%%%%%%%%%%%%%%%%%%%%%%%%%%%%%%%%%%%%%%%%%%%%%%%%%%%%%%%%%%%%%%
\newcommand{\multiprepareC}[2]{*+<1em,.9em>{\hphantom{#2}}\save[0,0].[#1,0];p\save !C
  *{#2},p+RU+<0em,0em>;+LU+<+.8em,0em> **\dir{-}\restore\save +RD;+RU **\dir{-}\restore\save
  +RD;+LD+<.8em,0em> **\dir{-} \restore\save +LD+<0em,.8em>;+LU-<0em,.8em> **\dir{-} \restore \POS
  !UL*!UL{\cir<.9em>{u_r}};!DL*!DL{\cir<.9em>{l_u}}\restore}
   % Draws a multiple qubit reverse-D-shaped preparation gate starting at the current position and spanning #1 additional gates below.
   % #2 gives the label for the gate.
   % You must use an argument of the same width as #2 in \pureghost for the wires to connect properly on
% the lower lines.
\newcommand{\prepareC}[1]{*{\xy*+=+<.5em>{\vphantom{#1\rule{0em}{.1em}}}*\cir{l^r};p\save*!L{#1} \restore\save+UC;+UC+<.5em,0em>*!L{\hphantom{#1}}+R **\dir{-} \restore\save+DC;+DC+<.5em,0em>*!L{\hphantom{#1}}+R **\dir{-} \restore\POS+UC+<.5em,0em>*!L{\hphantom{#1}}+R;+DC+<.5em,0em>*!L{\hphantom{#1}}+R **\dir{-} \endxy}}
   % Inserts a reverse-D-shaped preparation gate with user defined text.
\newcommand{\poloFantasmaCn}[1]{{{}^{#1}_{\phantom{#1}}}}

\newcommand{\qH}{\gate{H}}
\newcommand{\qT}{\gate{T}}
\newcommand{\qTi}{\gate{T^\dagger}}
\newcommand{\qS}{\gate{S}}
\newcommand{\qSi}{\gate{S^\dagger}}
\newcommand{\qA}{\lstick{|A\rangle}}
\newcommand{\qO}{\lstick{|0\rangle}}

\title{Halving the cost of quantum addition}
\author{Craig Gidney}
\affiliation{Google, Santa Barbara, CA 93117, USA}
\email{craiggidney@google.com}

\def\sectionautorefname{Section}

\begin{document}
\maketitle

\begin{abstract}
We improve the number of T gates needed to perform an $n$-bit adder from $8n + O(1)$ \citep{Amy2013, Cuccaro2004, AustinDiscussionsAndEmails2017} to $4n + O(1)$.
We do so via a ``temporary logical-AND" construction, which uses four T gates to store the logical-AND of two qubits into an ancilla and zero T gates to later erase the ancilla.

Temporary logical-ANDs are a generally useful tool when optimizing T-count.
They can be applied to integer arithmetic, modular arithmetic, rotation synthesis, the quantum Fourier transform, Shor's algorithm, Grover oracles, and many other circuits.
Because T gates dominate the cost of quantum computation based on the surface code, and temporary logical-ANDs are widely applicable, our constructions represent a significant reduction in projected costs of quantum computation.

In addition to our $n$-bit adder circuit with T-count of $4n + O(1)$, we show how to construct an $n$-bit controlled adder circuit with T-count of $8n + O(1)$, and a temporary adder that can be computed for the same cost as the normal adder but whose result can be kept until later uncomputed without T gates.
\end{abstract}


\section{Introduction}
\label{sec:introduction}

The surface code \citep{Brav98,Denn02,Raus07,Raus07d,Fowler2012} is a quantum error correcting code that works on a 2D nearest-neighbour array of qubits and achieves a threshold error rate of approximately 1\%.
This makes the surface code a likely component in the architecture of future error corrected quantum computers, because 2D arrays of qubits with nearest-neighbor connections are possible with many qubit technologies \citep{Schl11,Bare13,Gamb17,Leik17,Laht17} and other well understood error correcting codes either have lower thresholds or require stronger connectivity.

One of the downsides of the surface code is that it has no cheap mechanism to apply non-Clifford operations such as $T$ gates.
Instead, $T$ gates are performed by distilling and consuming $|A\rangle = \frac{1}{\sqrt{2}} (|0\rangle + e^{i \pi/4} |1\rangle)$ states.
Consuming an $|A\rangle$ state to perform a T gate is simple, but distilling $|A\rangle$ states has significant cost.
Because T gates are so expensive for the surface code, and the surface code is a likely component of future quantum computers, it is important to consider and optimize the number of T gates used by quantum circuits.
Optimizing the T-count of basic elements of quantum circuits, such as the construction of adders and Toffoli gates, is even more important.

The textbook construction of a Toffoli gate uses seven T gates \citep{Nielsen2009}.
When Toffoli operations are paired, i.e. when an initial Toffoli operation is later uncomputed by a second Toffoli operation, each Toffoli in the pair can omit three of the T gates from the textbook construction.
This introduces phase errors but, assuming intermediate operations aren't poisoned by the phase errors, the second Toffoli gate can uncompute the phase errors while uncomputing the state permutation \citep{Barenco1995, Nielsen2009}.
It is also possible to reduce the T-count of an unpaired Toffoli gate to 4 by using an ancilla qubit and a classically conditioned fixup operations \citep{Jones2013}.

Existing Toffoli constructions use at least 4 T gates per Toffoli gate.
The Cuccaro adder uses $2n-1$ Toffoli gates \citep{Cuccaro2004}.
Correspondingly, existing adder constructions have T-counts of $8n-4$ \citep{Amy2013}.
Adders with a T-count of $8n + O(1)$ have been known for over a decade \citep{Barenco1995, Cuccaro2004}.
The leading factor of 8 has been conjectured to be optimal \citep{AustinDiscussionsAndEmails2017}.

\begin{figure}
  \resizebox{\linewidth}{!}{
    \Qcircuit @R=0.7em @C=0.7em {
      &i_0 &&\ctrl{1} &\qw     &\qw     &\qw     &\qw     &\qw     &\qw     &\qw     &\qw     &\qw     &\qw     &\qw     &\qw     &\qw     &\qw     &\qw     &\qw     &\qw     &\qw     &\qw     &\ctrl{1}&\ctrl{1}&\qw &i_0     &&&&&\\
      &t_0 &&\ctrl{1} &\qw     &\qw     &\qw     &\qw     &\qw     &\qw     &\qw     &\qw     &\qw     &\qw     &\qw     &\qw     &\qw     &\qw     &\qw     &\qw     &\qw     &\qw     &\qw     &\ctrl{1}&\targ   &\qw &&&(t+i)_0 &&&\\
      &    &&         &\ctrl{2}&\qw     &\ctrl{3}&\qw     &\qw     &\qw     &\qw     &\qw     &\qw     &\qw     &\qw     &\qw     &\qw     &\qw     &\qw     &\qw     &\ctrl{3}&\qw     &\ctrl{1}&\qw     &        &    &&&        &&&\\
      &i_1 &&\qw      &\targ   &\ctrl{1}&\qw     &\qw     &\qw     &\qw     &\qw     &\qw     &\qw     &\qw     &\qw     &\qw     &\qw     &\qw     &\qw     &\qw     &\qw     &\ctrl{1}&\targ   &\qw     &\ctrl{1}&\qw &i_1     &&&&&\\
      &t_1 &&\qw      &\targ   &\ctrl{1}&\qw     &\qw     &\qw     &\qw     &\qw     &\qw     &\qw     &\qw     &\qw     &\qw     &\qw     &\qw     &\qw     &\qw     &\qw     &\ctrl{1}&\qw     &\qw     &\targ   &\qw &&&(t+i)_1 &&&\\
      &    &&         &        &        &\targ   &\ctrl{2}&\qw     &\ctrl{3}&\qw     &\qw     &\qw     &\qw     &\qw     &\qw     &\qw     &\ctrl{3}&\qw     &\ctrl{1}&\targ   &\qw     &        &        &        &    &&&        &&&\\
      &i_2 &&\qw      &\qw     &\qw     &\qw     &\targ   &\ctrl{1}&\qw     &\qw     &\qw     &\qw     &\qw     &\qw     &\qw     &\qw     &\qw     &\ctrl{1}&\targ   &\qw     &\qw     &\qw     &\qw     &\ctrl{1}&\qw &i_2     &&&&&\\
      &t_2 &&\qw      &\qw     &\qw     &\qw     &\targ   &\ctrl{1}&\qw     &\qw     &\qw     &\qw     &\qw     &\qw     &\qw     &\qw     &\qw     &\ctrl{1}&\qw     &\qw     &\qw     &\qw     &\qw     &\targ   &\qw &&&(t+i)_2 &&&\\
      &    &&         &        &        &        &        &        &\targ   &\ctrl{2}&\qw     &\ctrl{3}&\qw     &\ctrl{3}&\qw     &\ctrl{1}&\targ   &\qw     &        &        &        &        &        &        &    &&&        &&&\\
      &i_3 &&\qw      &\qw     &\qw     &\qw     &\qw     &\qw     &\qw     &\targ   &\ctrl{1}&\qw     &\qw     &\qw     &\ctrl{1}&\targ   &\qw     &\qw     &\qw     &\qw     &\qw     &\qw     &\qw     &\ctrl{1}&\qw &i_3     &&&&&\\
      &t_3 &&\qw      &\qw     &\qw     &\qw     &\qw     &\qw     &\qw     &\targ   &\ctrl{1}&\qw     &\qw     &\qw     &\ctrl{1}&\qw     &\qw     &\qw     &\qw     &\qw     &\qw     &\qw     &\qw     &\targ   &\qw &&&(t+i)_3 &&&\\
      &    &&         &        &        &        &        &        &        &        &        &\targ   &\ctrl{2}&\targ   &\qw     &        &        &        &        &        &        &        &        &        &    &&&        &&&\\
      &i_4 &&\qw      &\qw     &\qw     &\qw     &\qw     &\qw     &\qw     &\qw     &\qw     &\qw     &\qw     &\qw     &\qw     &\qw     &\qw     &\qw     &\qw     &\qw     &\qw     &\qw     &\qw     &\ctrl{1}&\qw &i_4     &&&&&\\
      &t_4 &&\qw      &\qw     &\qw     &\qw     &\qw     &\qw     &\qw     &\qw     &\qw     &\qw     &\targ   &\qw     &\qw     &\qw     &\qw     &\qw     &\qw     &\qw     &\qw     &\qw     &\qw     &\targ   &\qw &&&(t+i)_4 &&&\\
    }
  }
  \caption{
	A 5-bit adder with T-count of 16.
	Uses Clifford operations, four logical-AND computations each with a T-count of 4, and four logical-AND erasures requiring no T gates.
	Generalizes to an $n$-bit adder with a T-count of $4n - 4$.
	See \autoref{fig:full-adder-block} for the adder building-block and \autoref{fig:temporary-logical-AND} for the logical-AND computation and uncomputation circuits.
  }
  \label{fig:multi-bit-adder-example}
\end{figure}

We halve the number of T gates needed to perform an adder by halving the number of T gates needed to perform Toffoli gates that appear in compute/uncompute pairs.
Although we focus on adders in this paper, paired Toffoli gates appear in a wide variety of circuits.
Other examples of operations with circuits whose T-count can be improved by our constructions include:

\begin{itemize}
  \item Integer arithmetic in general.
  \item Modular arithmetic.
  \item Applying the same rotation $R_Z(\theta)$ to many qubits.
  \item The quantum Fourier transform.
  \item Operations whose target is indexed by a quantum register.
  \item Phasing a register by a computable function $f$, i.e. applying the operation $U_f = \exp\left( i \sum_k f(k) |k\rangle \langle k| \right)$.
  \item Expanding a binary register into a unary register.
\end{itemize}

The above list is not exhaustive.
We leave further low hanging fruit as an exercise for the reader.

Our paper is divided into four sections.
In \autoref{sec:introduction}, we motivate the need to optimize T-counts, discuss existing work, and note the wide applicability of our low T-count constructions.
\autoref{sec:construction} presents our $4n + O(1)$ T-count adder and explains its pieces, including the temporary logical-AND.
\autoref{sec:other-applications} discusses several circuit constructions and estimates improved by our adder and temporary logical-AND constructions.
Finally, \autoref{sec:conclusion} summarizes our contributions and discusses future work.


\section{Adder Construction}
\label{sec:construction}

In \autoref{fig:multi-bit-adder-example}, we present a 5-bit adder with a T-count of 16.
It performs 4 temporary logical-ANDs, each with a T-count of 4.

We construct $n$-bit adders by nesting $n$ copies of the building block shown in \autoref{fig:full-adder-block} inside of each other.
We then specialize the outer-most and inner-most blocks (which act on the low bit and high bit respectively) based on the fact that they either have no carry input or no carry output.

\begin{figure}
  \resizebox{\linewidth}{!}{
    \Qcircuit @R=0.7em @C=0.7em {
      &c_k &&\ctrl{2}&\qw     &\ctrl{3}&\qw &\qw &\qw    &\qw &\qw &\qw &\qw    &\qw &\qw &\qw &\qw    &\qw &\qw &\ctrl{3}&\qw     &\ctrl{1}&\qw     &\qw &c_k     &&&&& \\
      &i_k &&\targ   &\ctrl{1}&\qw     &\qw &\qw &\qw    &\qw &\qw &\qw &\qw    &\qw &\qw &\qw &\qw    &\qw &\qw &\qw     &\ctrl{1}&\targ   &\ctrl{1}&\qw &i_k     &&&&& \\
      &t_k &&\targ   &\ctrl{1}&\qw     &\qw &\qw &\qw    &\qw &\qw &\qw &\qw    &\qw &\qw &\qw &\qw    &\qw &\qw &\qw     &\ctrl{1}&\qw     &\targ   &\qw &&&(t+i)_k &&& \\
      &    &&        &        &\targ   &\qw &    &c_{k+1}&    &    &    &\ldots &    &    &    &c_{k+1}&    &    &\targ   &\qw     &        &        &    &&&        &&& \\
    }
  }
  \caption{
	Adder circuit building-block with a T-count of 4.
  }
  \label{fig:full-adder-block}
\end{figure}

Our adder circuits use temporary logical-AND operations.
We define the temporary logical-AND computation (which we draw as a wire emerging from two controls), and the corresponding uncomputation (which we draw as a wire merging into two controls), in \autoref{fig:temporary-logical-AND}.
Computing the temporary logical-AND has a T-count of 4, but when uncomputing a temporary logical-AND we have access to measurement and only need Clifford operations.

\begin{figure}
  \resizebox{\linewidth}{!}{
    \Qcircuit @R=0.7em @C=0.7em {
      &x &&\ctrl{1} &\qw &x  && &&          &&\ctrl{2}&\qw     &\targ    &\qTi &\targ     &\qw &\qw &\qw &\\
      &y &&\ctrl{1} &\qw &y  &&=&&          &&\qw     &\ctrl{1}&\targ    &\qTi &\targ     &\qw &\qw &\qw &\\
      &  &&         &\qw &xy && &&|A\rangle &&\targ   &\targ   &\ctrl{-2}&\qT  &\ctrl{-2} &\qH &\qS &\qw &\\
    }
  }
  \\ \vspace*{0.25cm} \\
  \resizebox{\linewidth}{!}{
    \Qcircuit @R=0.7em @C=0.7em {
      &x  &&\ctrl{1} &\qw &x && &&&\qw &\qw    &\ctrl{1}         &\qw &&&&&&&&&&\\
      &y  &&\ctrl{1} &\qw &y &&=&&&\qw &\qw    &\gate{Z}         &\qw &&&&&&&&&&\\
      &xy &&\qw      &    &  && &&&\qH &\meter &\cw \cwx \bullet &    &&&&&&&&&&\\
    }
  }
  \caption{
	How to compute and uncompute the logical-AND of two qubits.
	The computation circuit has a T-count of 4 and a T-depth of 1.
	Note that the $|A\rangle$ state input contributes to the T-count, because $|A\rangle$ states are the resource used to perform T gates.
	The uncomputation circuit only uses Clifford gates, and so has a T-count of zero.
    \\
    An alternative uncomputation construction is to simply do the reverse of the computation circuit.
    This alternative approach has a net T-count of 2 (because an $|A\rangle$ state is recovered).
    That would still be an improvement on existing work, but inferior to the measure-and-fixup construction shown above.
  }
  \label{fig:temporary-logical-AND}
\end{figure}


\section{Other Applications}
\label{sec:other-applications}

\autoref{fig:variations-on-full-adder-block} shows the building-blocks for two variations on our adder: a controlled adder and a temporary adder.

Some additions, such as the ones performed by the multiplications within the modular exponentiation in Shor's algorithm, are conditioned on a control qubit.
Our controlled adder reduces the cost of these additions from $21n + O(1)$ \citep{Coreas2017} to $8n + O(1)$.

Temporary adders are useful when a circuit is going to compute an addition, then later uncompute it.
Instead of paying $4n + O(1)$ T gates to compute an addition, and then $4n + O(1)$ more T gates to uncompute the addition with a subtraction, our temporary adder stops halfway through the normal addition (after all the logical-ANDs have been computed, but before they've been uncomputed) and applies a few cleanup operations so that the sum is available for use.
When the sum is no longer needed, the partially-performed addition is uncomputed instead of completed.

\begin{figure}
  (a) \resizebox{\linewidth}{!}{
    \Qcircuit @R=0.7em @C=0.7em {
      &&\text{control} &&&&\qw     &\qw     &\qw     &\qw &\qw &\qw    &\qw &\qw &\qw &\qw    &\qw &\qw &\qw &\qw    &\qw &\qw &\qw     &\qw     &\ctrl{2}&\qw     &\ctrl{2}&\qw     &\qw &&&\text{control}               &&&&&&\\
      &&&c_k            &&&\ctrl{3}&\qw     &\ctrl{4}&\qw &\qw &\qw    &\qw &\qw &\qw &\qw    &\qw &\qw &\qw &\qw    &\qw &\qw &\ctrl{4}&\qw     &\qw     &\qw     &\qw     &\ctrl{3}&\qw &c_k                          &&&&&&&&\\
      &&&i_k            &&&\targ   &\ctrl{2}&\qw     &\qw &\qw &\qw    &\qw &\qw &\qw &\qw    &\qw &\qw &\qw &\qw    &\qw &\qw &\qw     &\ctrl{2}&\ctrl{1}&\qw     &\ctrl{1}&\targ   &\qw &i_k                          &&&&&&&&\\
      &&               &&&&        &        &        &    &    &       &    &    &    &       &    &    &    &       &    &    &        &        &        &\ctrl{1}&\qw     &        &    &    &&&&&&&&\\
      &&&t_k            &&&\targ   &\ctrl{1}&\qw     &\qw &\qw &\qw    &\qw &\qw &\qw &\qw    &\qw &\qw &\qw &\qw    &\qw &\qw &\qw     &\ctrl{1}&\qw     &\targ   &\qw     &\targ   &\qw &&&&&(t+i \cdot \text{control})_k &&&&&\\
      &&               &&&&        &        &\targ   &\qw &    &c_{k+1}&    &    &    &\ldots &    &    &    &c_{k+1}&    &    &\targ   &\qw     &        &        &        &        &    &    &&&&&&&&\\
    }
  }
  \\ \vspace*{0.25cm} \\
  (b) \resizebox{\linewidth}{!}{
    \Qcircuit @R=0.7em @C=0.7em {
      &a &&\ctrl{2}&\qw     &\qw     &\ctrl{3}&\qw &a          &&&&&&&&&&&&&&&&&&&&&&\\
      &b &&\targ   &\ctrl{1}&\ctrl{1}&\targ   &\qw &b          &&&&&&&&&&&&&&&&&&&&&&\\
      &c &&\targ   &\ctrl{1}&\targ   &\qw     &\qw &&&&(a+b+c)_0  &&&&&&&&&&&&&&&&&&&\\
      &  &&        &        &\qw     &\targ   &\qw &&&&(a+b+c)_1  &&&&&&&&&&&&&&&&&&&\\
    }
  }
  \caption{
	Variations on our adder construction.
	(a) is a controlled-adder building-block with a T-count of 8.
	The addition will only occur when the control is on.
	(b) is a temporary-adder building-block with a T-count of 4.
	Has the same T-count as a normal addition, but makes the result available before any uncomputation steps are needed.
  }
  \label{fig:variations-on-full-adder-block}
\end{figure}

Decreasing the T-count of addition reduces the T-count of any construction based on addition.
For example, in \citep{Fowler2012} it is estimated that factoring a 2048-bit number on a surface-code-based quantum computer would take 27 hours and $2 \cdot 10^{12}$ distilled $|A\rangle$ states.
The time estimate is based on Toffolis having a T-depth of 3, and the $|A\rangle$ count estimate is based on Toffolis having a T-count of 7.
Because Shor's algorithm is dominated by the cost of additions, our techniques (and previous work) reduce the average T-count and T-depth of its Toffolis to $2$ and $1$ respectively.
This reduces the estimates to 9 hours and $6 \cdot 10^{11}$ distilled $|A\rangle$ states.

On top of reducing the T-count of obviously related operations like multiplication and exponentiation, reducing the T-count of addition also reduces the T-count of quantum-specific operations such as rotating qubits.

For example, our improved adder allows the operation $R_Z(\theta)$ to be applied to $n$ qubits with a total T-cost $4n + O(\text{poly}(\frac{1}{\epsilon}) \lg n)$.
First, iteratively apply our temporary adder to compute the pop count of a register containing the target qubits.
This takes $4n + O(\lg n)$ T gates to do.
Then, for each position $k$ in the pop count output register, synthesize and apply the operation $R_Z(\theta \cdot 2^k)$ to the qubit at that position.
This uses a number of T gates proportional to some polynomial of the desired bit precision, times the number of bits in the output register.
But the output register is exponentially smaller than $n$, so this cost is negligible for large $n$.
Finally, uncompute the temporary pop count (using no T gates).
This applies the desired $R_Z(\theta)$ operations.

Another rotation-based operation that can be implemented via an adder is the $n$-bit phase gradient operation $\text{Grad}_n = \sum_{k=0}^{2^n-1} e^{2 i \pi k / 2^n} |k\rangle \langle k|$.
Normally this operation would be implemented by separately applying the operation $R_z(\pi 2^{-p})$ to each qubit of the target register, where $p$ is the qubit's index in the register and the number of T gates needed for each rotation depends on the maximum per-gate error $\epsilon$.
However, assuming a ``phase gradient register" prepared in the state $2^{-b/2} \sum_{k=0}^{2^b-1} e^{2 i \pi k / 2^b} |k\rangle$ is available, the phase gradient operation can be performed via addition using only $4n + O(1)$ T gates.
Add the target register into the phase gradient register, and phase kickback will apply the $\text{Grad}_n$ operation to the target.
Ignoring the one-time cost of initializing the reusable phase gradient register, this uses $4n + O(1)$ T gates.

Some quantum Fourier transform circuits use controlled phase gradient operations.
Controlled phase gradients can be applied in the same way as normal phase gradients, by adding into a reusable phase gradient register.
However, controlled adders are used instead of uncontrolled adders.

Assume a maximum per-gate error of $\epsilon$, start from the textbook QFT circuit based on the Cooley-Tukey algorithm \citep{Nielsen2009}, drop negligible rotations, merge individual controlled rotations into controlled phase gradients, and apply those controlled phase gradients using controlled adders.
The result is a simple approximate $n$-qubit QFT circuit that uses fewer than $8 n \lg \lg \frac{1}{\epsilon}$ T gates.

The temporary logical-AND is useful for optimizing an even wider variety of circuits than addition is.
Whenever Toffoli gates appear in compute/uncompute pairs, and intermediate operations are not sensitive to phase errors on the controls of the Toffoli gate (e.g. the condition in \autoref{fig:pair-definition} is satisfied), it is possible to save 4 T gates by replacing the pair of Toffoli gates with a temporary logical-AND.

\begin{figure}
  \resizebox{\linewidth}{!}{
    \Qcircuit @R=0.7em @C=0.7em {
      &&&\ustick{n}&\diagup\qw &\multigate{3}{U}&\qw     &\qw &&               &&&&\ustick{n}&\diagup\qw &\multigate{3}{U}&\qw     &\qw &&            &&\ustick{n}&\diagup\qw &\qw      &\multigate{3}{U}&\qw      &\qw     &\qw &\\
      &&&          &\ctrl{1}   &\ghost{U}       &\ctrl{1}&\qw &&               &&&&          &\ctrl{1}   &\ghost{U}       &\ctrl{1}&\qw &&            &&          &\ctrl{1}   &\qw      &\ghost{U}       &\qw      &\ctrl{1}&\qw &\\
      &&&          &\ctrl{1}   &\ghost{U}       &\ctrl{1}&\qw &&\stackrel{?}{=}&&&&          &\ctrl{2}   &\ghost{U}       &\ctrl{2}&\qw &&\Rightarrow &&          &\ctrl{2}   &\qw      &\ghost{U}       &\qw      &\ctrl{2}&\qw &\\
      &&&          &\targ      &\ghost{U}       &\targ   &\qw &&               &&&&          &\targ      &\ghost{U}       &\targ   &\qw &&            &&          &\qw        &\targ    &\ghost{U}       &\targ    &\qw     &\qw &\\
      &&&\qO       &\qw        &\qw             &\qw     &\qw &&               &&&&\qO       &\targ      &\qw             &\targ   &\qw &&            &&          &           &\ctrl{-1}&\qw             &\ctrl{-1}&\qw     &    &\\
    }
  }
  \caption{
	A sufficient condition for replacing a pair of Toffoli gates with a temporary logical-AND, saving 4 T gates.\\
	1) The later Toffoli gate must be uncomputing the earlier Toffoli gate.\\
	2) Intermediate operations must not be sensitive to the presence of the entangled ancilla.
  }
  \label{fig:pair-definition}
\end{figure}

For example, the Grover diffusion operator's cost is dominated by the cost of merging $n$ controls.
Merging the controls can be computed, and then uncomputed, with Toffoli gates.
By instead iteratively merging the controls with nested temporary logical-ANDs, the Grover diffusion operation can be applied using $4n + O(1)$ T gates.

For other cases where temporary logical-ANDs can be used to optimize circuits, we refer back to the introduction.
Even circuits based on simulating physical systems, not just abstract mathematical problems, can benefit \citep{RyanEmails2017}.


\section{Conclusion}
\label{sec:conclusion}

For over a decade, the T-count of addition has been $8n - 4$ \citep{Amy2013, Barenco1995, Cuccaro2004}.
This was conjectured to be optimal \citep{AustinDiscussionsAndEmails2017}.
We halved the leading factor of this cost, and we did so via a construction that applies to many other circuits: the temporary logical-AND gate.
Our work represents a significant and widely applicable improvement to the state of the art in synthesizing quantum circuits with low T gate counts.

We are unsure if addition or temporary logical-AND gates can be done with fewer T gates.
A simple argument based on synthesizing T gates proves that the T-count of the temporary logical-AND is at least 2 in \autoref{fig:lower-bound-logical-AND}.
Two is less than four, but we are not aware of any method for further reducing the T-count of temporary logical-ANDs or for synthesizing more T gates out of a temporary logical-AND.

\begin{figure}
  \resizebox{\linewidth}{!}{
    \Qcircuit @R=0.7em @C=0.7em {
      &\qT &\qw && &&&\ctrl{2}&\qw     &\ctrl{3}&\qw     &\qw &\qw     &\ctrl{3}&\qw     &\ctrl{2}&\qw &\\
      &\qT &\qw &&=&&&\targ   &\ctrl{1}&\qw     &\ctrl{1}&\qw &\ctrl{1}&\qw     &\ctrl{1}&\targ   &\qw &\\
      &\qT &\qw && &&&\targ   &\ctrl{1}&\targ   &\targ   &\qT &\targ   &\targ   &\ctrl{1}&\targ   &\qw &\\
      &    &    && &&&        &        &\targ   &\qw     &\qS &\qw     &\targ   &\qw     &        &    &\\
    }
  }
  \caption{
	Three T gates can be synthesized by performing a temporary logical-AND, one T gate, and various Clifford operations.
	Clifford operations can't produce T gates, so the extra T gates must be coming from the temporary logical-AND.
	This proves that at least two T gates went into performing the temporary logical-AND.
  }
  \label{fig:lower-bound-logical-AND}
\end{figure}

Regardless, it is clear that other T-count improvements as significant as the ones we have made in this paper are waiting to be found.
Not just in the construction of basic low-level operations, but in medium-level constructions and high-level constructions and generally across the whole technology stack that will be needed to perform error corrected quantum computation with the surface code for the first time.


\section{Acknowledgements}

We thank Austin Fowler for assistance in locating relevant references, sampling the opinions of other researchers, and for comments that greatly improved our paper.


\bibliographystyle{plainnat}
\bibliography{citations}

\end{document}
