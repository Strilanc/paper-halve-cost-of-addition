\documentclass[twocolumn,longbibliography]{quantumarticle-customized}
\usepackage{amsmath}
\usepackage{graphicx}
\usepackage[pdfpagelabels,pdftex,bookmarks,breaklinks]{hyperref}
\usepackage{tikz}
\usepackage[all]{hypcap}
\hypersetup{colorlinks,citecolor=blue,urlcolor=blue,linkcolor=blue}

\input{Qcircuit}
\newcommand{\qH}{\gate{H}}
\newcommand{\qT}{\gate{T}}
\newcommand{\qTi}{\gate{T^\dagger}}
\newcommand{\qS}{\gate{S}}
\newcommand{\qSi}{\gate{S^\dagger}}
\newcommand{\qA}{\lstick{|A\rangle}}
\newcommand{\qO}{\lstick{|0\rangle}}

\title{T gates, temporary ANDs, and halving the cost of quantum addition}
\author{Craig Gidney}
\affiliation{Google, Santa Barbara, CA 93117, USA}
\email{craiggidney@google.com}

\def\sectionautorefname{Section}

\begin{document}
\maketitle

\begin{abstract}
We improve the number of T gates needed to perform an $n$-bit adder from $8n + O(1)$ \cite{Amy2013, AustinDiscussionsAndEmails2017} to $4n + O(1)$.
We do so via a ``temporary AND gate" construction, which uses four T gates to store the logical-and of two qubits into an ancilla and zero T gates to later erase the ancilla.

Temporary AND gates are a generally useful tool when optimizing T-cost.
They can be applied to integer arithmetic, modular arithmetic, rotation synthesis, the quantum Fourier transform, Shor's algorithm, Grover oracles, and many other circuits.
Because T gates dominate the cost of quantum computation based on the surface code, and the temporary AND gate is widely applicable, our constructions represent a significant reduction in projected costs of quantum computation.

In addition to our $n$-bit adder circuit with T-cost of $4n-4$, we construct an $n$-bit controlled adder circuit with T-cost of $8n-4$, a temporary adder circuit that can be computed for the same cost as the normal adder but whose result can be kept until later uncomputed at no T-cost, an approximate quantum Fourier transform circuit with T-cost of $8 n \lg b + O(b (\lg n) (\lg b))$ where $b = \lg \frac{1}{\epsilon}$ is the desired bit-precision of the operation, and a circuit to rotate $n$ qubits by any angle $\theta$ with T-cost of $4n + O(b \lg n)$.
\end{abstract}


\section{Introduction}
\label{sec:introduction}

The surface code is a quantum error correcting code that works on a 2D nearest-neighbour array of qubits and achieves a threshold error rate of approximately 1\% [[[[cite so much]]]].
This makes the surface code a likely component in the architecture of future error corrected quantum computers, because 2D arrays of qubits with nearest-neighbor connections are possible with many qubit technologies [[[[cite so much]]]] and other known error correcting codes have lower thresholds or require stronger connectivity [[[[cite so much]]]].

One of the downsides of the surface code is that it has no cheap mechanism to apply non-Clifford operations such as $T$ gates.
Instead, $T$ gates are performed by distilling and consuming $|A\rangle = \frac{1}{\sqrt{2}} (|0\rangle + e^{i \pi/4} |1\rangle)$ states.
Consuming an $|A\rangle$ state to perform a T gate is simple, but distilling $|A\rangle$ states has significant cost.
Because T gates are so expensive for the surface code, and the surface code is a likely component of future quantum computers, it is important to consider and optimize the number of T gates used by quantum circuits.

In this paper, we focus on improving the number of T gates needed to perform Toffoli gates that will later be uncomputed by a second Toffoli gate.
We then apply these improvements.
The key ideas behind our construction are 1) perform the Toffoli indirectly by targeting a clean ancilla qubit and then using the ancilla to toggle the intended target, 2) don't uncompute and recompute the ancilla if it will be needed again, 3) if a T gate is being used to compute and/or uncompute an $|A\rangle$ state then just pass in and/or recover an $|A\rangle$ state instead, and 4) uncompute the ancilla by measuring it then applying a classically controlled fixup operation.
We refer to initializing the ancilla as ``computing the logical-AND of the controls", to uncomputing the ancilla as ``erasing the logical-AND", and to the combination of both pieces as a ``temporary AND gate".

Our paper is divided into six sections.
In \autoref{sec:introduction}, we motivate the problem and introduce terminology.
\autoref{sec:review} discusses how previous work on optimizing T gate counts improved the cost of matched pairs of Toffolis from 14 to 8.
In \autoref{sec:invest}, we show how to improve the T-cost from 8 to 6 by investing and later recovering an $|A\rangle$ state instead of consuming it to perform T gates.
\autoref{sec:temporary-and} expands on those ideas with an ancilla erasing trick that improves the T-cost from 6 to 4.
This leads to \autoref{sec:circuit-constructions}, where we demonstrate how to use our constructions to create low T-cost circuits for several basic arithmetic tasks (including cutting the T-cost of addition in half, with respect to previously reported circuits).
Finally, \autoref{sec:conclusion} summarizes our contributions, proves that at least two T gates are needed to perform a temporary AND, and discusses other possible applications of our constructions.


\section{Previous Work}
\label{sec:review}

The textbook construction of a Toffoli gate uses seven T gates \cite{Nielsen2009} (see \autoref{fig:textbook-toffoli}).
Assuming we aren't permitted to involve other qubits or to share work with other operations, this construction is optimal \cite{Gosset2014}.

When optimizations can cross qubit and gate boundaries, the textbook construction isn't optimal.
For example, when several adjacent Toffoli gates share the same controls, all but one can be replaced by CNOT operations (see \autoref{fig:shared-controls}).

\begin{figure}
  \resizebox{\linewidth}{!}{
    \Qcircuit @R=1.5em @C=0.7em {
      &\ctrl{1}&\qw & &   & & &\ctrl{1}&\qw  &\ctrl{1}&\qT &\qw     &\qw  &\ctrl{2}&\qw &\qw     &\qw  &\ctrl{2}&\qw &\qw \\
      &\ctrl{1}&\qw & & = & & &\targ   &\qTi &\targ   &\qT &\ctrl{1}&\qw  &\qw     &\qw &\ctrl{1}&\qw  &\qw     &\qw &\qw \\
      &\targ   &\qw & &   & & &\qw     &\qw  &\qH     &\qT &\targ   &\qTi &\targ   &\qT &\targ   &\qTi &\targ   &\qH &\qw \\
    }
  }
  \caption{
	Textbook Toffoli construction from \cite{Nielsen2009}.
	Uses eight Clifford gates and seven T gates.
  }
  \label{fig:textbook-toffoli}
\end{figure}

\begin{figure}
  \resizebox{\linewidth}{!}{
    \Qcircuit @R=1.5em @C=0.7em {
      &\ctrl{1}&\ctrl{1}&\ctrl{1}&\ctrl{1}&\qw && &&&\qw     &\qw     &\qw     &\ctrl{1}&\qw     &\qw     &\qw     &\qw &\\
      &\ctrl{1}&\ctrl{2}&\ctrl{3}&\ctrl{4}&\qw &&=&&&\qw     &\qw     &\qw     &\ctrl{1}&\qw     &\qw     &\qw     &\qw &\\
      &\targ   &\qw     &\qw     &\qw     &\qw && &&&\ctrl{3}&\ctrl{2}&\ctrl{1}&\targ   &\ctrl{1}&\ctrl{2}&\ctrl{3}&\qw &\\
      &\qw     &\targ   &\qw     &\qw     &\qw && &&&\qw     &\qw     &\targ   &\qw     &\targ   &\qw     &\qw     &\qw &\\
      &\qw     &\qw     &\targ   &\qw     &\qw && &&&\qw     &\targ   &\qw     &\qw     &\qw     &\targ   &\qw     &\qw &\\
      &\qw     &\qw     &\qw     &\targ   &\qw && &&&\targ   &\qw     &\qw     &\qw     &\qw     &\qw     &\targ   &\qw &\\
    }
  }
  \caption{
	The T-cost of $N$ adjacent Toffolis sharing the same controls is $0 \cdot N + O(1)$.
	The marginal T-cost is 0 because each additional Toffoli can be replaced with CNOTs framing a root Toffoli.
  }
  \label{fig:shared-controls}
\end{figure}

It isn't common for adjacent Toffolis to have the same controls, but it is common for a Toffoli to later be uncomputed by a second matching Toffoli (i.e. for the Toffoli's effect to be temporary).
When this occurs, the three $T$ gates on the control qubits of the textbook construction can be omitted.
This introduces phase errors (see \autoref{fig:bad-phase-toffoli}), but the second Toffoli gate can uncompute those errors while uncomputing the state permutation \cite{Barenco1995} (see \autoref{fig:cancelled-bad-phase-toffoli}).

\begin{figure}
  \resizebox{\linewidth}{!}{
    \Qcircuit @R=1.5em @C=0.7em {
      &\ctrl{1} &\qw & &       & & &\ctrl{1}     &\ctrl{1} &\qw & &   & & &\qw &\qw &\qw      &\qw  &\ctrl{2} &\qw &\qw      &\qw  &\qw &\qw \\
      &\ctrl{1} &\qw & & \cong & & &\qS          &\ctrl{1} &\qw & & = & & &\qw &\qw &\ctrl{1} &\qw  &\qw      &\qw &\ctrl{1} &\qw  &\qw &\qw \\
      &\targ    &\qw & &       & & &\gate{Z}\qwx &\targ    &\qw & &   & & &\qH &\qT &\targ    &\qTi &\targ    &\qT &\targ    &\qTi &\qH &\qw \\
    }
  }
  \caption{
	Starting with \autoref{fig:textbook-toffoli} then dropping T gates on the controls produces an operation with a T-cost of 4 that performs the correct permutation.
	However, the operation introduces phase errors.
  }
  \label{fig:bad-phase-toffoli}
\end{figure}

\begin{figure}
  \resizebox{\linewidth}{!}{
    \Qcircuit @R=1.5em @C=0.7em {
      &\ctrl{1}&\qw &\qw &\qw    &\qw &\qw &\ctrl{1}&\qw && &&&\ctrl{1}     &\ctrl{1}  &\qw &\qw &\qw    &\qw &\qw     &\ctrl{1}&\ctrl{1}     &\qw & \\
      &\ctrl{1}&\qw &\qw &\qw    &\qw &\qw &\ctrl{1}&\qw &&=&&&\qS          &\ctrl{1}  &\qw &\qw &\qw    &\qw &\qw     &\ctrl{1}&\qSi         &\qw & \\
      &\targ   &\qw &    &\ldots &    &    &\targ   &\qw && &&&\gate{Z}\qwx &\targ     &\qw &    &\ldots &    &        &\targ   &\gate{Z}\qwx &\qw & \\
    }
  }
  \caption{
	When two Toffolis form a compute/uncompute pair, they can cancel each others' phase errors.
	Fixes the problem with the construction in \autoref{fig:bad-phase-toffoli}, and achieves a per-Toffoli T-cost of 4 for paired Toffolis \cite{Barenco1995}.
  }
  \label{fig:cancelled-bad-phase-toffoli}
\end{figure}

A typical $n$-bit quantum adder will contain $2n + O(1)$ Toffoli gates, implying a naive T-cost of $14n + O(1)$.
However, almost all of the Toffoli gates in the first half of an adder are uncomputed by Toffoli gates in the second half.
This allows the T gates on the controls of the Toffoli gates to be omitted, reducing their T-cost from 7 to 4 and the T-cost of addition to $8n + O(1)$.

Even if a Toffoli isn't paired with a second Toffoli that uncomputes its effects, it's still possible to perform the Toffoli with T-cost of by using an ancilla, a measurement, and a conditional fixup operation \cite{Jones2013} (see \autoref{fig:jones-toffoli}).
Our key improvement over \cite{Jones2013} is to notice that the ancilla's lifetime can be extended, so it can be used both when computing and uncomputing a Toffoli.

\begin{figure}
  \resizebox{\linewidth}{!}{
    \Qcircuit @R=1.5em @C=0.7em {
      &\ctrl{1} &\qw & &   & & &     &\qw &\qw &\qw      &\qw  &\ctrl{3} &\qw &\qw      &\qw  &\qw &\qw  &\qw       &\qw &\qw    &\ctrl{1}       &\qw \\
      &\ctrl{1} &\qw & & = & & &     &\qw &\qw &\ctrl{2} &\qw  &\qw      &\qw &\ctrl{2} &\qw  &\qw &\qw  &\qw       &\qw &\qw    &\gate{Z}       &\qw \\
      &\targ    &\qw & &   & & &     &\qw &\qw &\qw      &\qw  &\qw      &\qw &\qw      &\qw  &\qw &\qw  &\targ     &\qw &\qw    &\qw\cwx        &\qw \\
      &         &    & &   & & & \qO &\qH &\qT &\targ    &\qTi &\targ    &\qT &\targ    &\qTi &\qH &\qSi &\ctrl{-1} &\qH &\meter &\cw\cwx\bullet & \\
    }
  }
  \caption{
	Toffoli construction with T-cost of 4 from \cite{Jones2013}.
  }
  \label{fig:jones-toffoli}
\end{figure}


\section{Investing $|A\rangle$ states}
\label{sec:invest}

This paper started with an idea that improved the T-cost of a Toffoli from 7 to 6 in a surprising way.
Instead of performing a Toffoli directly onto the target qubit, we tried performing the Toffoli indirectly.
We started with a clean ancilla in the $|0\rangle$ state, applied a Toffoli to store the logical-and of the two controls in the ancilla, used the ancilla to control a CNOT onto the intended target, then uncomputed the ancilla.

Initially, this indirect-Toffoli construction appeared to have a T-cost of 8.
However, as shown in \autoref{fig:indirect-toffoli}, the last T gate in the circuit is unnecessary.
In fact, it is actively harmful.
Dropping this T gate and the following Hadamard not only reduces the T-cost from 8 to 7, it leaves the ancilla in an $|A\rangle$ state that can be consumed to perform a T gate elsewhere.
This improves the net T-cost to 6.

\begin{figure}
  \resizebox{\linewidth}{!}{
    \Qcircuit @R=1.5em @C=0.7em {
      &\ctrl{1} &\qw && &&&    &\qw &\qw &\qw      &\qw  &\ctrl{3} &\qw &\qw      &\qw  &\qw &\qw       &\qw &\qw &\qw      &\qw  &\ctrl{3} &\qw &\qw      &\qw  &\qw &\qw &&\\
      &\ctrl{1} &\qw &&=&&&    &\qw &\qw &\ctrl{2} &\qw  &\qw      &\qw &\ctrl{2} &\qw  &\qw &\qw       &\qw &\qw &\ctrl{2} &\qw  &\qw      &\qw &\ctrl{2} &\qw  &\qw &\qw &&\\
      &\targ    &\qw && &&&    &\qw &\qw &\qw      &\qw  &\qw      &\qw &\qw      &\qw  &\qw &\targ     &\qw &\qw &\qw      &\qw  &\qw      &\qw &\qw      &\qw  &\qw &\qw &&\\
      &         &    && &&&\qO &\qH &\qT &\targ    &\qTi &\targ    &\qT &\targ    &\qTi &\qH &\ctrl{-1} &\qH &\qT &\targ    &\qTi &\targ    &\qT &\targ    &\qTi &\qH &\qw 
          \gategroup{4}{6}{4}{10}{.7em}{--} \gategroup{4}{26}{4}{30}{.7em}{--} &|0\rangle & \\
      &         &    && &&&    &    |A\rangle &  & &     &         &    &         &     &    &          &    &    &         &    &         &     &         &     & |A\rangle
    }
  }
  \caption{
	Performing a Toffoli indirectly, by applying a pair of Toffolis to an ancilla and using the intermediate value to control a CNOT onto the actual target, appears to have a T-cost of 8.
	However, the initial T gate is being used to compute an $|A\rangle$ state and the final T gate is being used to uncompute the $|A\rangle$ state.
    Instead of spending $|A\rangle$ states to perform T gates to make and unmake $|A\rangle$ states, we pass in and later recover a single $|A\rangle$ state.
	This ``investment" reduces the net T-cost to 6.
  }
  \label{fig:indirect-toffoli}
\end{figure}

A Toffoli with T-cost of 6 isn't optimal, but the reduction from 7 was done in an unexpected way.
While searching for other circuits where this optimization might apply, we realized it would be useful in the paired-Toffolis case.
Instead of computing and uncomputing the ancilla for the first Toffoli, then recomputing and re-uncomputing it for the second Toffoli, we simply kept the ancilla from the first Toffoli so that the second Toffoli could also use it.
This separated the construction into a control-combining step and an entanglement-erasing step.

Combining the controls had a net T-cost of 4, because it involved performing three $T$ gates and investing one $|A\rangle$ state.
But erasing the entanglement had a net T-cost of 2 because, although it performed three $T$ gates, it recovered the $|A\rangle$ state invested when combining the controls.
\autoref{fig:ancilla-temporary-toffoli} shows how to replace a pair of Toffolis, where the second Toffoli is uncomputing the first, with this construction.

\begin{figure}
  \resizebox{\linewidth}{!}{
    \Qcircuit @R=1.5em @C=0.7em {
      &\ctrl{1}&\qw &\qw &\qw    &\qw &\qw &\ctrl{1}&\qw && &&&    &\qw     &\qw  &\ctrl{3}&\qw &\qw     &\qw  &\qw &\qw      &\qw &\qw &\qw    &\qw &\qw &\qw      &\qw &\qw &\qw     &\qw  &\ctrl{3}&\qw &\qw     &\qw &&\\
      &\ctrl{1}&\qw &\qw &\qw    &\qw &\qw &\ctrl{1}&\qw &&=&&&    &\ctrl{2}&\qw  &\qw     &\qw &\ctrl{2}&\qw  &\qw &\qw      &\qw &\qw &\qw    &\qw &\qw &\qw      &\qw &\qw &\ctrl{2}&\qw  &\qw     &\qw &\ctrl{2}&\qw &&\\
      &\targ   &\qw &    &\ldots &    &    &\targ   &\qw && &&&    &\qw     &\qw  &\qw     &\qw &\qw     &\qw  &\qw &\targ    &\qw &    &\ldots &    &    &\targ    &\qw &\qw &\qw     &\qw  &\qw     &\qw &\qw     &\qw &&\\
      &        &    &    &       &    &    &        &    && &&&\qA &\targ   &\qTi &\targ   &\qT &\targ   &\qTi &\qH &\ctrl{-1}&\qw &\qw &\qw    &\qw &\qw &\ctrl{-1}&\qH &\qT &\targ   &\qTi &\targ   &\qT &\targ   &\qw &|A\rangle &\\
    }
  }
  \caption{
	Computing and uncomputing a Toffoli gate with a net T-cost of 6.
	Invests an $|A\rangle$ state when computing the Toffoli, holds an ancilla qubit storing the logical-and of the controls until it's time to uncompute the Toffoli, then recovers the $|A\rangle$ state.
  }
  \label{fig:ancilla-temporary-toffoli}
\end{figure}

Many circuits involve computing and later uncomputing a Toffoli (e.g. addition), and investing an $|A\rangle$ state reduced the T-cost of doing so from 8 to 6.
We realized this improved on existing work, and began writing this paper.
We then found a construction that achieves the same effect with lower T-cost, which we will cover in the next section.
However, there are likely circuits without temporary logical-ANDs that nevertheless would benefit from investing and recovering $|A\rangle$ states.
It's an interesting fallback optimization, which we mention for completeness.


\section{The temporary AND gate}
\label{sec:temporary-and}

In the previous section, we showed how to compute and uncompute a Toffoli gate by investing an $|A\rangle$ state and performing three $T$ gates.
This construction be improved in several ways, ultimately producing our temporary AND gate construction.

First, the computation step is introducing phase errors (the same as in \autoref{fig:bad-phase-toffoli}).
Correcting these phase errors requires applying a controlled-Z gate involving the target (before the Toffoli gate), and applying a controlled-S gate to the two controls.
We know our target is $|0\rangle$ beforehand, so the controlled-Z has no effect and can be ignored.
The effect of the controlled-S gate is to apply a phase factor of $i$ to the amplitudes of computational basis states where both controls on.
We happen to be computing a qubit storing whether or not both controls are on, so we can replace the controlled-S with an uncontrolled S gate on the ancilla.
This corrects all of the phase errors that were being introduced.

Second, the construction from the previous section is unnecessarily deep.
Each T gate is hiding a measurement and an S gate performed conditional on the outcome of that measurement.
It is beneficial to be waiting for several measurement results in parallel, instead of one after another serially.
As shown in \cite{Jones2013}, the T gates of a Toffoli can all be performed at the same time, and we do so.
Note that still pass an $|A\rangle$ state into the circuit instead of performing a fourth T gate, because passing in an $|A\rangle$ state already doesn't involve measuring and reacting.

Third, taking another hint from \cite{Jones2013}, we can uncompute the ancilla by using a measure-and-correct process instead of with a mirror of the computation circuit.
To uncompute the ancilla produced by the logical AND, we use a technique we refer to as ``erasure".
We note that a Toffoli gate would obviously clear the value.
Then, since the cleared ancilla was going to be discarded anyways, we are free to apply a Hadamard gate and a measurement to it.
We then hop the Hadamard over the Toffoli, transforming it into a CCZ operation.
Next we re-arrange the CCZ so that the ancilla is one of the controls, which is possible because the controls and targets of a CCZ are interchangeable.
We then hop the Measurement over the CCZ, turning the quantum control into a classical control.
Ultimately, the Toffoli gate we started with has been turned into a CZ gate that we apply or not based on the outcome of the measurement.
All of the remaining gates are Clifford operations, which have no T-cost.

With these three improvements applied, we are left with the key ingredients of our temporary AND gate.
We can initialize an ancilla storing the logical-AND of two qubits.
The creation process has a T-cost of 4.
Then we can keep the ancilla around as long as we want, using it as a control for operations that would otherwise have been conditioned on both input qubits.
Finally, once the ancilla is no longer needed, we can erase it at no T-cost.

In diagrams, we draw the computation of the AND gate (i.e. the ancilla being initialized) as an ancilla wire emerging vertically from two controls then heading rightward (see \autoref{fig:compute-logical-and}).
We draw the uncomputation analogously, with the ancilla wire coming in from the left then merging vertically into the two controls that created it (seee \autoref{fig:erase-logical-and}).
The initialized lifetime of the ancilla qubit corresponds to the span of the wire coming out of and later merging into the control qubits.

\begin{figure}
  \resizebox{\linewidth}{!}{
    \Qcircuit @R=1.5em @C=0.7em {
      &x &&\ctrl{1} &\qw & x  && &&          &&\targ    &\ctrl{2}&\qw     &\qTi &\qw     &\ctrl{2}&\targ     &\qw &\qw &\qw &\\
      &y &&\ctrl{1} &\qw & y  &&=&&          &&\targ    &\qw     &\ctrl{1}&\qTi &\ctrl{1}&\qw     &\targ     &\qw &\qw &\qw &\\
      &  &&         &\qw & xy && &&|A\rangle &&\ctrl{-2}&\targ   &\targ   &\qT  &\targ   &\targ   &\ctrl{-2} &\qH &\qS &\qw &\\
    }
  }
  \caption{
	Computing the logical-and of two qubits, with a T-cost of 4 and a measure-react depth of 1.
  }
  \label{fig:compute-logical-and}
\end{figure}

\begin{figure}
  \resizebox{\linewidth}{!}{
    \Qcircuit @R=1.5em @C=0.7em {
      &x  &&\ctrl{1} &\qw &x && &&\qw &\qw    &\ctrl{1} &\qw \\
      &y  &&\ctrl{1} &\qw &y &&=&&\qw &\qw    &\gate{Z} &\qw \\
      &xy &&\qw      &    &  && &&\qH &\meter &\cw \cwx \bullet &    \\
    }
  }
  \caption{
	Uncomputing the logical-and of two qubits, with a T-cost of 0 and a measure-react depth of 1.
  }
  \label{fig:erase-logical-and}
\end{figure}

We can perform a single Toffoli by computing the logical-and of two qubits , applying a CNOT from the logical-and onto the actual target, then uncomputing the logical-and.
This combined construction, shown in \autoref{fig:merge-use-erase-toffoli}, is equivalent to the one in \cite{Jones2013}.

\begin{figure}
  \resizebox{\linewidth}{!}{
    \Qcircuit @R=1.5em @C=0.7em {
      &\ctrl{1} &\qw && &&\ctrl{1} &\qw      &\ctrl{1} &\qw && &&          &&\targ    &\ctrl{2}&\qw     &\qTi &\qw     &\ctrl{2}&\targ     &\qw &\qw &\qw      &\qw &\qw    &\ctrl{1}         &\qw \\
      &\ctrl{2} &\qw && &&\ctrl{1} &\qw      &\ctrl{1} &\qw && &&          &&\targ    &\qw     &\ctrl{1}&\qTi &\ctrl{1}&\qw     &\targ     &\qw &\qw &\qw      &\qw &\qw    &\gate{Z}         &\qw \\
      &         &    &&=&&         &\ctrl{1} &\qw      &    &&=&&|A\rangle &&\ctrl{-2}&\targ   &\targ   &\qT  &\targ   &\targ   &\ctrl{-2} &\qH &\qS &\ctrl{1} &\qH &\meter &\cw \cwx \bullet &    \\
      &\targ    &\qw && &&\qw      &\targ    &\qw      &\qw && &&          &&\qw      &\qw     &\qw     &\qw  &\qw     &\qw     &\qw       &\qw &\qw &\targ    &\qw &\qw    &\qw              &\qw \\
    }
  }
  \caption{
	Performing a Toffoli gate by computing the logical-and of its controls, using the result, then erasing it.
	Has a T-cost of 4 and a measure-react depth of 2.
	Equivalent to the construction from \cite{Jones2013} shown in \autoref{fig:jones-toffoli}.
  }
  \label{fig:merge-use-erase-toffoli}
\end{figure}

Conceptually, our improvement on the construction in \cite{Jones2013} is merely noticing that it can be split it useful pieces that don't all need to be applied at the same time.
In particular, when a Toffoli is later uncomputed by a second Toffoli, we simply delay erasing the ancilla wire as shown in \autoref{fig:paired-toffoli-to-logical-and}.
This reduces the T-cost of such paired Toffolis from 8 to 4.

\begin{figure}
  \resizebox{\linewidth}{!}{
    \Qcircuit @R=1.5em @C=0.7em {
      &\ctrl{1} &\qw &\qw &\qw    &\qw &\qw &\ctrl{1} &\qw && &&\ctrl{1} &\qw      &\qw &\qw &\qw    &\qw &\qw &\qw      &\ctrl{1} &\qw & \\
      &\ctrl{2} &\qw &\qw &\qw    &\qw &\qw &\ctrl{2} &\qw && &&\ctrl{1} &\qw      &\qw &\qw &\qw    &\qw &\qw &\qw      &\ctrl{1} &\qw & \\
      &         &    &    &       &    &    &         &    &&=&&         &\ctrl{1} &\qw &\qw &\qw    &\qw &\qw &\ctrl{1} &\qw      &    & \\
      &\targ    &\qw &    &\ldots &    &    &\targ    &\qw && &&\qw      &\targ    &\qw &    &\ldots &    &    &\targ    &\qw      &\qw & \\
    }
  }
  \caption{
	Replacing a pair of compute/uncompute Toffolis with a temporary AND gate and Clifford operations.
	Improves the T-cost from 8 to 4.
  }
  \label{fig:paired-toffoli-to-logical-and}
\end{figure}


\section{T-optimized Circuit Constructions}
\label{sec:circuit-constructions}

In this section we will show how to apply temporary Toffolis to improve the complexity of several circuits.

First, we note that existing adder constructions contain temporary Toffolis.
The T-cost of adding two $n$-bit numbers was thought to be $8n + O(1)$.
We replace the basic adder unit with the building block shown in \ref{fig:full-adder-block}, putting it together as shown in \ref{fig:multi-bit-adder-example} in order to create an adder with a T-cost of $4n + O(1)$.

\begin{figure}
  \resizebox{\linewidth}{!}{
    \Qcircuit @R=1.5em @C=0.7em {
      &c_k &&\ctrl{2}&\qw     &\ctrl{3}&\qw &\qw &\qw    &\qw &\qw &\qw &\qw    &\qw &\qw &\qw &\qw    &\qw &\qw &\ctrl{3}&\qw     &\ctrl{1}&\qw     &\qw &c_k     &&&&& \\
      &i_k &&\targ   &\ctrl{1}&\qw     &\qw &\qw &\qw    &\qw &\qw &\qw &\qw    &\qw &\qw &\qw &\qw    &\qw &\qw &\qw     &\ctrl{1}&\targ   &\ctrl{1}&\qw &i_k     &&&&& \\
      &t_k &&\targ   &\ctrl{1}&\qw     &\qw &\qw &\qw    &\qw &\qw &\qw &\qw    &\qw &\qw &\qw &\qw    &\qw &\qw &\qw     &\ctrl{1}&\qw     &\targ   &\qw &&&(t+i)_k &&& \\
      &    &&        &        &\targ   &\qw &    &c_{k+1}&    &    &    &\ldots &    &    &    &c_{k+1}&    &    &\targ   &\qw     &        &        &    &&&        &&& \\
    }
  }
  \caption{
	Adder building block with a T-cost of 4.
  }
  \label{fig:full-adder-block}
\end{figure}

\begin{figure}
  \resizebox{\linewidth}{!}{
    \Qcircuit @R=1.5em @C=0.7em {
      &i_0 &&\ctrl{1} &\qw     &\qw     &\qw     &\qw &\qw     &\qw     &\qw     &\qw &\qw     &\qw     &\qw     &\qw     &\qw     &\qw     &\qw     &\qw     &\qw     &\qw     &\qw     &\qw     &\qw     &\ctrl{1}&\ctrl{1}&\qw &i_0     &&&&&\\
      &t_0 &&\ctrl{1} &\qw     &\qw     &\qw     &\qw &\qw     &\qw     &\qw     &\qw &\qw     &\qw     &\qw     &\qw     &\qw     &\qw     &\qw     &\qw     &\qw     &\qw     &\qw     &\qw     &\qw     &\ctrl{1}&\targ   &\qw &&&(t+i)_0 &&&\\
      &    &&         &\ctrl{2}&\qw     &\ctrl{3}&\qw &\qw     &\qw     &\qw     &\qw &\qw     &\qw     &\qw     &\qw     &\qw     &\qw     &\qw     &\qw     &\qw     &\qw     &\ctrl{3}&\qw     &\ctrl{1}&\qw     &        &    &&&        &&&\\
      &i_1 &&\qw      &\targ   &\ctrl{1}&\qw     &\qw &\qw     &\qw     &\qw     &\qw &\qw     &\qw     &\qw     &\qw     &\qw     &\qw     &\qw     &\qw     &\qw     &\qw     &\qw     &\ctrl{1}&\targ   &\qw     &\ctrl{1}&\qw &i_1     &&&&&\\
      &t_1 &&\qw      &\targ   &\ctrl{1}&\qw     &\qw &\qw     &\qw     &\qw     &\qw &\qw     &\qw     &\qw     &\qw     &\qw     &\qw     &\qw     &\qw     &\qw     &\qw     &\qw     &\ctrl{1}&\qw     &\qw     &\targ   &\qw &&&(t+i)_1 &&&\\
      &    &&         &        &        &\targ   &\qw &\ctrl{2}&\qw     &\ctrl{3}&\qw &\qw     &\qw     &\qw     &\qw     &\qw     &\qw     &\qw     &\ctrl{3}&\qw     &\ctrl{1}&\targ   &\qw     &        &        &        &    &&&        &&&\\
      &i_2 &&\qw      &\qw     &\qw     &\qw     &\qw &\targ   &\ctrl{1}&\qw     &\qw &\qw     &\qw     &\qw     &\qw     &\qw     &\qw     &\qw     &\qw     &\ctrl{1}&\targ   &\qw     &\qw     &\qw     &\qw     &\ctrl{1}&\qw &i_2     &&&&&\\
      &t_2 &&\qw      &\qw     &\qw     &\qw     &\qw &\targ   &\ctrl{1}&\qw     &\qw &\qw     &\qw     &\qw     &\qw     &\qw     &\qw     &\qw     &\qw     &\ctrl{1}&\qw     &\qw     &\qw     &\qw     &\qw     &\targ   &\qw &&&(t+i)_2 &&&\\
      &    &&         &        &        &        &    &        &        &\targ   &\qw &\ctrl{2}&\qw     &\ctrl{3}&\qw     &\ctrl{3}&\qw     &\ctrl{1}&\targ   &\qw     &        &        &        &        &        &        &    &&&        &&&\\
      &i_3 &&\qw      &\qw     &\qw     &\qw     &\qw &\qw     &\qw     &\qw     &\qw &\targ   &\ctrl{1}&\qw     &\qw     &\qw     &\ctrl{1}&\targ   &\qw     &\qw     &\qw     &\qw     &\qw     &\qw     &\qw     &\ctrl{1}&\qw &i_3     &&&&&\\
      &t_3 &&\qw      &\qw     &\qw     &\qw     &\qw &\qw     &\qw     &\qw     &\qw &\targ   &\ctrl{1}&\qw     &\qw     &\qw     &\ctrl{1}&\qw     &\qw     &\qw     &\qw     &\qw     &\qw     &\qw     &\qw     &\targ   &\qw &&&(t+i)_3 &&&\\
      &    &&         &        &        &        &    &        &        &        &    &        &        &\targ   &\ctrl{2}&\targ   &\qw     &        &        &        &        &        &        &        &        &        &    &&&        &&&\\
      &i_4 &&\qw      &\qw     &\qw     &\qw     &\qw &\qw     &\qw     &\qw     &\qw &\qw     &\qw     &\qw     &\qw     &\qw     &\qw     &\qw     &\qw     &\qw     &\qw     &\qw     &\qw     &\qw     &\qw     &\ctrl{1}&\qw &i_4     &&&&&\\
      &t_4 &&\qw      &\qw     &\qw     &\qw     &\qw &\qw     &\qw     &\qw     &\qw &\qw     &\qw     &\qw     &\targ   &\qw     &\qw     &\qw     &\qw     &\qw     &\qw     &\qw     &\qw     &\qw     &\qw     &\targ   &\qw &&&(t+i)_4 &&&\\
    }
  }
  \caption{
	5-bit adder with T-cost of 16.
	General $n$-bit adder has T-cost of $4n - 4$.
  }
  \label{fig:multi-bit-adder-example}
\end{figure}

\begin{figure}
  \resizebox{\linewidth}{!}{
    \Qcircuit @R=1.5em @C=0.7em {
      &c_k &&\ctrl{2}&\qw     &\ctrl{3}&\qw     &\qw &&c_k     &&& \\
      &i_k &&\targ   &\ctrl{1}&\qw     &\ctrl{1}&\qw &&i_k     &&& \\
      &t_k &&\targ   &\ctrl{1}&\qw     &\targ   &\qw &&&(t+i)_k && \\
      &    &&        &        &\targ   &\qw     &\qw &&c_{k+1} &&& \\
    }
  }
  \caption{
	  Temporary addition building block with a T-cost of 4 to compute and a T-cost of 0 to uncompute (not shown).
  }
  \label{fig:temporary-full-adder-block}
\end{figure}

We also note how to transform our adder into a controlled adder by controlling one of the operations.
The cost of this controlled adder is $8n + O(1)$; a significant improvement over a recently proposed construction with a T-cost of $21n + O(1)$.


\begin{figure}
  \resizebox{\linewidth}{!}{
    \Qcircuit @R=1.5em @C=0.7em {
      &\text{control} &&&&\qw     &\qw     &\qw     &\qw &\qw &\qw    &\qw &\qw &\qw &\qw    &\qw &\qw &\qw &\qw    &\qw &\qw &\qw     &\qw     &\ctrl{2}&\qw     &\qw &\\
      &&c_k            &&&\ctrl{2}&\qw     &\ctrl{3}&\qw &\qw &\qw    &\qw &\qw &\qw &\qw    &\qw &\qw &\qw &\qw    &\qw &\qw &\ctrl{3}&\qw     &\qw     &\ctrl{2}&\qw &\\
      &&i_k            &&&\targ   &\ctrl{1}&\qw     &\qw &\qw &\qw    &\qw &\qw &\qw &\qw    &\qw &\qw &\qw &\qw    &\qw &\qw &\qw     &\ctrl{1}&\ctrl{1}&\targ   &\qw &\\
      &&t_k            &&&\targ   &\ctrl{1}&\qw     &\qw &\qw &\qw    &\qw &\qw &\qw &\qw    &\qw &\qw &\qw &\qw    &\qw &\qw &\qw     &\ctrl{1}&\targ   &\targ   &\qw &\\
      &               &&&&        &        &\targ   &\qw &    &c_{k+1}&    &    &    &\ldots &    &    &    &c_{k+1}&    &    &\targ   &\qw     &        &        &    &\\
    }
  }
  \caption{
	Controlled full adder block with a T-cost of 8.
  }
  \label{fig:controlled-full-adder-block}
\end{figure}




\begin{figure}
  \centering
  \resizebox{\linewidth}{!}{
    \Qcircuit @R=1.5em @C=0.7em {
      &\text{control} &&&&\qw      &\qw     &\qw     &\qw     &\qw &\qw     &\qw     &\qw     &\qw &\qw     &\qw     &\qw     &\qw     &\ctrl{13}&\qw     &\qw     &\qw     &\ctrl{10}&\qw     &\qw     &\ctrl{7}&\qw     &\qw     &\qw     &\ctrl{4}&\qw     &\qw     &\ctrl{1}&\qw &&&\text{control}               &&&&&&\\
      &&i_0            &&&\ctrl{1} &\qw     &\qw     &\qw     &\qw &\qw     &\qw     &\qw     &\qw &\qw     &\qw     &\qw     &\qw     &\qw      &\qw     &\qw     &\qw     &\qw      &\qw     &\qw     &\qw     &\qw     &\qw     &\qw     &\qw     &\qw     &\ctrl{1}&\ctrl{1}&\qw &i_0                          &&&&&&&&\\
      &&t_0            &&&\ctrl{1} &\qw     &\qw     &\qw     &\qw &\qw     &\qw     &\qw     &\qw &\qw     &\qw     &\qw     &\qw     &\qw      &\qw     &\qw     &\qw     &\qw      &\qw     &\qw     &\qw     &\qw     &\qw     &\qw     &\qw     &\qw     &\ctrl{1}&\targ   &\qw &&&&&&(t+i \cdot \text{control})_0 &&&\\
      &&               &&&         &\ctrl{2}&\qw     &\ctrl{3}&\qw &\qw     &\qw     &\qw     &\qw &\qw     &\qw     &\qw     &\qw     &\qw      &\qw     &\qw     &\qw     &\qw      &\qw     &\qw     &\qw     &\qw     &\ctrl{3}&\qw     &\qw     &\ctrl{2}&\qw     &        &    &&&                             &&&&&&\\
      &&i_1            &&&\qw      &\targ   &\ctrl{1}&\qw     &\qw &\qw     &\qw     &\qw     &\qw &\qw     &\qw     &\qw     &\qw     &\qw      &\qw     &\qw     &\qw     &\qw      &\qw     &\qw     &\qw     &\qw     &\qw     &\ctrl{1}&\ctrl{1}&\targ   &\qw     &\qw     &\qw &i_1                          &&&&&&&&\\
      &&t_1            &&&\qw      &\targ   &\ctrl{1}&\qw     &\qw &\qw     &\qw     &\qw     &\qw &\qw     &\qw     &\qw     &\qw     &\qw      &\qw     &\qw     &\qw     &\qw      &\qw     &\qw     &\qw     &\qw     &\qw     &\ctrl{1}&\targ   &\targ   &\qw     &\qw     &\qw &&&&&&(t+i \cdot \text{control})_1 &&&\\
      &&               &&&         &        &        &\targ   &\qw &\ctrl{2}&\qw     &\ctrl{3}&\qw &\qw     &\qw     &\qw     &\qw     &\qw      &\qw     &\qw     &\qw     &\qw      &\ctrl{3}&\qw     &\qw     &\ctrl{2}&\targ   &\qw     &        &        &        &        &    &&&                             &&&&&&\\
      &&i_2            &&&\qw      &\qw     &\qw     &\qw     &\qw &\targ   &\ctrl{1}&\qw     &\qw &\qw     &\qw     &\qw     &\qw     &\qw      &\qw     &\qw     &\qw     &\qw      &\qw     &\ctrl{1}&\ctrl{1}&\targ   &\qw     &\qw     &\qw     &\qw     &\qw     &\qw     &\qw &i_2                          &&&&&&&&\\
      &&t_2            &&&\qw      &\qw     &\qw     &\qw     &\qw &\targ   &\ctrl{1}&\qw     &\qw &\qw     &\qw     &\qw     &\qw     &\qw      &\qw     &\qw     &\qw     &\qw      &\qw     &\ctrl{1}&\targ   &\targ   &\qw     &\qw     &\qw     &\qw     &\qw     &\qw     &\qw &&&&&&(t+i \cdot \text{control})_2 &&&\\
      &&               &&&         &        &        &        &    &        &        &\targ   &\qw &\ctrl{2}&\qw     &\ctrl{3}&\qw     &\qw      &\qw     &\ctrl{3}&\qw     &\qw      &\ctrl{2}&\qw     &        &        &        &        &        &        &        &        &    &&&                             &&&&&&\\
      &&i_3            &&&\qw      &\qw     &\qw     &\qw     &\qw &\qw     &\qw     &\qw     &\qw &\targ   &\ctrl{1}&\qw     &\qw     &\qw      &\qw     &\qw     &\ctrl{1}&\ctrl{1} &\targ   &\qw     &\qw     &\qw     &\qw     &\qw     &\qw     &\qw     &\qw     &\qw     &\qw &i_3                          &&&&&&&&\\
      &&t_3            &&&\qw      &\qw     &\qw     &\qw     &\qw &\qw     &\qw     &\qw     &\qw &\targ   &\ctrl{1}&\qw     &\qw     &\qw      &\qw     &\qw     &\ctrl{1}&\targ    &\targ   &\qw     &\qw     &\qw     &\qw     &\qw     &\qw     &\qw     &\qw     &\qw     &\qw &&&&&&(t+i \cdot \text{control})_3 &&&\\
      &&               &&&         &        &        &        &    &        &        &        &    &        &        &\targ   &\ctrl{1}&\qw      &\ctrl{1}&\targ   &\qw     &         &        &        &        &        &        &        &        &        &        &        &    &&&                             &&&&&&\\
      &&i_4            &&&\qw      &\qw     &\qw     &\qw     &\qw &\qw     &\qw     &\qw     &\qw &\qw     &\qw     &\qw     &\targ   &\ctrl{1} &\targ   &\qw     &\qw     &\qw      &\qw     &\qw     &\qw     &\qw     &\qw     &\qw     &\qw     &\qw     &\qw     &\qw     &\qw &i_4                          &&&&&&&&\\
      &&t_4            &&&\qw      &\qw     &\qw     &\qw     &\qw &\qw     &\qw     &\qw     &\qw &\qw     &\qw     &\qw     &\qw     &\targ    &\qw     &\qw     &\qw     &\qw      &\qw     &\qw     &\qw     &\qw     &\qw     &\qw     &\qw     &\qw     &\qw     &\qw     &\qw &&&&&&(t+i \cdot \text{control})_4 &&&\\
    }
  }
  \caption{
	5-bit controlled adder with a T-cost of 36.
	General $n$-bit controlled adder has T-cost of $8n - 4$.
  }
  \label{fig:ancilla-addition}
\end{figure}

Other examples of operations which benefit from this optimization include:

- Integer comparisons.

- Integer multiplication.

- Incrementing and counting.

- Modular arithmetic.

- Operations with a target qubit indexed by a binary qubit register.

- Expanding a binary register into a unary register.

- Phasing a register by a computable function $f$ of its value (i.e. applying the operation $U_f = \exp\left( i \sum_k f(k) |k\rangle \langle k| \right)$).

- Temporary permutations.

- Oracles in Grover's algorithm.

- etc


\section{Conclusion}
\label{sec:conclusion}

It was commonly believed that the optimal number of T gates to perform addition was 8, or 7 \cite{AustinDiscussionsAndEmails2017}.
We showed that the number of T gates is actually 4.
We are not aware of any method for improving the T cost further, but we do prove the temporary AND gatea lower bound of 2 In \autoref{fig:lower-bound} we show a lower bound of 2


\section{Acknowledgements}

Austin [reference finding, idea bouncing, review].


\bibliographystyle{plain}
\bibliography{citations}

https://arxiv.org/pdf/quant-ph/9503016.pdf   ->   the paired Toffoli construction that uses 4T

https://arxiv.org/pdf/cond-mat/9409111.pdf   -> shows CCNOT can be done with three 2-qubit interactions, ignores single-qubit???


https://arxiv.org/pdf/1308.4134.pdf -> proof of 7T gate minimum with no ancilla

https://arxiv.org/pdf/1212.5069.pdf  ->   does a single Toffoli with four T gates by turning one of the controls classical

https://arxiv.org/pdf/1210.0974.pdf  ->   a phase-neglected 4T gate construction that only uses a single layer of T gates


Quantum Computation and Quantum Information -> shows standard decomposition into 7T gates. But do they say it's optimal?

https://arxiv.org/pdf/1206.0758.pdf -> just an example of the 7T construction;;; probablynot relevant




>>>>>>>>>>>>>>>>>>>>>>> https://arxiv.org/pdf/1206.0758.pdf          paper that says addition costs 8N T


https://arxiv.org/pdf/1706.05113.pdf   has a ctrl-add with 21N + O(1). This technique+commutators achieves 8N.


https://pdfs.semanticscholar.org/8b84/d1bc2928922937a16205f9d8c925ff76689b.pdf     has a pretty good ripple carry


\end{document}
