\documentclass[twocolumn,longbibliography]{quantumarticle-customized}
\usepackage{amsmath}
\usepackage{graphicx}
\usepackage[pdfpagelabels,pdftex,bookmarks,breaklinks]{hyperref}
\usepackage{tikz}
\usepackage[all]{hypcap}
\input{Qcircuit}
\hypersetup{colorlinks,citecolor=blue,urlcolor=blue,linkcolor=blue}

\title{Saving $T$ gates by reusing $|A\rangle$ states}
\author{Craig Gidney}
\affiliation{Google, Santa Barbara, CA 93117, USA}
\email{craiggidney@google.com}

\def\sectionautorefname{Section}

\begin{document}
\maketitle

\begin{abstract}
A temporary Toffoli gate (sometimes called ``phase-congruent" or ``matched") performs the same operation as a normal Toffoli gate, but may incorrectly phase or entangle the involved qubits until a later Toffoli gate fixes these errors while uncomputing the permutation.
We improve the number of T gates needed to perform temporary Toffoli gates from 4 to 3 via non-destructive use of an ancilla in the state $|A\rangle = \frac{1}{\sqrt{2}} |0\rangle + \frac{1+i}{2} |1\rangle$.

Many quantum circuits use temporary Toffoli gates, and $T$ gates dominate the cost of error corrected quantum computing, so this optimization is significant and widely applicable.
For example, we reduce the T-cost of $n$-bit addition from $8n + O(1)$ to $6n + O(1)$ which in turn reduces the overall cost of Shor's factoring algorithm by nearly 25\%.
\end{abstract}

\section{Introduction} \label{sec:introduction}

When no ancillae are available, a Toffoli gate requires seven T gates to perform [[[cite]]].
If ancillae are available, that cost can be reduced to four [[[[cite]]] by doing a measurement teleportation thing.
When a Toffoli gate is applied temporarily, in a fashion where phase errors are acceptable until the Toffoli is uncomputed, a simpler inline trick suffices for achieving 4 T gates per Toffoli.

An operation $\tilde{U}$ is a temporary variant of $U$ if it meets the following criteria:

$$\forall b \in \{0, 1\}^{\otimes n}, \; \tilde{U} \cdot \left( |b\rangle \otimes | \psi_0 \rangle \right) = \left( U |b \rangle \right) \otimes | \psi_{1,b} \rangle$$

where $n$ is the number of qubits that $U$ acts on, $b$ is a computational basis state (a bitstring), $\psi_0$ is an input ancilla state, and $\psi_{1,c}$ is the various output ancilla states for each input $b$.
If we start with a computational basis state, apply the operation, and trace out the ancilla, then the output state will be correct up to a global phase (that may differ depending on the input state).
The phase and value of the traced-out ancilla may differ for different inputs, so $\tilde{U}$ will not perform the correct operation when its input is in superposition: it causes phase errors and produces entangled garbage.
Informally speaking, $\tilde{U}$ does the right thing classically but causes quantum-specific problems (i.e. incoherent phases) that must be fixed by a later operation.



See \autoref{fig:inline-toffoli}.

\newcommand{\qH}{\gate{H}}
\newcommand{\qT}{\gate{T}}
\newcommand{\qTi}{\gate{T^\dagger}}
\newcommand{\qS}{\gate{S}}
\newcommand{\qSi}{\gate{S^\dagger}}
\newcommand{\qA}{\lstick{|A\rangle}}
\newcommand{\qO}{\lstick{|0\rangle}}

\begin{figure}
  \resizebox{\linewidth}{!}{
    \Qcircuit @R=1.5em @C=0.7em {
      &\ctrl{1} &\qw & &   & & &\qw &\qT &\qw      &\qw  &\ctrl{2} &\qw &\qw      &\qw  &\ctrl{2} &\qw  &\ctrl{1} &\qw \\
      &\ctrl{1} &\qw & & = & & &\qw &\qT &\ctrl{1} &\qw  &\qw      &\qw &\ctrl{1} &\qw  &\targ    &\qTi &\targ    &\qw \\
      &\targ    &\qw & &   & & &\qH &\qT &\targ    &\qTi &\targ    &\qT &\targ    &\qTi &\targ    &\qH  &\qw      &\qw \\
    }
  }
  \caption{
	Simple Toffoli construction with seven T gates.
	Doesn't require any ancillae or a later paired Toffoli.
  }
  \label{fig:inline-toffoli}
\end{figure}

\begin{figure}
  \resizebox{\linewidth}{!}{
    \Qcircuit @R=1.5em @C=0.7em {
      &\ctrl{1} &\qw & &       & & &\ctrl{1} &\ctrl{1}     &\qw & &   & & &\qw &\qw &\qw      &\qw  &\ctrl{2} &\qw &\qw      &\qw  &\qw &\qw \\
      &\ctrl{1} &\qw & & \cong & & &\ctrl{1} &\qSi         &\qw & & = & & &\qw &\qw &\ctrl{1} &\qw  &\qw      &\qw &\ctrl{1} &\qw  &\qw &\qw \\
      &\targ    &\qw & &       & & &\targ    &\gate{Z}\qwx &\qw & &   & & &\qH &\qT &\targ    &\qTi &\targ    &\qT &\targ    &\qTi &\qH &\qw \\
    }
  }
  \caption{
	Existing temporary Toffoli construction with 4 T gates.
  }
  \label{fig:inline-temporary-toffoli}
\end{figure}

\begin{figure}
  \resizebox{\linewidth}{!}{
    \Qcircuit @R=1.5em @C=0.7em {
      &\ctrl{1} &\qw & &   & & &     &\qw &\qw &\qw      &\qw  &\ctrl{3} &\qw &\qw      &\qw  &\qw &\qw       &\qw  &\qw &\qw    &\ctrl{1}   &\qw \\
      &\ctrl{1} &\qw & & = & & &     &\qw &\qw &\ctrl{2} &\qw  &\qw      &\qw &\ctrl{2} &\qw  &\qw &\qw       &\qw  &\qw &\qw    &\gate{Z}   &\qw \\
      &\targ    &\qw & &   & & &     &\qw &\qw &\qw      &\qw  &\qw      &\qw &\qw      &\qw  &\qw &\targ     &\qw  &\qw &\qw    &\qw\cwx    &\qw \\
      &         &    & &   & & & \qO &\qH &\qT &\targ    &\qTi &\targ    &\qT &\targ    &\qTi &\qH &\ctrl{-1} &\qSi &\qH &\meter &\cw\cwx    & \\
    }
  }
  \caption{
	Toffoli construction with 4 T gates based on measurement teleportation.
  }
  \label{fig:ancilla-toffoli}
\end{figure}

\begin{figure}
  \resizebox{\linewidth}{!}{
    \Qcircuit @R=1.5em @C=0.7em {
      &    &\ctrl{1} &\qw & &       &&&     &\ctrl{1} &\ctrl{1} &\qw  &&&   & & & \lstick{x} &\qw      &\qw  &\ctrl{2} &\qw &\qw      &\qw  &\qw &\qw &\rstick{x} \\
      &    &\ctrl{1} &\qw & & \cong &&&     &\ctrl{1} &\qS      &\qw  &&& = & & & \lstick{y} &\ctrl{1} &\qw  &\qw      &\qw &\ctrl{1} &\qw  &\qw &\qw &\rstick{y} \\
      &\qO &\targ    &\qw & &       &&& \qO &\targ    &\qw      &\qw  &&&   & & & \qA        &\targ    &\qTi &\targ    &\qT &\targ    &\qTi &\qH &\qw &\rstick{x \land y} \\
    }
  }
  \caption{
	Control merging with three T gates.
	The $|A\rangle$ state is recovered when uncomputing the merged control.
  }
  \label{fig:ancilla-temporary-toffoli}
\end{figure}

\begin{figure}
  \resizebox{\linewidth}{!}{
    \Qcircuit @R=1.5em @C=0.7em {
      &\ctrl{1} &\qw & &       &&&     &\ctrl{1} &\ctrl{1} &\qw  &&&   & & & \lstick{x} &\qw      &\qw  &\ctrl{3} &\qw &\qw      &\qw  &\qw &\qw       &\qw &\rstick{x} \\
      &\ctrl{1} &\qw & & \cong &&&     &\ctrl{2} &\qS      &\qw  &&& = & & & \lstick{y} &\ctrl{2} &\qw  &\qw      &\qw &\ctrl{2} &\qw  &\qw &\qw       &\qw &\rstick{y} \\
      &\targ    &\qw & &       &&&     &\targ    &\qw      &\qw  &&&   & & & \lstick{t} &\qw      &\qw  &\qw      &\qw &\qw      &\qw  &\qw &\targ     &\qw &\rstick{t \oplus (x \land y)} \\
      &         &    & &       &&& \qO &\targ    &\qw      &\qw  &&&   & & & \qA        &\targ    &\qTi &\targ    &\qT &\targ    &\qTi &\qH &\ctrl{-1} &\qw &\rstick{x \land y} \\
    }
  }
  \caption{
	Our temporary Toffoli construction, which uses an $|A\rangle$ ancilla and 3 T gates.
	The $|A\rangle$ state is recovered when uncomputing the temporary Toffoli.
  }
  \label{fig:ancilla-temporary-toffoli}
\end{figure}

\begin{figure}
  \centering
  \makebox[\linewidth]{
  }
  \caption{
	$n$-bit inline addition with T-cost of $8n + O(1)$ based on [[[cite]]].
  }
  \label{fig:inline-addition}
\end{figure}

\begin{figure}
  \centering
  \makebox[\linewidth]{
  }
  \caption{
	$n$-bit addition with T-cost of $6n + O(1)$.
  }
  \label{fig:ancilla-addition}
\end{figure}

\begin{figure}
  \resizebox{\linewidth}{!}{
    \Qcircuit @R=1.5em @C=0.7em {
      &    &\ctrl{2}          &\qw               &\qw &&     &&&     &\ctrl{2}          &\qw               &\qw &\ctrl{1} &\qw      &\qw &&& &&& \lstick{x} &\ctrl{2} &\qw      &\qw &\ctrl{3} &\qw  &\qw      &\qw &\qw      &\qw  &\qw      &\qw &\rstick{x}         \\
      &    &\qw               &\ctrl{1}          &\qw &&\cong&&&     &\qw               &\ctrl{1}          &\qw &\qS      &\ctrl{1} &\qw &&&=&&& \lstick{y} &\qw      &\qw      &\qw &\qw      &\qw  &\ctrl{2} &\qw &\qw      &\qw  &\ctrl{1} &\qw &\rstick{y}         \\
      &    &\multigate{1}{+1} &\multigate{1}{+1} &\qw &&     &&&     &\multigate{1}{+1} &\multigate{1}{+1} &\qw &\qS\qwx  &\qS      &\qw &&& &&& \lstick{z} &\targ    &\ctrl{1} &\qw &\qw      &\qw  &\qw      &\qw &\ctrl{1} &\qw  &\targ    &\qw &\rstick{(x+y+z)_0} \\
      &\qO &\ghost{+1}        &\ghost{+1}        &\qw &&     &&& \qO &\ghost{+1}        &\ghost{+1}        &\qw &\qw      &\qw      &\qw &&& &&& \qA        &\qw      &\targ    &\qT &\targ    &\qSi &\targ    &\qT &\targ    &\qTi &\qH      &\qw &\rstick{(x+y+z)_1} \\
    }
  }
  \caption{
	Temporary full adder.
  }
  \label{fig:temporary-ancilla-addition}
\end{figure}

Mention binary to unary as an example?


Show example doing an $n$-control NOT with $6n + O(1)$ or $3n + O(1)$ if temporary.

Mention that this trick works more generally. Anytime you have an ancilla that starts HT and ends Tdag H.


\section{Acknowledgements}

Austin.


\bibliographystyle{plain}
\bibliography{citations}

\cite{barenco1995}
https://arxiv.org/pdf/quant-ph/9503016.pdf   ->   the paired Toffoli construction that uses 4T

https://arxiv.org/pdf/cond-mat/9409111.pdf   -> shows CCNOT can be done with three 2-qubit interactions, ignores single-qubit???


https://arxiv.org/pdf/1308.4134.pdf -> proof of 7T gate minimum with no ancilla

https://arxiv.org/pdf/1212.5069.pdf  ->   does a single Toffoli with four T gates by turning one of the controls classical

https://arxiv.org/pdf/1210.0974.pdf  ->   a phase-neglected 4T gate construction that only uses a single layer of T gates


Quantum Computation and Quantum Information -> shows standard decomposition into 7T gates. But do they say it's optimal?

https://arxiv.org/pdf/1206.0758.pdf -> just an example of the 7T construction;;; probablynot relevant

\end{document}
