\documentclass[twocolumn,longbibliography]{quantumarticle-customized}
\usepackage{amsmath}
\usepackage{graphicx}
\usepackage[pdfpagelabels,pdftex,bookmarks,breaklinks]{hyperref}
\usepackage{tikz}
\usepackage[all]{hypcap}
\hypersetup{colorlinks,citecolor=blue,urlcolor=blue,linkcolor=blue}

\input{Qcircuit}
\newcommand{\qH}{\gate{H}}
\newcommand{\qT}{\gate{T}}
\newcommand{\qTi}{\gate{T^\dagger}}
\newcommand{\qS}{\gate{S}}
\newcommand{\qSi}{\gate{S^\dagger}}
\newcommand{\qA}{\lstick{|A\rangle}}
\newcommand{\qO}{\lstick{|0\rangle}}

\title{T gates, temporary Toffolis, and halving the cost of quantum computation}
\author{Craig Gidney}
\affiliation{Google, Santa Barbara, CA 93117, USA}
\email{craiggidney@google.com}

\def\sectionautorefname{Section}

\begin{document}
\maketitle

\begin{abstract}
We show how to compute the logical-and of two controls using four T gates, and uncompute it using no T gates.
This construction halves the average T-cost of temporary Toffoli gates.
A temporary Toffoli gate is one which will be uncomputed later (typically by a second Toffoli gate).

We use our temporary logical-and construction to produce quantum circuits with T gate counts lower than previously reported.
T gates dominate the cost of surface-code-based error corrected quantum computation.
Temporary Toffolis are used in many common quantum circuits, such as addition.
As such, the optimization we present in this paper are widely applicable and represent significant savings in the projected costs of quantum computation.
For example, by applying previously known optimizations and the ones presented in this paper, we improve on the projected T-cost of Shor's algorithm from [[[[cite]]] by 70\% and the projected minimum time down by 66%.

We demonstrate an $n$-bit adder with T-cost $4n + O(1)$, a controlled $n$-bit adder with T-cost $8n + O(1)$, an $n$-controlled NOT with T-cost $4n + O(1)$, and a temporary pop-count with T-cost $6n + O(1)$.

We discuss investing $|A\rangle$ states, a fallback optimization technique that may reduce the number of T gates needed to compute quantum circuits not amenable to temporary ANDs.
\end{abstract}


\section{Introduction}
\label{sec:introduction}

A common construction technique used by quantum circuits is {\em uncomputation}.
For example, suppose we wish to phase the amplitude of each computational basis state $|k\rangle$ of some register $R_{\text{in}}$ by a number of radians computed by some function $f(k)$.
A naive method to perform this task is to 1) construct a classical reversible circuit $C$ for approximating $f$, 2) decompose the gates used by $C$ into a quantum gate set so that the circuit can be run on a quantum computer, 3) prepare a fresh register $R_\text{out}$ and use the circuit $C$ on it and $R_\text{in}$ to produce a binary fixed-point 2s-complement output stored in $R_{\text{out}}$, and then 4) phase by an amount proportional to $R_\text{out}$ by applying a $Z^{\theta 2^r}$ operation with appropriate scaling factor $\theta$ to each bit position $r$ within $R_\text{out}$.

The problem with this naive method is that it produces garbage.
The register $R_\text{out}$ ends up entangled with $R_\text{in}$.
If the garbage isn't cleaned up, this decoheres $R_\text{in}$; defeating the purpose of using a quantum computer.
The method we use to clean up the garbage is to uncompute $R_\text{out}$ by running $C$ in reverse.
This destroys the garbage, without undoing the phasing effects.

Because uncomputation is so common, it is important to understand how to perform it efficiently.
In this paper, our main focus will be to optimize the uncomputation of Toffoli gates.
Even more specifically, we focus on surface-code-based quantum computation, where the dominant cost of any circuit is the number of T gates.
As such, we focus on optimizing the number of T gates needed to compute and uncompute Toffoli gates and then on applying these optimizations elsewhere.

The paper is divided into sections.
In \autoref{sec:introduction}, we introduce.
In \autoref{sec:review}, we discuss previous work on optimizing T gate counts and note that they use 8 T gates to compute and uncompute a Toffoli gate.
In ????, we show how to improve the T-count from 8 to 6 by investing A states.
In ????, we show how erasing further reduces T-cost from 6 to 4.
In ????, we review our constructions and demonstrate how they significantly improve the T-cost of several basic arithmetic tasks.
Finally, ???? concludes and talks about the future.

On a surface-code-based quantum computer, T gates are XXX orders of magnitude more expensive than Clifford gates [[[cite]]].
Because of this disparity, optimizing error corrected quantum circuits largely consists of reducing their T-cost (the number of T gates used, or rather the number of $|A\rangle = \frac{1}{\sqrt{2}} |0\rangle + \frac{1+i}{2} |1\rangle$ states which must be distilled in order to perform those T gates).

\section{Previous Work}
\label{sec:review}

The textbook construction of a Toffoli gate uses seven T gates \cite{Nielsen2009} (see \autoref{fig:textbook-toffoli}).

\begin{figure}
  \resizebox{\linewidth}{!}{
    \Qcircuit @R=1.5em @C=0.7em {
      &\ctrl{1}&\qw & &   & & &\ctrl{1}&\qw  &\ctrl{1}&\qT &\qw     &\qw  &\ctrl{2}&\qw &\qw     &\qw  &\ctrl{2}&\qw &\qw \\
      &\ctrl{1}&\qw & & = & & &\targ   &\qTi &\targ   &\qT &\ctrl{1}&\qw  &\qw     &\qw &\ctrl{1}&\qw  &\qw     &\qw &\qw \\
      &\targ   &\qw & &   & & &\qw     &\qw  &\qH     &\qT &\targ   &\qTi &\targ   &\qT &\targ   &\qTi &\targ   &\qH &\qw \\
    }
  }
  \caption{
	Textbook Toffoli construction from \cite{Nielsen2009}.
	Uses eight Clifford gates and seven T gates.
  }
  \label{fig:textbook-toffoli}
\end{figure}

We can see why this construction works by considering which computation basis states are phased when applying a $Z$ rotation to a qubit temporarily storing a parity such as $x \oplus y$.
We start with an 8-dimensional vector $[0, 0, 0, 0, 0, 0, 0, 0]$ representing the amount of phase added into each basis state.
Applying a $Z^t$ rotation to a qubit will add $t$ into half of the vector entries (which four entries are increased depents on the value the qubit is temporarily storing).
W.l.o.g. we order the vector so that applying $Z^{t}$ to qubit \#1 adds $t \cdot [0, 1, 0, 1, 0, 1, 0, 1]$ into the phase-accumulation vector, applying $Z^{t}$ to qubit \#2 adds $t \cdot [0, 0, 1, 1, 0, 0, 1, 1]$, and applying $Z^{t}$ to qubit \#3 adds $t \cdot [0, 0, 0, 0, 1, 1, 1, 1]$.

If we temporarily CNOT qubit \#1 into qubit \#2 and phase qubit \#2, the corresponding vector is $[0, 1, 1, 0, 0, 1, 1, 0]$.
Given this ordering and the parity rule, we can prepare a table of vectors corresponding to phasing qubits storing various additions:

$$
\begin{aligned}
\text{target}             &\rightarrow & \text{phase vector} \\
\hline \\
q_1                       &\rightarrow &[0, 1, 0, 1, 0, 1, 0, 1] \\
q_2                       &\rightarrow &[0, 0, 1, 1, 0, 0, 1, 1] \\
q_3                       &\rightarrow &[0, 0, 0, 0, 1, 1, 1, 1] \\
q_1 \oplus q_2            &\rightarrow &[0, 1, 1, 0, 0, 1, 1, 0] \\
q_1 \oplus q_3            &\rightarrow &[0, 1, 0, 1, 1, 0, 1, 0] \\
q_2 \oplus q_3            &\rightarrow &[0, 0, 1, 1, 1, 1, 0, 0] \\
q_1 \oplus q_2 \oplus q_3 &\rightarrow &[0, 1, 1, 0, 1, 0, 0, 1] \\
\text{(global)}           &\rightarrow &[1, 1, 1, 1, 1, 1, 1, 1] \\
\end{aligned}
$$

Note that this set of vectors spans the full eight dimensional space.
Any desired phase function can be implemented by rotating parities.
(And in general this continues to be true for any number of qubits.)
For the specific case of the CCZ gate, we wish to perform $[0, 0, 0, 0, 0, 0, 0, 1]$.
We do so by adding the odd-number-of-qubit parities and subtracting the even-number-of-qubit parities, all times a quarter.
This is what the circuit construction is doing, but after framing the target qubit with Hadamards to switch the axis of interaction from Z to X so that a conditional NOT is performed instead of a conditional phase-flip.

Assuming we aren't permitted to move work to other qubits or to share work with other operations, the textbook Toffoli construction is optimal \cite{Gosset2014}.
Of course, in practice, we can do both.
For example, when several adjacent Toffoli gates share the same controls, all but one can be replaced by toggle-propagating CNOT operations (see \autoref{fig:shared-controls}).

\begin{figure}
  \resizebox{\linewidth}{!}{
    \Qcircuit @R=1.5em @C=0.7em {
      &\ctrl{1}&\ctrl{1}&\ctrl{1}&\ctrl{1}&\qw & &   & & &\qw     &\qw     &\qw     &\ctrl{1}&\qw     &\qw     &\qw     &\\
      &\ctrl{1}&\ctrl{2}&\ctrl{3}&\ctrl{4}&\qw & & = & & &\qw     &\qw     &\qw     &\ctrl{1}&\qw     &\qw     &\qw     &\\
      &\targ   &\qw     &\qw     &\qw     &\qw & &   & & &\ctrl{3}&\ctrl{2}&\ctrl{1}&\targ   &\ctrl{1}&\ctrl{2}&\ctrl{3}&\\
      &\qw     &\targ   &\qw     &\qw     &\qw & &   & & &\qw     &\qw     &\targ   &\qw     &\targ   &\qw     &\qw     &\\
      &\qw     &\qw     &\targ   &\qw     &\qw & &   & & &\qw     &\targ   &\qw     &\qw     &\qw     &\targ   &\qw     &\\
      &\qw     &\qw     &\qw     &\targ   &\qw & &   & & &\targ   &\qw     &\qw     &\qw     &\qw     &\qw     &\targ   &\\
    }
  }
  \caption{
	The T-cost of $N$ adjacent Toffolis sharing the same controls is $0 \cdot N + O(1)$.
	Each additional Toffoli can be replaced with CNOTs framing a root Toffoli.
  }
  \label{fig:shared-controls}
\end{figure}

It is not common for adjacent Toffolis to have the same controls, but it is common for a Toffoli to later be uncomputed by a second matching Toffoli (i.e. for the Toffoli to be {\em temporary}).
When this occurs, the three $T$ gates on the control qubits of the textbook construction can be omitted.
This introduces phase errors (see \autoref{fig:bad-phase-toffoli}), but the second Toffoli gate will uncompute those errors while uncomputing the state permutation (see \autoref{fig:cancelled-bad-phase-toffoli}) \cite{Barenco1995}.
This technique reduces the T-cost of paired Toffolis from 7 to 4.

\begin{figure}
  \resizebox{\linewidth}{!}{
    \Qcircuit @R=1.5em @C=0.7em {
      &\ctrl{1} &\qw & &       & & &\ctrl{1}     &\ctrl{1} &\qw & &   & & &\qw &\qw &\qw      &\qw  &\ctrl{2} &\qw &\qw      &\qw  &\qw &\qw \\
      &\ctrl{1} &\qw & & \cong & & &\qS          &\ctrl{1} &\qw & & = & & &\qw &\qw &\ctrl{1} &\qw  &\qw      &\qw &\ctrl{1} &\qw  &\qw &\qw \\
      &\targ    &\qw & &       & & &\gate{Z}\qwx &\targ    &\qw & &   & & &\qH &\qT &\targ    &\qTi &\targ    &\qT &\targ    &\qTi &\qH &\qw \\
    }
  }
  \caption{
	Starting with \autoref{fig:textbook-toffoli} then dropping T gates on the controls produces an operation with a T-cost of 4 that performs the correct permutation.
	However, the operation introduces phase errors.
  }
  \label{fig:bad-phase-toffoli}
\end{figure}

\begin{figure}
  \resizebox{\linewidth}{!}{
    \Qcircuit @R=1.5em @C=0.7em {
      &\ctrl{1}&\qw &\qw &\qw    &\qw &\qw &\ctrl{1}&\qw && &&&\ctrl{1}     &\ctrl{1}  &\qw &\qw &\qw    &\qw &\qw     &\ctrl{1}&\ctrl{1}     &\qw & \\
      &\ctrl{1}&\qw &\qw &\qw    &\qw &\qw &\ctrl{1}&\qw &&=&&&\qS          &\ctrl{1}  &\qw &\qw &\qw    &\qw &\qw     &\ctrl{1}&\qSi         &\qw & \\
      &\targ   &\qw &    &\ldots &    &    &\targ   &\qw && &&&\gate{Z}\qwx &\targ     &\qw &    &\ldots &    &        &\targ   &\gate{Z}\qwx &\qw & \\
    }
  }
  \caption{
	When two Toffolis form a compute/uncompute pair, they cancel phase errors in underlying constructions.
	Fixes the problem with the construction in \autoref{fig:bad-phase-toffoli}, achieving a per-Toffoli T-cost of 4 for paired Toffolis \cite{Barenco1995}.
  }
  \label{fig:cancelled-bad-phase-toffoli}
\end{figure}

In \cite{Jones2013}, Jones shows how to perform an unpaired Toffoli with a T-cost of 4 by using an ancilla and an erasure technique that replaces one of the quantum controls on a Toffoli with a classical control.
See \autoref{fig:jones-toffoli}.
Jones also shows how to rearrange all of the T gates into the same column, so that the Toffoli construction has a T-depth of 1 (see figure?).

\begin{figure}
  \resizebox{\linewidth}{!}{
    \Qcircuit @R=1.5em @C=0.7em {
      &\ctrl{1} &\qw & &   & & &     &\qw &\qw &\qw      &\qw  &\ctrl{3} &\qw &\qw      &\qw  &\qw &\qw  &\qw       &\qw &\qw    &\ctrl{1}       &\qw \\
      &\ctrl{1} &\qw & & = & & &     &\qw &\qw &\ctrl{2} &\qw  &\qw      &\qw &\ctrl{2} &\qw  &\qw &\qw  &\qw       &\qw &\qw    &\gate{Z}       &\qw \\
      &\targ    &\qw & &   & & &     &\qw &\qw &\qw      &\qw  &\qw      &\qw &\qw      &\qw  &\qw &\qw  &\targ     &\qw &\qw    &\qw\cwx        &\qw \\
      &         &    & &   & & & \qO &\qH &\qT &\targ    &\qTi &\targ    &\qT &\targ    &\qTi &\qH &\qSi &\ctrl{-1} &\qH &\meter &\cw\cwx\bullet & \\
    }
  }
  \caption{
	Toffoli construction with 4 T gates based on measurement teleportation.
  }
  \label{fig:jones-toffoli}
\end{figure}

We improve on Jones by noting that their construction separates cleanly into three pieces (merge, use, erase), using these pieces to reduce the average T-cost of paired Toffolis from 4 to 2, and then applying this improvement to common circuits.







\section{AND-ing and Erasing}
\label{sec:temporary-and}

When the target of a Toffoli gate is a qubit in the $|0\rangle$ state, and the controls are in a computation basis state, the target qubit ends up storing the logical-and of the two controls.
We continue to say that the target stores the logical-and of the two controls even when they are under superposition, with the understanding that we mean ``within each computation basis state making up the superposition of the overall system".

In the textbook construction of the Toffoli gate, the three T gates on the control wires are implementing a controlled-S.
A controlled-S phases states by $i$ where both the control and the target are ON.
If the target of the Toffoli gate starts in the $|0\rangle$ state, then the target ends up ON when both controls are satisfied.
Therefore the target is ON in exactly the state where a phase factor of $i$ should be applied, and the controlled-S on the controls can be replaced by an unconditional S gate on the target after the Toffoli is finished.

A minor optimization which we apply is to note that the first two operations on the target are transitioning from the $|0\rangle$ state to the $|A\rangle$ state.
Instead of consuming an $|A\rangle$ state to perform the $T$ gate (and potentially performing a fixup S gate), we simply feed the $|A\rangle$ state we would have consumed directly into the circuit.

This construction defines our logical AND operation, which we draw as a wire emerging from two controls as shown in the left hand side of \autoref{fig:logical-and} (which also shows our construction for the operation).

\begin{figure}
  \resizebox{\linewidth}{!}{
    \Qcircuit @R=1.5em @C=0.7em {
      &x &&\ctrl{1} &\qw & x  && &&          &&\qw      &\qw  &\ctrl{2} &\qw &\qw      &\qw  &\ctrl{2} &\qw &\qw &\qw &\\
      &y &&\ctrl{1} &\qw & y  &&=&&          &&\ctrl{1} &\qw  &\qw      &\qw &\ctrl{1} &\qw  &\qw      &\qw &\qw &\qw &\\
      &  &&         &\qw & xy && &&|A\rangle &&\targ    &\qTi &\targ    &\qT &\targ    &\qTi &\targ    &\qH &\qS &\qw &\\
    }
  }
  \caption{
	Control merging with T-cost of 4 (three T gates, one $|A\rangle$ state).
	The $|A\rangle$ state is fed directly into the circuit, instead of using an $H$ and $T$ gate on a $|0\rangle$ state (which would consume an $|A\rangle$ state anyways).
	The input $|A\rangle$ state can be recovered by using the un-merging construction in \autoref{fig:unmerge-recover-a}, reducing the net T cost of this circuit to 3.
	However, the un-merging construction in \autoref{fig:unmerge-free} is a better choice, despite not recovering the $|A\rangle$ state, because the total T-cost of merge+unmerge using \autoref{fig:unmerge-free} is 4+0=4 instead of 4+3-1=6.
  }
  \label{fig:logical-and}
\end{figure}

The logical AND construction consumes four $|A\rangle$ states, so its T-cost is 4.

To erase the ancilla produced by the logical AND, we use a technique we refer to as ``erasure".
We note that a Toffoli gate would obviously clear the value.
Then, since the target qubit can be discarded after clearing it, we are free to introduce a Hadamard gate and a measurement before discarding it.
We then hop the Hadamard over the Toffoli, transforming it into a CCZ operation.
The controls and targets of a CCZ are equivalent, so we rewrite the circuit into a CZC.
We then hop the Measurement over the CZC, turning a quantum control into a classical control.
These circuit moves turned the Toffoli gate into a CZ gate that we apply or not based on the outcome of the measreument
We ``Clifford-ized" the circuit by adding a Hadamard gate and a measurement.

We represent erasing a logical AND with a wire terminating into two controls, as shown in \autoref{fig:erase-and} which also shows the underlying construction.
The T-cost of erasure is 0.

\begin{figure}
  \resizebox{\linewidth}{!}{
    \Qcircuit @R=1.5em @C=0.7em {
      &x  &&\ctrl{1} &\qw &x && &&\qw &\qw    &\ctrl{1} &\qw \\
      &y  &&\ctrl{1} &\qw &y &&=&&\qw &\qw    &\gate{Z} &\qw \\
      &xy &&\qw      &    &  && &&\qH &\meter &\cw \cwx \bullet &    \\
    }
  }
  \caption{
	Control un-merging with T-cost of 0.
  }
  \label{fig:erase-and}
\end{figure}

The benefit of separating the logical-and-creating operation from the logical-and-erasing operation is that, as long as we have the ancilla storing the logical AND around, we can Cliffordize any Toffoli gate whose controls would be the two qubits we merged by replacing the Toffoli with a CNOT with the ancilla qubit as a control.
When the parts are simply placed next to each other, we end up with a construction exactly equivalent to the ones presented by Jones \cite{Jones2013}.
See \autoref{fig:merge-use-erase-toffoli}.

\begin{figure}
  \resizebox{\linewidth}{!}{
    \Qcircuit @R=1.5em @C=0.7em {
      &\ctrl{1} &\qw && &&\ctrl{1} &\qw      &\ctrl{1} &\qw && &&          &&\qw &\qw &\qw      &\qw  &\ctrl{2} &\qw &\qw      &\qw  &\ctrl{2} &\qw &\qw  &\qw      &\qw &\qw    &\ctrl{1}         &\qw \\
      &\ctrl{2} &\qw && &&\ctrl{1} &\qw      &\ctrl{1} &\qw && &&          &&\qw &\qw &\ctrl{1} &\qw  &\qw      &\qw &\ctrl{1} &\qw  &\qw      &\qw &\qw  &\qw      &\qw &\qw    &\gate{Z}         &\qw \\
      &         &    &&=&&         &\ctrl{1} &\qw      &    &&=&&|0\rangle &&\qH &\qT &\targ    &\qTi &\targ    &\qT &\targ    &\qTi &\targ    &\qH &\qSi &\ctrl{1} &\qH &\meter &\cw \cwx \bullet &    \\
      &\targ    &\qw && &&\qw      &\targ    &\qw      &\qw && &&          &&\qw &\qw &\qw      &\qw  &\qw      &\qw &\qw      &\qw  &\qw      &\qw &\qw  &\targ    &\qw &\qw    &\qw              &\qw \\
    }
  }
  \caption{
	Performing a Toffoli gate by computing, using, and erasing a logical-AND.
	Has a T-cost of 4, and is exactly equivalent to the construction from \cite{Jones2013}.
  }
  \label{fig:merge-use-erase-toffoli}
\end{figure}

In order to make savings, we need to keep the ancilla around for longer.
We need a situation like the one shown in \autoref{fig:paired-toffoli-to-logical-and}, where a single logical-AND computation replaces two Toffolis.

\begin{figure}
  \resizebox{\linewidth}{!}{
    \Qcircuit @R=1.5em @C=0.7em {
      &\ctrl{1} &\qw &\qw &\qw    &\qw &\qw &\ctrl{1} &\qw && &&\ctrl{1} &\qw      &\qw &\qw &\qw    &\qw &\qw &\qw      &\ctrl{1} &\qw & \\
      &\ctrl{2} &\qw &\qw &\qw    &\qw &\qw &\ctrl{2} &\qw && &&\ctrl{1} &\qw      &\qw &\qw &\qw    &\qw &\qw &\qw      &\ctrl{1} &\qw & \\
      &         &    &    &       &    &    &         &    &&=&&         &\ctrl{1} &\qw &\qw &\qw    &\qw &\qw &\ctrl{1} &\qw      &    & \\
      &\targ    &\qw &    &\ldots &    &    &\targ    &\qw && &&\qw      &\targ    &\qw &    &\ldots &    &    &\targ    &\qw      &\qw & \\
    }
  }
  \caption{
	When distant Toffolis share the same controls, a single logical-AND can Cliffordize both.
    The T-cost of computing and uncomputing the logical-AND is 4, so in effect this reduces the average T-cost of paired Toffolis to 2.
  }
  \label{fig:paired-toffoli-to-logical-and}
\end{figure}

Fortunately, this pattern of distant Toffolis sharing controls is very common in quantum circuits.
It occurs during uncomputation, and also in circuits that sweep back and forth across qubits like addition.


\section{T-optimized Circuit Constructions}
\label{sec:circuit-constructions}


\begin{figure}
  \resizebox{\linewidth}{!}{
    \Qcircuit @R=1.5em @C=0.7em {
      &x_k &&\targ    &\ctrl{1}&\qw      &\qw &\qw &\qw &\qw     &\qw &\qw &\qw &\qw    &\qw &\qw &\qw &\qw     &\qw &\qw &\qw      &\ctrl{1}&\targ    &\ctrl{1}&\qw &x_k     &&&&& \\
      &y_k &&\targ    &\ctrl{1}&\qw      &\qw &\qw &\qw &\qw     &\qw &\qw &\qw &\qw    &\qw &\qw &\qw &\qw     &\qw &\qw &\qw      &\ctrl{1}&\qw      &\targ   &\qw &&&(x+y)_k &&& \\
      &    &&         &        &\ctrl{1} &\qw &\qw &\qw &\qw     &\qw &\qw &\qw &\qw    &\qw &\qw &\qw &\qw     &\qw &\qw &\ctrl{1} &\qw     &         &        &    &&&        &&& \\
      &c_k &&\ctrl{-3}&\qw     &\targ    &\qw &\qw &    &c_{k+1} &    &    &    &\ldots &    &    &    &c_{k+1} &    &    &\targ    &\qw     &\ctrl{-3}&\qw     &\qw &c_k     &&&&& \\
    }
  }
  \caption{
	Inline full adder block with a T-cost of 4.
  }
  \label{fig:addition}
\end{figure}





\begin{figure}
  \centering
  \makebox[\linewidth]{
  }
  \caption{
	$n$-bit inline addition with T-cost of $8n + O(1)$ based on [[[cite]]].
  }
  \label{fig:inline-addition}
\end{figure}


\begin{figure}
  \centering
  \makebox[\linewidth]{
  }
  \caption{
	$n$-bit addition with T-cost of $6n + O(1)$.
  }
  \label{fig:ancilla-addition}
\end{figure}

\begin{figure}
  \resizebox{\linewidth}{!}{
    \Qcircuit @R=1.5em @C=0.7em {
      &    &\ctrl{2}          &\qw               &\qw &&     &&&     &\qS      &\qS      &\gate{Z}     &\ctrl{2}          &\qw               &\qw &&& &&& \lstick{x} &\ctrl{2} &\qw      &\qw &\ctrl{3} &\qw  &\qw      &\qw &\qw      &\qw  &\qw      &\qw &\rstick{x}         \\
      &    &\qw               &\ctrl{1}          &\qw &&\cong&&&     &\ctrl{-1}&\qS \qwx &\gate{Z}\qwx &\qw               &\ctrl{1}          &\qw &&&=&&& \lstick{y} &\qw      &\qw      &\qw &\qw      &\qw  &\ctrl{2} &\qw &\qw      &\qw  &\ctrl{1} &\qw &\rstick{y}         \\
      &    &\multigate{1}{+1} &\multigate{1}{+1} &\qw &&     &&&     &\qw      &\ctrl{-1}&\qw          &\multigate{1}{+1} &\multigate{1}{+1} &\qw &&& &&& \lstick{z} &\targ    &\ctrl{1} &\qw &\qw      &\qw  &\qw      &\qw &\ctrl{1} &\qw  &\targ    &\qw &\rstick{(x+y+z)_0} \\
      &\qO &\ghost{+1}        &\ghost{+1}        &\qw &&     &&& \qO &\qw      &\qw      &\ctrl{-2}    &\ghost{+1}        &\ghost{+1}        &\qw &&& &&& \qA        &\qw      &\targ    &\qT &\targ    &\qSi &\targ    &\qT &\targ    &\qTi &\qH      &\qw &\rstick{(x+y+z)_1} \\
    }
  }
  \caption{
	Temporary full adder.
  }
  \label{fig:temporary-ancilla-addition}
\end{figure}

Mention binary to unary as an example?


Show example doing an $n$-control NOT with $6n + O(1)$ or $3n + O(1)$ if temporary.

Mention that this trick works more generally. Anytime you have an ancilla that starts HT and ends Tdag H.

\section{Investing $|A\rangle$ states}
\label{sec:invest}

\begin{figure}
  \resizebox{\linewidth}{!}{
    \Qcircuit @R=1.5em @C=0.7em {
      &\ctrl{1} &\qw & &       &&&     &\ctrl{1} &\ctrl{1} &\qw  &&&   & & & \lstick{x} &\qw      &\qw  &\ctrl{3} &\qw &\qw      &\qw  &\qw &\qw       &\qw &\rstick{x} \\
      &\ctrl{1} &\qw & & \cong &&&     &\ctrl{2} &\qS      &\qw  &&& = & & & \lstick{y} &\ctrl{2} &\qw  &\qw      &\qw &\ctrl{2} &\qw  &\qw &\qw       &\qw &\rstick{y} \\
      &\targ    &\qw & &       &&&     &\targ    &\qw      &\qw  &&&   & & & \lstick{t} &\qw      &\qw  &\qw      &\qw &\qw      &\qw  &\qw &\targ     &\qw &\rstick{t \oplus (x \land y)} \\
      &         &    & &       &&& \qO &\targ    &\qw      &\qw  &&&   & & & \qA        &\targ    &\qTi &\targ    &\qT &\targ    &\qTi &\qH &\ctrl{-1} &\qw &\rstick{x \land y} \\
    }
  }
  \caption{
	Our temporary Toffoli construction, which uses an $|A\rangle$ ancilla and 3 T gates.
	The $|A\rangle$ state is recovered when uncomputing the temporary Toffoli.
  }
  \label{fig:ancilla-temporary-toffoli}
\end{figure}

\begin{figure}
  \resizebox{\linewidth}{!}{
    \Qcircuit @R=1.5em @C=0.7em {
      &x  &&\ctrl{1} &\qw &x && && \qw  &\qw &\ctrl{2} &\qw &\qw      &\qw  &\ctrl{2} &\qw &\qw      &\qw &           \\
      &y  &&\ctrl{1} &\qw &y &&=&& \qw  &\qw &\qw      &\qw &\ctrl{1} &\qw  &\qw      &\qw &\ctrl{1} &\qw &           \\
      &xy &&\qw      &    &  && && \qSi &\qH &\targ    &\qT &\targ    &\qTi &\targ    &\qT &\targ    &\qw &|A \rangle \\
    }
  }
  \caption{
	A non-optimal unmerging circuit that uses three $T$ gates, but recovers an $|A\rangle$ state.
	Avoids measurement, but has a net T-cost of two (which is worse than \autoref{fig:unmerge-free}'s T-cost of zero).
  }
  \label{fig:unmerge-recover-a}
\end{figure}


\section{Acknowledgements}

Austin.


\bibliographystyle{plain}
\bibliography{citations}

https://arxiv.org/pdf/quant-ph/9503016.pdf   ->   the paired Toffoli construction that uses 4T

https://arxiv.org/pdf/cond-mat/9409111.pdf   -> shows CCNOT can be done with three 2-qubit interactions, ignores single-qubit???


https://arxiv.org/pdf/1308.4134.pdf -> proof of 7T gate minimum with no ancilla

https://arxiv.org/pdf/1212.5069.pdf  ->   does a single Toffoli with four T gates by turning one of the controls classical

https://arxiv.org/pdf/1210.0974.pdf  ->   a phase-neglected 4T gate construction that only uses a single layer of T gates


Quantum Computation and Quantum Information -> shows standard decomposition into 7T gates. But do they say it's optimal?

https://arxiv.org/pdf/1206.0758.pdf -> just an example of the 7T construction;;; probablynot relevant



https://arxiv.org/pdf/1706.05113.pdf   has a ctrl-add with 21N + O(1). This technique+commutators achieves 8N.


https://pdfs.semanticscholar.org/8b84/d1bc2928922937a16205f9d8c925ff76689b.pdf     has a pretty good ripple carry

\end{document}
