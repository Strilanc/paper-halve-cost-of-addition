\documentclass[twocolumn,longbibliography]{quantumarticle-customized}
\usepackage{amsmath}
\usepackage{graphicx}
\usepackage[pdfpagelabels,pdftex,bookmarks,breaklinks]{hyperref}
\usepackage{tikz}
\usepackage[all]{hypcap}
\hypersetup{colorlinks,citecolor=blue,urlcolor=blue,linkcolor=blue}

%    Q-circuit version 2
%    Copyright (C) 2004  Steve Flammia & Bryan Eastin
%    Last modified on: 9/16/2011
%
%    This program is free software; you can redistribute it and/or modify
%    it under the terms of the GNU General Public License as published by
%    the Free Software Foundation; either version 2 of the License, or
%    (at your option) any later version.
%
%    This program is distributed in the hope that it will be useful,
%    but WITHOUT ANY WARRANTY; without even the implied warranty of
%    MERCHANTABILITY or FITNESS FOR A PARTICULAR PURPOSE.  See the
%    GNU General Public License for more details.
%
%    You should have received a copy of the GNU General Public License
%    along with this program; if not, write to the Free Software
%    Foundation, Inc., 59 Temple Place, Suite 330, Boston, MA  02111-1307  USA

% Thanks to the Xy-pic guys, Kristoffer H Rose, Ross Moore, and Daniel Müllner,
% for their help in making Qcircuit work with Xy-pic version 3.8.  
% Thanks also to Dave Clader, Andrew Childs, Rafael Possignolo, Tyson Williams,
% Sergio Boixo, Cris Moore, Jonas Anderson, and Stephan Mertens for helping us test 
% and/or develop the new version.

\usepackage[color]{xy}
\UseCrayolaColors
\xyoption{matrix}
\xyoption{frame}
\xyoption{arrow}
\xyoption{arc}

\usepackage{ifpdf}
\ifpdf
\else
\PackageWarningNoLine{Qcircuit}{Qcircuit is loading in Postscript mode.  The Xy-pic options ps and dvips will be loaded.  If you wish to use other Postscript drivers for Xy-pic, you must modify the code in Qcircuit.tex}
%    The following options load the drivers most commonly required to
%    get proper Postscript output from Xy-pic.  Should these fail to work,
%    try replacing the following two lines with some of the other options
%    given in the Xy-pic reference manual.
\xyoption{ps}
\xyoption{dvips}
\fi

% The following resets Xy-pic matrix alignment to the pre-3.8 default, as
% required by Qcircuit.
\entrymodifiers={!C\entrybox}

%\newcommand{\bra}[1]{{\left\langle{#1}\right\vert}}
%\newcommand{\ket}[1]{{\left\vert{#1}\right\rangle}}
    % Defines Dirac notation. %7/5/07 added extra braces so that the commands will work in subscripts.
\newcommand{\qw}[1][-1]{\ar @{-} [0,#1]}
\newcommand{\eqw}[1][-1]{\ar @{-} @[Red] [0,#1]}
    % Defines a wire that connects horizontally.  By default it connects to the object on the left of the current object.
    % WARNING: Wire commands must appear after the gate in any given entry.
\newcommand{\qwx}[1][-1]{\ar @{-} [#1,0]}
    % Defines a wire that connects vertically.  By default it connects to the object above the current object.
    % WARNING: Wire commands must appear after the gate in any given entry.
\newcommand{\cw}[1][-1]{\ar @{=} [0,#1]}
    % Defines a classical wire that connects horizontally.  By default it connects to the object on the left of the current object.
    % WARNING: Wire commands must appear after the gate in any given entry.
\newcommand{\cwx}[1][-1]{\ar @{=} [#1,0]}
    % Defines a classical wire that connects vertically.  By default it connects to the object above the current object.
    % WARNING: Wire commands must appear after the gate in any given entry.
\newcommand{\gate}[1]{*+<.6em>{#1} \POS ="i","i"+UR;"i"+UL **\dir{-};"i"+DL **\dir{-};"i"+DR **\dir{-};"i"+UR **\dir{-},"i" \qw}
\newcommand{\eboxgate} [1]{*+<.6em>{#1} \POS ="i","i"+UR;"i"+UL **[red]\dir{-};"i"+DL **[red]\dir{-};"i"+DR **[red]\dir{-};"i"+UR **[red]\dir{-},"i" \eqw}
\newcommand{\circgate}[1]{*+<0.6em>[o][F-]{#1} \eqw}
\newcommand{\ecircgate}[1]{*+<0.6em>[o][F-:red]{#1} \eqw}
\newcommand{\filtergt}[1]{\eboxgate{\scriptscriptstyle{#1}}}
\newcommand{\idealdec}{*+<1.2em>{\phantom{*}} \POS ="i","i"+UL;"i"+DL **[red]\dir{-};"i"+R **[red]\dir{-};"i"+UL **[red]\dir{-},"i" \eqw}

    % Boxes the argument, making a gate.
\newcommand{\meter}{*=<1.8em,1.4em>{\xy ="j","j"-<.778em,.322em>;{"j"+<.778em,-.322em> \ellipse ur,_{}},"j"-<0em,.4em>;p+<.5em,.9em> **\dir{-},"j"+<2.2em,2.2em>*{},"j"-<2.2em,2.2em>*{} \endxy} \POS ="i","i"+UR;"i"+UL **\dir{-};"i"+DL **\dir{-};"i"+DR **\dir{-};"i"+UR **\dir{-},"i" \qw}
    % Inserts a measurement meter.
    % In case you're wondering, the constants .778em and .322em specify
    % one quarter of a circle with radius 1.1em.
    % The points added at + and - <2.2em,2.2em> are there to strech the
    % canvas, ensuring that the size is unaffected by erratic spacing issues
    % with the arc.
\newcommand{\measure}[1]{*+[F-:<.9em>]{#1} \qw}
    % Inserts a measurement bubble with user defined text.
\newcommand{\measuretab}[1]{*{\xy*+<.6em>{#1}="e";"e"+UL;"e"+UR **\dir{-};"e"+DR **\dir{-};"e"+DL **\dir{-};"e"+LC-<.5em,0em> **\dir{-};"e"+UL **\dir{-} \endxy} \qw}
    % Inserts a measurement tab with user defined text.
\newcommand{\measureD}[1]{*{\xy*+=<0em,.1em>{#1}="e";"e"+UR+<0em,.25em>;"e"+UL+<-.5em,.25em> **\dir{-};"e"+DL+<-.5em,-.25em> **\dir{-};"e"+DR+<0em,-.25em> **\dir{-};{"e"+UR+<0em,.25em>\ellipse^{}};"e"+C:,+(0,1)*{} \endxy} \qw}
\newcommand{\emeasureD}[1]{*{\xy*+=<0em,.1em>{#1}="e";"e"+UR+<0em,.25em>;"e"+UL+<-.5em,.25em> **[red]\dir{-};"e"+DL+<-.5em,-.25em> **[red]\dir{-};"e"+DR+<0em,-.25em> **[red]\dir{-};{"e"+UR+<0em,.25em>\ellipse{}};"e"+C:,+(0,1)*{} \endxy} \qw}
    % Inserts a D-shaped measurement gate with user defined text.
\newcommand{\multimeasure}[2]{*+<1em,.9em>{\hphantom{#2}} \qw \POS[0,0].[#1,0];p !C *{#2},p \drop\frm<.9em>{-}}
    % Draws a multiple qubit measurement bubble starting at the current position and spanning #1 additional gates below.
    % #2 gives the label for the gate.
    % You must use an argument of the same width as #2 in \ghost for the wires to connect properly on the lower lines.
\newcommand{\multimeasureD}[2]{*+<1em,.9em>{\hphantom{#2}} \POS [0,0]="i",[0,0].[#1,0]="e",!C *{#2},"e"+UR-<.8em,0em>;"e"+UL **\dir{-};"e"+DL **\dir{-};"e"+DR+<-.8em,0em> **\dir{-};{"e"+DR+<0em,.8em>\ellipse^{}};"e"+UR+<0em,-.8em> **\dir{-};{"e"+UR-<.8em,0em>\ellipse^{}},"i" \qw}
    % Draws a multiple qubit D-shaped measurement gate starting at the current position and spanning #1 additional gates below.
    % #2 gives the label for the gate.
    % You must use an argument of the same width as #2 in \ghost for the wires to connect properly on the lower lines.
\newcommand{\control}{*!<0em,.025em>-=-<.2em>{\bullet}}
    % Inserts an unconnected control.
\newcommand{\controlo}{*+<.01em>{\xy -<.095em>*\xycircle<.19em>{} \endxy}}
    % Inserts a unconnected control-on-0.
\newcommand{\ctrl}[1]{\control \qwx[#1] \qw}
    % Inserts a control and connects it to the object #1 wires below.
\newcommand{\ctrlo}[1]{\controlo \qwx[#1] \qw}
    % Inserts a control-on-0 and connects it to the object #1 wires below.
\newcommand{\targ}{*+<.02em,.02em>{\xy ="i","i"-<.39em,0em>;"i"+<.39em,0em> **\dir{-}, "i"-<0em,.39em>;"i"+<0em,.39em> **\dir{-},"i"*\xycircle<.4em>{} \endxy} \qw}
    % Inserts a CNOT target.
\newcommand{\qswap}{*=<0em>{\times} \qw}
    % Inserts half a swap gate.
    % Must be connected to the other swap with \qwx.
\newcommand{\multigate}[2]{*+<1em,.9em>{\hphantom{#2}} \POS [0,0]="i",[0,0].[#1,0]="e",!C *{#2},"e"+UR;"e"+UL **\dir{-};"e"+DL **\dir{-};"e"+DR **\dir{-};"e"+UR **\dir{-},"i" \qw}
    % Draws a multiple qubit gate starting at the current position and spanning #1 additional gates below.
    % #2 gives the label for the gate.
    % You must use an argument of the same width as #2 in \ghost for the wires to connect properly on the lower lines.
\newcommand{\ghost}[1]{*+<1em,.9em>{\hphantom{#1}} \qw}
    % Leaves space for \multigate on wires other than the one on which \multigate appears.  Without this command wires will cross your gate.
    % #1 should match the second argument in the corresponding \multigate.
\newcommand{\push}[1]{*{#1}}
    % Inserts #1, overriding the default that causes entries to have zero size.  This command takes the place of a gate.
    % Like a gate, it must precede any wire commands.
    % \push is useful for forcing columns apart.
    % NOTE: It might be useful to know that a gate is about 1.3 times the height of its contents.  I.e. \gate{M} is 1.3em tall.
    % WARNING: \push must appear before any wire commands and may not appear in an entry with a gate or label.
\newcommand{\gategroup}[6]{\POS"#1,#2"."#3,#2"."#1,#4"."#3,#4"!C*+<#5>\frm{#6}}
    % Constructs a box or bracket enclosing the square block spanning rows #1-#3 and columns=#2-#4.
    % The block is given a margin #5/2, so #5 should be a valid length.
    % #6 can take the following arguments -- or . or _\} or ^\} or \{ or \} or _) or ^) or ( or ) where the first two options yield dashed and
    % dotted boxes respectively, and the last eight options yield bottom, top, left, and right braces of the curly or normal variety.  See the Xy-pic reference manual for more options.
    % \gategroup can appear at the end of any gate entry, but it's good form to pick either the last entry or one of the corner gates.
    % BUG: \gategroup uses the four corner gates to determine the size of the bounding box.  Other gates may stick out of that box.  See \prop.

\newcommand{\rstick}[1]{*!L!<-.5em,0em>=<0em>{#1}}
    % Centers the left side of #1 in the cell.  Intended for lining up wire labels.  Note that non-gates have default size zero.
\newcommand{\lstick}[1]{*!R!<.5em,0em>=<0em>{#1}}
    % Centers the right side of #1 in the cell.  Intended for lining up wire labels.  Note that non-gates have default size zero.
\newcommand{\ustick}[1]{*!D!<0em,-.5em>=<0em>{#1}}
    % Centers the bottom of #1 in the cell.  Intended for lining up wire labels.  Note that non-gates have default size zero.
\newcommand{\dstick}[1]{*!U!<0em,.5em>=<0em>{#1}}
    % Centers the top of #1 in the cell.  Intended for lining up wire labels.  Note that non-gates have default size zero.
\newcommand{\Qcircuit}{\xymatrix @*=<0em>}
    % Defines \Qcircuit as an \xymatrix with entries of default size 0em.
\newcommand{\link}[2]{\ar @{-} [#1,#2]}
    % Draws a wire or connecting line to the element #1 rows down and #2 columns forward.
\newcommand{\pureghost}[1]{*+<1em,.9em>{\hphantom{#1}}}
    % Same as \ghost except it omits the wire leading to the left. 
%%%%%%%%%%%%%%%%%%%%%%%%%%%%%%%%%%%%%%%%%%%%%%%%%%%%%%%%%%%%%%%%%%%%%%%%%%%%%%%%%%%%%%%%%%
\newcommand{\multiprepareC}[2]{*+<1em,.9em>{\hphantom{#2}}\save[0,0].[#1,0];p\save !C
  *{#2},p+RU+<0em,0em>;+LU+<+.8em,0em> **\dir{-}\restore\save +RD;+RU **\dir{-}\restore\save
  +RD;+LD+<.8em,0em> **\dir{-} \restore\save +LD+<0em,.8em>;+LU-<0em,.8em> **\dir{-} \restore \POS
  !UL*!UL{\cir<.9em>{u_r}};!DL*!DL{\cir<.9em>{l_u}}\restore}
   % Draws a multiple qubit reverse-D-shaped preparation gate starting at the current position and spanning #1 additional gates below.
   % #2 gives the label for the gate.
   % You must use an argument of the same width as #2 in \pureghost for the wires to connect properly on
% the lower lines.
\newcommand{\prepareC}[1]{*{\xy*+=+<.5em>{\vphantom{#1\rule{0em}{.1em}}}*\cir{l^r};p\save*!L{#1} \restore\save+UC;+UC+<.5em,0em>*!L{\hphantom{#1}}+R **\dir{-} \restore\save+DC;+DC+<.5em,0em>*!L{\hphantom{#1}}+R **\dir{-} \restore\POS+UC+<.5em,0em>*!L{\hphantom{#1}}+R;+DC+<.5em,0em>*!L{\hphantom{#1}}+R **\dir{-} \endxy}}
   % Inserts a reverse-D-shaped preparation gate with user defined text.
\newcommand{\poloFantasmaCn}[1]{{{}^{#1}_{\phantom{#1}}}}

\newcommand{\qH}{\gate{H}}
\newcommand{\qT}{\gate{T}}
\newcommand{\qTi}{\gate{T^\dagger}}
\newcommand{\qS}{\gate{S}}
\newcommand{\qSi}{\gate{S^\dagger}}
\newcommand{\qA}{\lstick{|A\rangle}}
\newcommand{\qO}{\lstick{|0\rangle}}

\title{T gates, temporary Toffolis, and halving the cost of quantum computation}
\author{Craig Gidney}
\affiliation{Google, Santa Barbara, CA 93117, USA}
\email{craiggidney@google.com}

\def\sectionautorefname{Section}

\begin{document}
\maketitle

\begin{abstract}
A ``temporary Toffoli" is a doubly-controlled-not operation that will later be uncomputed by a second double-controlled-not.
We construct temporary Toffoli gates using four T gates; half as many as previously reported.
We demonstrate how to apply our construction (and other optimizations) to common operations and algorithms, and show significant improvement over previously reported T-costs.
We reduce the reported T-cost of addition by 50\% and the reported T-cost of Shor's algorithm by ?????\%.

T gates dominate the cost of surface-code-based quantum computation.
Temporary Toffolis are a basic building block that applies to many quantum circuits.
As such, a temporary Toffoli gate that uses fewer T gates represents a widely applicable and significant improvement to the projected costs of quantum computation.
\end{abstract}


\section{Introduction}
\label{sec:introduction}

The surface code is a quantum error correcting code that works on a 2D nearest-neighbour array of qubits and achieves a threshold error rate of approximately 1\% [[[[cite so much]]]].
This makes the surface code a likely component in the architecture of large error corrected quantum computers, because 2D arrays of qubits with nearest-neighbor connections are possible with many qubit technologies [[[[cite so much]]]] and other known error correcting codes have lower thresholds or require higher connectivity [[[[cite so much]]]].

One of the downsides of the surface code is that it has no cheap mechanism to apply non-Clifford operations such as $T$ gates.
Instead, $T$ gates are performed by distilling and consuming $|A\rangle = \frac{1}{\sqrt{2}} (|0\rangle + e^{i \pi/4} |1\rangle)$ states.
Consuming an $|A\rangle$ state to perform a T gate is simple (see \autoref{fig:t-equals-a}), but distilling $|A\rangle$ states has significant cost.

\begin{figure}
  \resizebox{\linewidth}{!}{
    \Qcircuit @R=0.7em @C=0.7em {
      &\gate{T}&\qw &&=&&&          &&\qw &\ctrl{1}&\qw    &\qS            &\qw &\\
      &        &    && &&&|A\rangle &&\qw &\targ   &\meter &\cw\cwx\bullet &    &\\
    }
  }
  \caption{
	Performing a T gate with Clifford operations by consuming an $|A\rangle$ state, where $|A\rangle = \frac{1}{\sqrt{2}} (|0\rangle + e^{i \pi/4} |1\rangle$.
  }
  \label{fig:t-equals-a}
\end{figure}

In [[[[cite]]]] the $|A\rangle$ state distillation is done by preparing an EPR pair $(P, Q)$, translating $P$ out of the surface code and into a different error correcting code that can perform $T$ gates, and using a $T$ gate within that code to rotate and measure $P$ along an axis correlated with the diagonal $X+Y$ axis of $Q$.
If this process succeeds without an error being detected, it steers \cite{Wiseman2007} $Q$ into the $|A\rangle$ state or else into the complementary state $Z|A\rangle$.
The $Z$ rotation is easily corrected, and the result of measuring $P$ indicates whether the correction is needed or not.

The resulting `foothold' $|A\rangle$ states have a non-negligible amount of noise, but can be used to perform approximate T gates powering additional rounds of distillation occurring entirely within the surface code.
After accounting for the cost of distilling a sufficiently accurate $|A\rangle$ state, the spacetime volume cost of performing a T gate is one to two orders of magnitude higher than performing operations native to the surface code (such as the CNOT gate) [[[[cite]]]].

Because T gates are so expensive for the surface code, and the surface code is a likely component of future quantum computers, it is important to consider and optimize the number of T gates used by quantum circuits.

In this paper, we focus on optimizing the number of T gates needed to perform Toffoli gates that will later be uncomputed by a second Toffoli gate (``temporary Toffoli gates").
Temporary Toffolis show up in many circuits, especially ones that involve computing and uncomputing classical functions, so these optimizations are widely applicable.

Our paper is divided into six sections.
In \autoref{sec:introduction}, we motivate the problem and insert self-referential descriptions.
\autoref{sec:review} discusses how previous work on optimizing the T gate counts of Toffolis improved the cost of temporary Toffolis from 14 to 8.
In \autoref{sec:invest}, we show how to improve the cost of temporary Toffolis from 8 to 6 by investing and later recovering an $|A\rangle$ state instead of consuming it to perform T gates.
\autoref{sec:temporary-and} explains an ancilla erasing trick that improves the cost of temporary Toffolis further, from 6 to 4.
This leads to \autoref{sec:circuit-constructions}, where we demonstrate how to use our constructions to improve the T-cost of several basic arithmetic tasks (e.g. halving the T-cost of addition).
Finally, \autoref{sec:conclusion} concludes and discusses future work.


\section{Previous Work}
\label{sec:review}

The textbook construction of a Toffoli gate uses seven T gates \cite{Nielsen2009} (see \autoref{fig:textbook-toffoli}).
Assuming we aren't permitted to involve other qubits or to share work with other operations, this construction is optimal \cite{Gosset2014}.
Of course, in practice, we can do both.
For example, when several adjacent Toffoli gates share the same controls, all but one can be replaced by CNOT operations (see \autoref{fig:shared-controls}).

\begin{figure}
  \resizebox{\linewidth}{!}{
    \Qcircuit @R=1.5em @C=0.7em {
      &\ctrl{1}&\qw & &   & & &\ctrl{1}&\qw  &\ctrl{1}&\qT &\qw     &\qw  &\ctrl{2}&\qw &\qw     &\qw  &\ctrl{2}&\qw &\qw \\
      &\ctrl{1}&\qw & & = & & &\targ   &\qTi &\targ   &\qT &\ctrl{1}&\qw  &\qw     &\qw &\ctrl{1}&\qw  &\qw     &\qw &\qw \\
      &\targ   &\qw & &   & & &\qw     &\qw  &\qH     &\qT &\targ   &\qTi &\targ   &\qT &\targ   &\qTi &\targ   &\qH &\qw \\
    }
  }
  \caption{
	Textbook Toffoli construction from \cite{Nielsen2009}.
	Uses eight Clifford gates and seven T gates.
  }
  \label{fig:textbook-toffoli}
\end{figure}

\begin{figure}
  \resizebox{\linewidth}{!}{
    \Qcircuit @R=1.5em @C=0.7em {
      &\ctrl{1}&\ctrl{1}&\ctrl{1}&\ctrl{1}&\qw && &&&\qw     &\qw     &\qw     &\ctrl{1}&\qw     &\qw     &\qw     &\qw &\\
      &\ctrl{1}&\ctrl{2}&\ctrl{3}&\ctrl{4}&\qw &&=&&&\qw     &\qw     &\qw     &\ctrl{1}&\qw     &\qw     &\qw     &\qw &\\
      &\targ   &\qw     &\qw     &\qw     &\qw && &&&\ctrl{3}&\ctrl{2}&\ctrl{1}&\targ   &\ctrl{1}&\ctrl{2}&\ctrl{3}&\qw &\\
      &\qw     &\targ   &\qw     &\qw     &\qw && &&&\qw     &\qw     &\targ   &\qw     &\targ   &\qw     &\qw     &\qw &\\
      &\qw     &\qw     &\targ   &\qw     &\qw && &&&\qw     &\targ   &\qw     &\qw     &\qw     &\targ   &\qw     &\qw &\\
      &\qw     &\qw     &\qw     &\targ   &\qw && &&&\targ   &\qw     &\qw     &\qw     &\qw     &\qw     &\targ   &\qw &\\
    }
  }
  \caption{
	The T-cost of $N$ adjacent Toffolis sharing the same controls is $0 \cdot N + O(1)$.
	The marginal T-cost is 0 because each additional Toffoli can be replaced with CNOTs framing a root Toffoli.
  }
  \label{fig:shared-controls}
\end{figure}

It is not common for adjacent Toffolis to have the same controls, but it is common for a Toffoli to later be uncomputed by a second matching Toffoli (i.e. for the Toffoli's effect to be temporary).
When this occurs, the three $T$ gates on the control qubits of the textbook construction can be omitted.
This introduces phase errors (see \autoref{fig:bad-phase-toffoli}), but the second Toffoli gate can uncompute those errors while uncomputing the state permutation \cite{Barenco1995} (see \autoref{fig:cancelled-bad-phase-toffoli}).

\begin{figure}
  \resizebox{\linewidth}{!}{
    \Qcircuit @R=1.5em @C=0.7em {
      &\ctrl{1} &\qw & &       & & &\ctrl{1}     &\ctrl{1} &\qw & &   & & &\qw &\qw &\qw      &\qw  &\ctrl{2} &\qw &\qw      &\qw  &\qw &\qw \\
      &\ctrl{1} &\qw & & \cong & & &\qS          &\ctrl{1} &\qw & & = & & &\qw &\qw &\ctrl{1} &\qw  &\qw      &\qw &\ctrl{1} &\qw  &\qw &\qw \\
      &\targ    &\qw & &       & & &\gate{Z}\qwx &\targ    &\qw & &   & & &\qH &\qT &\targ    &\qTi &\targ    &\qT &\targ    &\qTi &\qH &\qw \\
    }
  }
  \caption{
	Starting with \autoref{fig:textbook-toffoli} then dropping T gates on the controls produces an operation with a T-cost of 4 that performs the correct permutation.
	However, the operation introduces phase errors.
  }
  \label{fig:bad-phase-toffoli}
\end{figure}

\begin{figure}
  \resizebox{\linewidth}{!}{
    \Qcircuit @R=1.5em @C=0.7em {
      &\ctrl{1}&\qw &\qw &\qw    &\qw &\qw &\ctrl{1}&\qw && &&&\ctrl{1}     &\ctrl{1}  &\qw &\qw &\qw    &\qw &\qw     &\ctrl{1}&\ctrl{1}     &\qw & \\
      &\ctrl{1}&\qw &\qw &\qw    &\qw &\qw &\ctrl{1}&\qw &&=&&&\qS          &\ctrl{1}  &\qw &\qw &\qw    &\qw &\qw     &\ctrl{1}&\qSi         &\qw & \\
      &\targ   &\qw &    &\ldots &    &    &\targ   &\qw && &&&\gate{Z}\qwx &\targ     &\qw &    &\ldots &    &        &\targ   &\gate{Z}\qwx &\qw & \\
    }
  }
  \caption{
	When two Toffolis form a compute/uncompute pair, they can cancel each others' phase errors.
	Fixes the problem with the construction in \autoref{fig:bad-phase-toffoli}, and achieves a per-Toffoli T-cost of 4 for paired Toffolis \cite{Barenco1995}.
  }
  \label{fig:cancelled-bad-phase-toffoli}
\end{figure}

The paired-phase-correction trick improves the T-cost of temporary Toffolis from 14 (seven to compute, seven to uncompute) to 8 (four to compute, four to uncompute).

In \cite{Jones2013}, Jones shows how to perform an {\em unpaired} Toffoli with four T gates.
They use an ancilla qubit and an erasure technique that turns what would have been a quantum control into a classical control.
They also re-arrange the remaining T gates into a single column, creating a circuit with T-depth of 1.
We show this construction in \autoref{fig:jones-toffoli}.

\begin{figure}
  \resizebox{\linewidth}{!}{
    \Qcircuit @R=1.5em @C=0.7em {
      &\ctrl{1} &\qw & &   & & &     &\qw &\qw &\qw      &\qw  &\ctrl{3} &\qw &\qw      &\qw  &\qw &\qw  &\qw       &\qw &\qw    &\ctrl{1}       &\qw \\
      &\ctrl{1} &\qw & & = & & &     &\qw &\qw &\ctrl{2} &\qw  &\qw      &\qw &\ctrl{2} &\qw  &\qw &\qw  &\qw       &\qw &\qw    &\gate{Z}       &\qw \\
      &\targ    &\qw & &   & & &     &\qw &\qw &\qw      &\qw  &\qw      &\qw &\qw      &\qw  &\qw &\qw  &\targ     &\qw &\qw    &\qw\cwx        &\qw \\
      &         &    & &   & & & \qO &\qH &\qT &\targ    &\qTi &\targ    &\qT &\targ    &\qTi &\qH &\qSi &\ctrl{-1} &\qH &\meter &\cw\cwx\bullet & \\
    }
  }
  \caption{
	Toffoli construction with T-cost of 4 from \cite{Jones2013}.
  }
  \label{fig:jones-toffoli}
\end{figure}

By using Jones' construction twice (or the phase-correct-by-pairing construction), a temporary Toffoli can be computed with eight T gates.
Four T gates for the initial Toffoli, and four T gates to uncompute its effect later with another Toffoli.


\section{Investing $|A\rangle$ states}
\label{sec:invest}

The seed for this paper was an idea that accidentally improved the T-cost of a Toffoli from 7 to 6 in a surprising way.
Instead of performing a Toffoli directly onto the target qubit, we performed the Toffoli indirectly via an ancilla.
We started with a clean ancilla in the $|0\rangle$ state, applied a Toffoli to store the logical-and of the two controls in the ancilla, used the ancilla to control a CNOT onto the intended target, then uncomputed the ancilla.

Initially, this indirect-Toffoli construction seems pointless.
It has a T-cost of 8 even when using the phase-neglected construction, which is worse than the naive T-cost of 7.
However, as shown in \autoref{fig:indirect-toffoli}, it turns out that the last T gate in the circuit is being used to uncompute an $|A\rangle$ state on the ancilla.
Dropping this T gate not only reduces the T-cost from 8 to 7, it recovers an $|A\rangle$ state that can be used to power some other T gate.
This improves the net T-cost to 6.

\begin{figure}
  \resizebox{\linewidth}{!}{
    \Qcircuit @R=1.5em @C=0.7em {
      &\ctrl{1} &\qw && &&&    &\qw &\qw &\qw      &\qw  &\ctrl{3} &\qw &\qw      &\qw  &\qw &\qw       &\qw &\qw &\qw      &\qw  &\ctrl{3} &\qw &\qw      &\qw  &\qw &\qw &&\\
      &\ctrl{1} &\qw &&=&&&    &\qw &\qw &\ctrl{2} &\qw  &\qw      &\qw &\ctrl{2} &\qw  &\qw &\qw       &\qw &\qw &\ctrl{2} &\qw  &\qw      &\qw &\ctrl{2} &\qw  &\qw &\qw &&\\
      &\targ    &\qw && &&&    &\qw &\qw &\qw      &\qw  &\qw      &\qw &\qw      &\qw  &\qw &\targ     &\qw &\qw &\qw      &\qw  &\qw      &\qw &\qw      &\qw  &\qw &\qw &&\\
      &         &    && &&&\qO &\qH &\qT &\targ    &\qTi &\targ    &\qT &\targ    &\qTi &\qH &\ctrl{-1} &\qH &\qT &\targ    &\qTi &\targ    &\qT &\targ    &\qTi &\qH &\qw 
          \gategroup{4}{6}{4}{10}{.7em}{--} \gategroup{4}{26}{4}{30}{.7em}{--} &|0\rangle & \\
      &         &    && &&&    &    |A\rangle &  & &     &         &    &         &     &    &          &    &    &         &    &         &     &         &     & |A\rangle
    }
  }
  \caption{
	Performing a Toffoli indirectly, by applying a pair of Toffolis to an ancilla and using the intermediate value to control a CNOT onto the actual target, appears to have a T-cost of 8.
	However, the initial T gate is being used to compute an $|A\rangle$ state and the final T gate is being used to uncompute the $|A\rangle$ state.
    \\	
	Instead of spending $|A\rangle$ states to perform T gates to make and unmake $|A\rangle$ states, we pass in and later recover a single $|A\rangle$ state.
	This ``investment" reduces the net T-cost to 6.
  }
  \label{fig:indirect-toffoli}
\end{figure}

A Toffoli with T-cost of 6 isn't optimal, but the reduction from the textbook T-cost of 7 was done in an unexpected way.
There were likely other situations where this optimization would improve on the state of the art, and we soon found one: temporary Toffolis.
When a Toffoli is going to be uncomputed, it is not necessary to compute, uncompute, recompute, and reuncompute the ancilla qubit.
Instead, compute the ancilla for the first Toffoli, keep the ancilla around to make performing the second Toffoli easy, then uncompute the ancilla.
Separating the control-merging step from the control-unmerging step improves the T-cost of temporary Toffolis from 8 to 6.

[[[[Keeping ancilla around is counter-intuitive, because the ancilla represents entangled garbage that can prevent interference from occurring correctly.
It just so happens that it makes sense to do so for uncomputation tasks related to classical functions.]]]]


\begin{figure}
  \resizebox{\linewidth}{!}{
    \Qcircuit @R=1.5em @C=0.7em {
      &\ctrl{1}&\qw &\qw &\qw    &\qw &\qw &\ctrl{1}&\qw && &&&    &\qw     &\qw  &\ctrl{3}&\qw &\qw     &\qw  &\qw &\qw      &\qw &\qw &\qw    &\qw &\qw &\qw      &\qw &\qw &\qw     &\qw  &\ctrl{3}&\qw &\qw     &\qw &&\\
      &\ctrl{1}&\qw &\qw &\qw    &\qw &\qw &\ctrl{1}&\qw &&=&&&    &\ctrl{2}&\qw  &\qw     &\qw &\ctrl{2}&\qw  &\qw &\qw      &\qw &\qw &\qw    &\qw &\qw &\qw      &\qw &\qw &\ctrl{2}&\qw  &\qw     &\qw &\ctrl{2}&\qw &&\\
      &\targ   &\qw &    &\ldots &    &    &\targ   &\qw && &&&    &\qw     &\qw  &\qw     &\qw &\qw     &\qw  &\qw &\targ    &\qw &    &\ldots &    &    &\targ    &\qw &\qw &\qw     &\qw  &\qw     &\qw &\qw     &\qw &&\\
      &        &    &    &       &    &    &        &    && &&&\qA &\targ   &\qTi &\targ   &\qT &\targ   &\qTi &\qH &\ctrl{-1}&\qw &\qw &\qw    &\qw &\qw &\ctrl{-1}&\qH &\qT &\targ   &\qTi &\targ   &\qT &\targ   &\qw &|A\rangle &\\
    }
  }
  \caption{
	A temporary Toffoli construction with a net T-cost of 6.
	Invests an $|A\rangle$ state when computing the Toffoli, holds an ancilla qubit storing the logical-and of the controls until it's time to uncompute the Toffoli, then recovers the $|A\rangle$ state.
  }
  \label{fig:ancilla-temporary-toffoli}
\end{figure}





\section{AND-ing and Erasing}
\label{sec:temporary-and}

When the target of a Toffoli gate is a qubit in the $|0\rangle$ state, and the controls are in a computation basis state, the target qubit ends up storing the logical-and of the two controls.
We continue to say that the target stores the logical-and of the two controls even when they are under superposition, with the understanding that we mean ``within each computation basis state making up the superposition of the overall system".

In the textbook construction of the Toffoli gate, the three T gates on the control wires are implementing a controlled-S.
A controlled-S phases states by $i$ where both the control and the target are ON.
If the target of the Toffoli gate starts in the $|0\rangle$ state, then the target ends up ON when both controls are satisfied.
Therefore the target is ON in exactly the state where a phase factor of $i$ should be applied, and the controlled-S on the controls can be replaced by an unconditional S gate on the target after the Toffoli is finished.

A minor optimization which we apply is to note that the first two operations on the target are transitioning from the $|0\rangle$ state to the $|A\rangle$ state.
Instead of consuming an $|A\rangle$ state to perform the $T$ gate (and potentially performing a fixup S gate), we simply feed the $|A\rangle$ state we would have consumed directly into the circuit.

This construction defines our logical AND operation, which we draw as a wire emerging from two controls as shown in the left hand side of \autoref{fig:logical-and} (which also shows our construction for the operation).

\begin{figure}
  \resizebox{\linewidth}{!}{
    \Qcircuit @R=1.5em @C=0.7em {
      &x &&\ctrl{1} &\qw & x  && &&          &&\qw      &\qw  &\ctrl{2} &\qw &\qw      &\qw  &\qw &\qw &\qw &\\
      &y &&\ctrl{1} &\qw & y  &&=&&          &&\ctrl{1} &\qw  &\qw      &\qw &\ctrl{1} &\qw  &\qw &\qw &\qw &\\
      &  &&         &\qw & xy && &&|A\rangle &&\targ    &\qTi &\targ    &\qT &\targ    &\qTi &\qH &\qS &\qw &\\
    }
  }
  \caption{
	Control merging with T-cost of 4 (three T gates, one $|A\rangle$ state).
	The $|A\rangle$ state is fed directly into the circuit, instead of using an $H$ and $T$ gate on a $|0\rangle$ state (which would consume an $|A\rangle$ state anyways).
	The input $|A\rangle$ state can be recovered by using the un-merging construction in \autoref{fig:unmerge-recover-a}, reducing the net T cost of this circuit to 3.
	However, the un-merging construction in \autoref{fig:unmerge-free} is a better choice, despite not recovering the $|A\rangle$ state, because the total T-cost of merge+unmerge using \autoref{fig:unmerge-free} is 4+0=4 instead of 4+3-1=6.
  }
  \label{fig:logical-and}
\end{figure}

The logical AND construction consumes four $|A\rangle$ states, so its T-cost is 4.

To erase the ancilla produced by the logical AND, we use a technique we refer to as ``erasure".
We note that a Toffoli gate would obviously clear the value.
Then, since the target qubit can be discarded after clearing it, we are free to introduce a Hadamard gate and a measurement before discarding it.
We then hop the Hadamard over the Toffoli, transforming it into a CCZ operation.
The controls and targets of a CCZ are equivalent, so we rewrite the circuit into a CZC.
We then hop the Measurement over the CZC, turning a quantum control into a classical control.
These circuit moves turned the Toffoli gate into a CZ gate that we apply or not based on the outcome of the measreument
We ``Clifford-ized" the circuit by adding a Hadamard gate and a measurement.

We represent erasing a logical AND with a wire terminating into two controls, as shown in \autoref{fig:erase-and} which also shows the underlying construction.
The T-cost of erasure is 0.

\begin{figure}
  \resizebox{\linewidth}{!}{
    \Qcircuit @R=1.5em @C=0.7em {
      &x  &&\ctrl{1} &\qw &x && &&\qw &\qw    &\ctrl{1} &\qw \\
      &y  &&\ctrl{1} &\qw &y &&=&&\qw &\qw    &\gate{Z} &\qw \\
      &xy &&\qw      &    &  && &&\qH &\meter &\cw \cwx \bullet &    \\
    }
  }
  \caption{
	Control un-merging with T-cost of 0.
  }
  \label{fig:erase-and}
\end{figure}

The benefit of separating the logical-and-creating operation from the logical-and-erasing operation is that, as long as we have the ancilla storing the logical AND around, we can Cliffordize any Toffoli gate whose controls would be the two qubits we merged by replacing the Toffoli with a CNOT with the ancilla qubit as a control.
When the parts are simply placed next to each other, we end up with a construction exactly equivalent to the ones presented by Jones \cite{Jones2013}.
See \autoref{fig:merge-use-erase-toffoli}.

\begin{figure}
  \resizebox{\linewidth}{!}{
    \Qcircuit @R=1.5em @C=0.7em {
      &\ctrl{1} &\qw && &&\ctrl{1} &\qw      &\ctrl{1} &\qw && &&          &&\qw &\qw &\qw      &\qw  &\ctrl{2} &\qw &\qw      &\qw  &\ctrl{2} &\qw &\qw  &\qw      &\qw &\qw    &\ctrl{1}         &\qw \\
      &\ctrl{2} &\qw && &&\ctrl{1} &\qw      &\ctrl{1} &\qw && &&          &&\qw &\qw &\ctrl{1} &\qw  &\qw      &\qw &\ctrl{1} &\qw  &\qw      &\qw &\qw  &\qw      &\qw &\qw    &\gate{Z}         &\qw \\
      &         &    &&=&&         &\ctrl{1} &\qw      &    &&=&&|0\rangle &&\qH &\qT &\targ    &\qTi &\targ    &\qT &\targ    &\qTi &\targ    &\qH &\qSi &\ctrl{1} &\qH &\meter &\cw \cwx \bullet &    \\
      &\targ    &\qw && &&\qw      &\targ    &\qw      &\qw && &&          &&\qw &\qw &\qw      &\qw  &\qw      &\qw &\qw      &\qw  &\qw      &\qw &\qw  &\targ    &\qw &\qw    &\qw              &\qw \\
    }
  }
  \caption{
	Performing a Toffoli gate by computing, using, and erasing a logical-AND.
	Has a T-cost of 4, and is exactly equivalent to the construction from \cite{Jones2013}.
  }
  \label{fig:merge-use-erase-toffoli}
\end{figure}

In order to make savings, we need to keep the ancilla around for longer.
We need a situation like the one shown in \autoref{fig:paired-toffoli-to-logical-and}, where a single logical-AND computation replaces two Toffolis.

\begin{figure}
  \resizebox{\linewidth}{!}{
    \Qcircuit @R=1.5em @C=0.7em {
      &\ctrl{1} &\qw &\qw &\qw    &\qw &\qw &\ctrl{1} &\qw && &&\ctrl{1} &\qw      &\qw &\qw &\qw    &\qw &\qw &\qw      &\ctrl{1} &\qw & \\
      &\ctrl{2} &\qw &\qw &\qw    &\qw &\qw &\ctrl{2} &\qw && &&\ctrl{1} &\qw      &\qw &\qw &\qw    &\qw &\qw &\qw      &\ctrl{1} &\qw & \\
      &         &    &    &       &    &    &         &    &&=&&         &\ctrl{1} &\qw &\qw &\qw    &\qw &\qw &\ctrl{1} &\qw      &    & \\
      &\targ    &\qw &    &\ldots &    &    &\targ    &\qw && &&\qw      &\targ    &\qw &    &\ldots &    &    &\targ    &\qw      &\qw & \\
    }
  }
  \caption{
	When distant Toffolis share the same controls, a single logical-AND can Cliffordize both.
    The T-cost of computing and uncomputing the logical-AND is 4, so in effect this reduces the average T-cost of paired Toffolis to 2.
  }
  \label{fig:paired-toffoli-to-logical-and}
\end{figure}

Fortunately, this pattern of distant Toffolis sharing controls is very common in quantum circuits.
It occurs during uncomputation, and also in circuits that sweep back and forth across qubits like addition.


\section{T-optimized Circuit Constructions}
\label{sec:circuit-constructions}


\begin{figure}
  \resizebox{\linewidth}{!}{
    \Qcircuit @R=1.5em @C=0.7em {
      &x_k &&\targ    &\ctrl{1}&\qw      &\qw &\qw &\qw &\qw     &\qw &\qw &\qw &\qw    &\qw &\qw &\qw &\qw     &\qw &\qw &\qw      &\ctrl{1}&\targ    &\ctrl{1}&\qw &x_k     &&&&& \\
      &y_k &&\targ    &\ctrl{1}&\qw      &\qw &\qw &\qw &\qw     &\qw &\qw &\qw &\qw    &\qw &\qw &\qw &\qw     &\qw &\qw &\qw      &\ctrl{1}&\qw      &\targ   &\qw &&&(x+y)_k &&& \\
      &    &&         &        &\ctrl{1} &\qw &\qw &\qw &\qw     &\qw &\qw &\qw &\qw    &\qw &\qw &\qw &\qw     &\qw &\qw &\ctrl{1} &\qw     &         &        &    &&&        &&& \\
      &c_k &&\ctrl{-3}&\qw     &\targ    &\qw &\qw &    &c_{k+1} &    &    &    &\ldots &    &    &    &c_{k+1} &    &    &\targ    &\qw     &\ctrl{-3}&\qw     &\qw &c_k     &&&&& \\
    }
  }
  \caption{
	Inline full adder block with a T-cost of 4.
  }
  \label{fig:addition}
\end{figure}





\begin{figure}
  \centering
  \makebox[\linewidth]{
  }
  \caption{
	$n$-bit inline addition with T-cost of $8n + O(1)$ based on [[[cite]]].
  }
  \label{fig:inline-addition}
\end{figure}


\begin{figure}
  \centering
  \makebox[\linewidth]{
  }
  \caption{
	$n$-bit addition with T-cost of $6n + O(1)$.
  }
  \label{fig:ancilla-addition}
\end{figure}

\begin{figure}
  \resizebox{\linewidth}{!}{
    \Qcircuit @R=1.5em @C=0.7em {
      &    &\ctrl{2}          &\qw               &\qw &&     &&&     &\qS      &\qS      &\gate{Z}     &\ctrl{2}          &\qw               &\qw &&& &&& \lstick{x} &\ctrl{2} &\qw      &\qw &\ctrl{3} &\qw  &\qw      &\qw &\qw      &\qw  &\qw      &\qw &\rstick{x}         \\
      &    &\qw               &\ctrl{1}          &\qw &&\cong&&&     &\ctrl{-1}&\qS \qwx &\gate{Z}\qwx &\qw               &\ctrl{1}          &\qw &&&=&&& \lstick{y} &\qw      &\qw      &\qw &\qw      &\qw  &\ctrl{2} &\qw &\qw      &\qw  &\ctrl{1} &\qw &\rstick{y}         \\
      &    &\multigate{1}{+1} &\multigate{1}{+1} &\qw &&     &&&     &\qw      &\ctrl{-1}&\qw          &\multigate{1}{+1} &\multigate{1}{+1} &\qw &&& &&& \lstick{z} &\targ    &\ctrl{1} &\qw &\qw      &\qw  &\qw      &\qw &\ctrl{1} &\qw  &\targ    &\qw &\rstick{(x+y+z)_0} \\
      &\qO &\ghost{+1}        &\ghost{+1}        &\qw &&     &&& \qO &\qw      &\qw      &\ctrl{-2}    &\ghost{+1}        &\ghost{+1}        &\qw &&& &&& \qA        &\qw      &\targ    &\qT &\targ    &\qSi &\targ    &\qT &\targ    &\qTi &\qH      &\qw &\rstick{(x+y+z)_1} \\
    }
  }
  \caption{
	Temporary full adder.
  }
  \label{fig:temporary-ancilla-addition}
\end{figure}

Mention binary to unary as an example?


Show example doing an $n$-control NOT with $6n + O(1)$ or $3n + O(1)$ if temporary.

Mention that this trick works more generally. Anytime you have an ancilla that starts HT and ends Tdag H.


\section{Acknowledgements}

Austin.


\bibliographystyle{plain}
\bibliography{citations}

https://arxiv.org/pdf/quant-ph/9503016.pdf   ->   the paired Toffoli construction that uses 4T

https://arxiv.org/pdf/cond-mat/9409111.pdf   -> shows CCNOT can be done with three 2-qubit interactions, ignores single-qubit???


https://arxiv.org/pdf/1308.4134.pdf -> proof of 7T gate minimum with no ancilla

https://arxiv.org/pdf/1212.5069.pdf  ->   does a single Toffoli with four T gates by turning one of the controls classical

https://arxiv.org/pdf/1210.0974.pdf  ->   a phase-neglected 4T gate construction that only uses a single layer of T gates


Quantum Computation and Quantum Information -> shows standard decomposition into 7T gates. But do they say it's optimal?

https://arxiv.org/pdf/1206.0758.pdf -> just an example of the 7T construction;;; probablynot relevant



https://arxiv.org/pdf/1706.05113.pdf   has a ctrl-add with 21N + O(1). This technique+commutators achieves 8N.


https://pdfs.semanticscholar.org/8b84/d1bc2928922937a16205f9d8c925ff76689b.pdf     has a pretty good ripple carry








As an example, suppose we wish to phase the amplitude of each computational basis state $|k\rangle$ of some register $R_{\text{in}}$ by a number of radians computed by some function $f(k)$.
We want to apply the operation $U_f = \exp\left( i \sum_k f(k) |k\rangle \langle k| \right)$.
A naive method to perform this task is to 1) construct a classical reversible circuit $C$ for approximating $f$, 2) decompose the gates used by $C$ into a quantum gate set so that the circuit can be run on a quantum computer, 3) prepare a fresh register $R_\text{out}$ and use the circuit $C$ on it and $R_\text{in}$ to produce a binary fixed-point 2s-complement output stored in $R_{\text{out}}$, and then 4) phase by an amount proportional to $R_\text{out}$ by applying a $Z^{\theta 2^r}$ operation with appropriate scaling factor $\theta$ to each bit position $r$ within $R_\text{out}$.

The problem with this naive method is that it produces garbage.
The register $R_\text{out}$ ends up entangled with $R_\text{in}$.
If the garbage isn't cleaned up, this decoheres $R_\text{in}$; defeating the purpose of using a quantum computer.
The method we use to clean up the garbage is to uncompute $R_\text{out}$ by running $C$ in reverse.
This destroys the garbage, without undoing the phasing effects.
See \autoref{fig:phase-by-function}.

\begin{figure}
  \resizebox{\linewidth}{!}{
    \Qcircuit @R=2.0em @C=0.7em {
      &\gate{\exp \big(i \sum_k f(k)|k\rangle\langle k|\big)}&\qw &&=&&&                      &&&\qw &\gate{\text{input k}}         &\qw               &\gate{\text{input k}}         &\qw &\\
      &                                                      &    && &&&|0\rangle^{\otimes p} &&&\qw &\gate{0 \rightarrow \left\lfloor \frac{f(k)}{2^{p+1} \pi} \right\rfloor \pmod{2^p}}\qwx &\gate{\text{Grad}^{2^{-p}}}&\gate{\left\lfloor \frac{f(k)}{2^{p+1} \pi} \right\rfloor \pmod{2^p} \rightarrow 0}\qwx &\qw &\\
    }
  }
  \caption{
	Example use of uncomputation: phasing by any computable function $f$, up to a precision of $\epsilon = 2 \pi / 2^p$ radians, by computing an approximation of $f$ under superposition then applying a phase gradient to the output before uncomputing it.
    \\
	The gradient operation is implemented by a column of Z-axis rotations, one for each of the $p$ qubits in the temporary output register (the Z rotation applied to the $b$'th qubit is the gate $Z^{2^{-b}}$).
  }
  \label{fig:phase-by-function}
\end{figure}

\end{document}
