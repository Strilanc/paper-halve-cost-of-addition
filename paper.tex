\documentclass[twocolumn,longbibliography]{quantumarticle-customized}
\usepackage{amsmath}
\usepackage{graphicx}
\usepackage[pdfpagelabels,pdftex,bookmarks,breaklinks]{hyperref}
\usepackage{tikz}
\usepackage[all]{hypcap}
\hypersetup{colorlinks,citecolor=blue,urlcolor=blue,linkcolor=blue}

\input{Qcircuit}
\newcommand{\qH}{\gate{H}}
\newcommand{\qT}{\gate{T}}
\newcommand{\qTi}{\gate{T^\dagger}}
\newcommand{\qS}{\gate{S}}
\newcommand{\qSi}{\gate{S^\dagger}}
\newcommand{\qA}{\lstick{|A\rangle}}
\newcommand{\qO}{\lstick{|0\rangle}}

\title{T gates, temporary Toffolis, and halving the cost of quantum computation}
\author{Craig Gidney}
\affiliation{Google, Santa Barbara, CA 93117, USA}
\email{craiggidney@google.com}

\def\sectionautorefname{Section}

\begin{document}
\maketitle

\begin{abstract}
We show how to compute the logical-and of two controls using four T gates, and uncompute it using no T gates.
This construction halves the average T-cost of temporary Toffoli gates.
A temporary Toffoli gate is one which will be uncomputed later (typically by a second Toffoli gate).

We use our temporary logical-and construction to produce quantum circuits with T gate counts lower than previously reported.
T gates dominate the cost of surface-code-based error corrected quantum computation.
Temporary Toffolis are used in many common quantum circuits, such as addition.
As such, the optimization we present in this paper are widely applicable and represent significant savings in the projected costs of quantum computation.
For example, by applying previously known optimizations and the ones presented in this paper, we improve on the projected T-cost of Shor's algorithm from [[[[cite]]] by 70\% and the projected minimum time down by 66%.

We demonstrate an $n$-bit adder with T-cost $4n + O(1)$, a controlled $n$-bit adder with T-cost $8n + O(1)$, an $n$-controlled NOT with T-cost $4n + O(1)$, and a temporary pop-count with T-cost $6n + O(1)$.

We discuss investing $|A\rangle$ states, a fallback optimization technique that may reduce the number of T gates needed to compute quantum circuits not amenable to temporary ANDs.
\end{abstract}

\section{Introduction} \label{sec:introduction}

The textbook construction of a Toffoli gate uses seven T gates [[[cite textbook]]] \autoref{fig:textbook-toffoli} and eight Clifford gates.
On an error corrected quantum computer that uses the surface code, T gates are XXX orders of magnitude more expensive than Clifford gates [[[cite]]].
Because of this disparity, optimizing error corrected quantum circuits largely consists of reducing their T-cost (the number of T gates used, or rather the number of $|A\rangle = \frac{1}{\sqrt{2}} |0\rangle + \frac{1+i}{2} |1\rangle$ states which must be distilled in order to perform those T gates).

\begin{figure}
  \resizebox{\linewidth}{!}{
    \Qcircuit @R=1.5em @C=0.7em {
      &\ctrl{1}&\qw & &   & & &\ctrl{1}&\qw  &\ctrl{1}&\qT &\qw     &\qw  &\ctrl{2}&\qw &\qw     &\qw  &\ctrl{2}&\qw &\qw \\
      &\ctrl{1}&\qw & & = & & &\targ   &\qTi &\targ   &\qT &\ctrl{1}&\qw  &\qw     &\qw &\ctrl{1}&\qw  &\qw     &\qw &\qw \\
      &\targ   &\qw & &   & & &\qw     &\qw  &\qH     &\qT &\targ   &\qTi &\targ   &\qT &\targ   &\qTi &\targ   &\qH &\qw \\
    }
  }
  \caption{
	Textbook Toffoli construction [[[[cite]]]] from eight Clifford gates and seven T gates.
  }
  \label{fig:textbook-toffoli}
\end{figure}

The textbook Toffoli construction is optimal in some highly constrained situations (e.g. not allowed to use any ancilla qubits and not allowed to merge T gates across operations [[[cite]]]), but is not optimal in general.
For example, when two Toffoli gates share a control, applying the textbook construction to each places two $T$ gates on the shared control.
These two $T$ gates can be merged into a single $S$ gate, thereby reducing the T-cost of each Toffoli from 7 to 6.

More commonly, Toffoli gates tend to come in matched pairs where one is uncomputing the effect of the other.
When this occurs, the three $T$ gates on the control qubits of the textbook construction are no longer necessary \autoref{fig:bad-phase-toffoli}.
Removing those $T$ gates introduces phase errors, but the second Toffoli gate will uncompute those errors while uncomputing the state permutation.
This technique reduces the T-cost of paired Toffolis from 7 to 4.

\begin{figure}
  \resizebox{\linewidth}{!}{
    \Qcircuit @R=1.5em @C=0.7em {
      &\ctrl{1} &\qw & &       & & &\ctrl{1}     &\ctrl{1} &\qw & &   & & &\qw &\qw &\qw      &\qw  &\ctrl{2} &\qw &\qw      &\qw  &\qw &\qw \\
      &\ctrl{1} &\qw & & \cong & & &\qS          &\ctrl{1} &\qw & & = & & &\qw &\qw &\ctrl{1} &\qw  &\qw      &\qw &\ctrl{1} &\qw  &\qw &\qw \\
      &\targ    &\qw & &       & & &\gate{Z}\qwx &\targ    &\qw & &   & & &\qH &\qT &\targ    &\qTi &\targ    &\qT &\targ    &\qTi &\qH &\qw \\
    }
  }
  \caption{
	Existing temporary Toffoli construction with 4 T gates.
  }
  \label{fig:bad-phase-toffoli}
\end{figure}

In [[[cite]]], Jones improves on the paired construction by demonstrating how to reduce the T-cost of any Toffoli to 4.
Jones does this by using a temporary ancilla and applying an erasure technique. [[[figures]]]

We improve on Jones by separating their construction into three pieces (merge, use, erase), and showing how to apply these pieces to circuit construction tasks.
In particular, we reduce the average T-cost of paired Toffolis from 4 to 2.
Many common circuits (e.g. addition) have paired Toffolis, and so this optimization can halve their T-cost.




\section{Temporary ANDs}


See \autoref{fig:inline-toffoli}.




\begin{figure}
  \resizebox{\linewidth}{!}{
    \Qcircuit @R=1.5em @C=0.7em {
      &\ctrl{1} &\qw & &   & & &     &\qw &\qw &\qw      &\qw  &\ctrl{3} &\qw &\qw      &\qw  &\qw &\qw  &\qw       &\qw &\qw    &\ctrl{1}       &\qw \\
      &\ctrl{1} &\qw & & = & & &     &\qw &\qw &\ctrl{2} &\qw  &\qw      &\qw &\ctrl{2} &\qw  &\qw &\qw  &\qw       &\qw &\qw    &\gate{Z}       &\qw \\
      &\targ    &\qw & &   & & &     &\qw &\qw &\qw      &\qw  &\qw      &\qw &\qw      &\qw  &\qw &\qw  &\targ     &\qw &\qw    &\qw\cwx        &\qw \\
      &         &    & &   & & & \qO &\qH &\qT &\targ    &\qTi &\targ    &\qT &\targ    &\qTi &\qH &\qSi &\ctrl{-1} &\qH &\meter &\cw\cwx\bullet & \\
    }
  }
  \caption{
	Toffoli construction with 4 T gates based on measurement teleportation.
  }
  \label{fig:ancilla-toffoli}
\end{figure}

\begin{figure}
  \resizebox{\linewidth}{!}{
    \Qcircuit @R=1.5em @C=0.7em {
      &x &&\ctrl{1} &\qw & x  && &&          &&\qw      &\qw  &\ctrl{2} &\qw &\qw      &\qw  &\ctrl{2} &\qw &\qw &\qw &\\
      &y &&\ctrl{1} &\qw & y  &&=&&          &&\ctrl{1} &\qw  &\qw      &\qw &\ctrl{1} &\qw  &\qw      &\qw &\qw &\qw &\\
      &  &&         &\qw & xy && &&|A\rangle &&\targ    &\qTi &\targ    &\qT &\targ    &\qTi &\targ    &\qH &\qS &\qw &\\
    }
  }
  \caption{
	Control merging with T-cost of 4 (three T gates, one $|A\rangle$ state).
	The $|A\rangle$ state is fed directly into the circuit, instead of using an $H$ and $T$ gate on a $|0\rangle$ state (which would consume an $|A\rangle$ state anyways).
	The input $|A\rangle$ state can be recovered by using the un-merging construction in \autoref{fig:unmerge-recover-a}, reducing the net T cost of this circuit to 3.
	However, the un-merging construction in \autoref{fig:unmerge-free} is a better choice, despite not recovering the $|A\rangle$ state, because the total T-cost of merge+unmerge using \autoref{fig:unmerge-free} is 4+0=4 instead of 4+3-1=6.
  }
  \label{fig:ancilla-temporary-toffoli}
\end{figure}

\begin{figure}
  \resizebox{\linewidth}{!}{
    \Qcircuit @R=1.5em @C=0.7em {
      &x_k &&\targ    &\ctrl{1}&\qw      &\qw &\qw &\qw &\qw     &\qw &\qw &\qw &\qw    &\qw &\qw &\qw &\qw     &\qw &\qw &\qw      &\ctrl{1}&\targ    &\ctrl{1}&\qw &x_k     &&&&& \\
      &y_k &&\targ    &\ctrl{1}&\qw      &\qw &\qw &\qw &\qw     &\qw &\qw &\qw &\qw    &\qw &\qw &\qw &\qw     &\qw &\qw &\qw      &\ctrl{1}&\qw      &\targ   &\qw &&&(x+y)_k &&& \\
      &    &&         &        &\ctrl{1} &\qw &\qw &\qw &\qw     &\qw &\qw &\qw &\qw    &\qw &\qw &\qw &\qw     &\qw &\qw &\ctrl{1} &\qw     &         &        &    &&&        &&& \\
      &c_k &&\ctrl{-3}&\qw     &\targ    &\qw &\qw &    &c_{k+1} &    &    &    &\ldots &    &    &    &c_{k+1} &    &    &\targ    &\qw     &\ctrl{-3}&\qw     &\qw &c_k     &&&&& \\
    }
  }
  \caption{
	Inline full adder block with a T-cost of 4.
  }
  \label{fig:addition}
\end{figure}

\begin{figure}
  \resizebox{\linewidth}{!}{
    \Qcircuit @R=1.5em @C=0.7em {
      &x  &&\ctrl{1} &\qw &x && &&\qw &\qw    &\ctrl{1} &\qw \\
      &y  &&\ctrl{1} &\qw &y &&=&&\qw &\qw    &\gate{Z} &\qw \\
      &xy &&\qw      &    &  && &&\qH &\meter &\cw \cwx \bullet &    \\
    }
  }
  \caption{
	Control un-merging with T-cost of 0.
  }
  \label{fig:unmerge-free}
\end{figure}

\begin{figure}
  \resizebox{\linewidth}{!}{
    \Qcircuit @R=1.5em @C=0.7em {
      &\ctrl{1} &\qw && &&\ctrl{1} &\qw      &\ctrl{1} &\qw && &&          &&\qw &\qw &\qw      &\qw  &\ctrl{2} &\qw &\qw      &\qw  &\ctrl{2} &\qw &\qw  &\qw      &\qw &\qw    &\ctrl{1}         &\qw \\
      &\ctrl{2} &\qw && &&\ctrl{1} &\qw      &\ctrl{1} &\qw && &&          &&\qw &\qw &\ctrl{1} &\qw  &\qw      &\qw &\ctrl{1} &\qw  &\qw      &\qw &\qw  &\qw      &\qw &\qw    &\gate{Z}         &\qw \\
      &         &    &&=&&         &\ctrl{1} &\qw      &    &&=&&|0\rangle &&\qH &\qT &\targ    &\qTi &\targ    &\qT &\targ    &\qTi &\targ    &\qH &\qSi &\ctrl{1} &\qH &\meter &\cw \cwx \bullet &    \\
      &\targ    &\qw && &&\qw      &\targ    &\qw      &\qw && &&          &&\qw &\qw &\qw      &\qw  &\qw      &\qw &\qw      &\qw  &\qw      &\qw &\qw  &\targ    &\qw &\qw    &\qw              &\qw \\
    }
  }
  \caption{
	Toffoli gate from control merging and un-merging with T-cost 4.
	Equivalent to the construction from [[[[cite]]]
  }
  \label{fig:ancilla-temporary-toffoli}
\end{figure}

\begin{figure}
  \resizebox{\linewidth}{!}{
    \Qcircuit @R=1.5em @C=0.7em {
      &\ctrl{1} &\qw &\qw &\qw    &\qw &\qw &\ctrl{1} &\qw && &&\ctrl{1} &\qw      &\qw &\qw &\qw    &\qw &\qw &\qw      &\ctrl{1} &\qw & \\
      &\ctrl{2} &\qw &\qw &\qw    &\qw &\qw &\ctrl{2} &\qw && &&\ctrl{1} &\qw      &\qw &\qw &\qw    &\qw &\qw &\qw      &\ctrl{1} &\qw & \\
      &         &    &    &       &    &    &         &    &&=&&         &\ctrl{1} &\qw &\qw &\qw    &\qw &\qw &\ctrl{1} &\qw      &    & \\
      &\targ    &\qw &    &\ldots &    &    &\targ    &\qw && &&\qw      &\targ    &\qw &    &\ldots &    &    &\targ    &\qw      &\qw & \\
    }
  }
  \caption{
	Toffoli gate pair from control merging and un-merging with T-cost 4.
	Equivalent to the construction from [[[[cite]]].
  }
  \label{fig:ancilla-temporary-toffoli}
\end{figure}

\begin{figure}
  \resizebox{\linewidth}{!}{
    \Qcircuit @R=1.5em @C=0.7em {
      &\ctrl{1} &\qw & &       &&&     &\ctrl{1} &\ctrl{1} &\qw  &&&   & & & \lstick{x} &\qw      &\qw  &\ctrl{3} &\qw &\qw      &\qw  &\qw &\qw       &\qw &\rstick{x} \\
      &\ctrl{1} &\qw & & \cong &&&     &\ctrl{2} &\qS      &\qw  &&& = & & & \lstick{y} &\ctrl{2} &\qw  &\qw      &\qw &\ctrl{2} &\qw  &\qw &\qw       &\qw &\rstick{y} \\
      &\targ    &\qw & &       &&&     &\targ    &\qw      &\qw  &&&   & & & \lstick{t} &\qw      &\qw  &\qw      &\qw &\qw      &\qw  &\qw &\targ     &\qw &\rstick{t \oplus (x \land y)} \\
      &         &    & &       &&& \qO &\targ    &\qw      &\qw  &&&   & & & \qA        &\targ    &\qTi &\targ    &\qT &\targ    &\qTi &\qH &\ctrl{-1} &\qw &\rstick{x \land y} \\
    }
  }
  \caption{
	Our temporary Toffoli construction, which uses an $|A\rangle$ ancilla and 3 T gates.
	The $|A\rangle$ state is recovered when uncomputing the temporary Toffoli.
  }
  \label{fig:ancilla-temporary-toffoli}
\end{figure}

\begin{figure}
  \centering
  \makebox[\linewidth]{
  }
  \caption{
	$n$-bit inline addition with T-cost of $8n + O(1)$ based on [[[cite]]].
  }
  \label{fig:inline-addition}
\end{figure}

\begin{figure}
  \resizebox{\linewidth}{!}{
    \Qcircuit @R=1.5em @C=0.7em {
      &x  &&\ctrl{1} &\qw &x && && \qw  &\qw &\ctrl{2} &\qw &\qw      &\qw  &\ctrl{2} &\qw &\qw      &\qw &           \\
      &y  &&\ctrl{1} &\qw &y &&=&& \qw  &\qw &\qw      &\qw &\ctrl{1} &\qw  &\qw      &\qw &\ctrl{1} &\qw &           \\
      &xy &&\qw      &    &  && && \qSi &\qH &\targ    &\qT &\targ    &\qTi &\targ    &\qT &\targ    &\qw &|A \rangle \\
    }
  }
  \caption{
	A non-optimal unmerging circuit that uses three $T$ gates, but recovers an $|A\rangle$ state.
	Avoids measurement, but has a net T-cost of two (which is worse than \autoref{fig:unmerge-free}'s T-cost of zero).
  }
  \label{fig:unmerge-recover-a}
\end{figure}

\begin{figure}
  \centering
  \makebox[\linewidth]{
  }
  \caption{
	$n$-bit addition with T-cost of $6n + O(1)$.
  }
  \label{fig:ancilla-addition}
\end{figure}

\begin{figure}
  \resizebox{\linewidth}{!}{
    \Qcircuit @R=1.5em @C=0.7em {
      &    &\ctrl{2}          &\qw               &\qw &&     &&&     &\qS      &\qS      &\gate{Z}     &\ctrl{2}          &\qw               &\qw &&& &&& \lstick{x} &\ctrl{2} &\qw      &\qw &\ctrl{3} &\qw  &\qw      &\qw &\qw      &\qw  &\qw      &\qw &\rstick{x}         \\
      &    &\qw               &\ctrl{1}          &\qw &&\cong&&&     &\ctrl{-1}&\qS \qwx &\gate{Z}\qwx &\qw               &\ctrl{1}          &\qw &&&=&&& \lstick{y} &\qw      &\qw      &\qw &\qw      &\qw  &\ctrl{2} &\qw &\qw      &\qw  &\ctrl{1} &\qw &\rstick{y}         \\
      &    &\multigate{1}{+1} &\multigate{1}{+1} &\qw &&     &&&     &\qw      &\ctrl{-1}&\qw          &\multigate{1}{+1} &\multigate{1}{+1} &\qw &&& &&& \lstick{z} &\targ    &\ctrl{1} &\qw &\qw      &\qw  &\qw      &\qw &\ctrl{1} &\qw  &\targ    &\qw &\rstick{(x+y+z)_0} \\
      &\qO &\ghost{+1}        &\ghost{+1}        &\qw &&     &&& \qO &\qw      &\qw      &\ctrl{-2}    &\ghost{+1}        &\ghost{+1}        &\qw &&& &&& \qA        &\qw      &\targ    &\qT &\targ    &\qSi &\targ    &\qT &\targ    &\qTi &\qH      &\qw &\rstick{(x+y+z)_1} \\
    }
  }
  \caption{
	Temporary full adder.
  }
  \label{fig:temporary-ancilla-addition}
\end{figure}

Mention binary to unary as an example?


Show example doing an $n$-control NOT with $6n + O(1)$ or $3n + O(1)$ if temporary.

Mention that this trick works more generally. Anytime you have an ancilla that starts HT and ends Tdag H.


\section{Acknowledgements}

Austin.


\bibliographystyle{plain}
\bibliography{citations}

\cite{barenco1995}
https://arxiv.org/pdf/quant-ph/9503016.pdf   ->   the paired Toffoli construction that uses 4T

https://arxiv.org/pdf/cond-mat/9409111.pdf   -> shows CCNOT can be done with three 2-qubit interactions, ignores single-qubit???


https://arxiv.org/pdf/1308.4134.pdf -> proof of 7T gate minimum with no ancilla

https://arxiv.org/pdf/1212.5069.pdf  ->   does a single Toffoli with four T gates by turning one of the controls classical

https://arxiv.org/pdf/1210.0974.pdf  ->   a phase-neglected 4T gate construction that only uses a single layer of T gates


Quantum Computation and Quantum Information -> shows standard decomposition into 7T gates. But do they say it's optimal?

https://arxiv.org/pdf/1206.0758.pdf -> just an example of the 7T construction;;; probablynot relevant



https://arxiv.org/pdf/1706.05113.pdf   has a ctrl-add with 21N + O(1). This technique+commutators achieves 8N.


https://pdfs.semanticscholar.org/8b84/d1bc2928922937a16205f9d8c925ff76689b.pdf     has a pretty good ripple carry

\end{document}
